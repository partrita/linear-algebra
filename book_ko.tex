% Options for packages loaded elsewhere
\PassOptionsToPackage{unicode}{hyperref}
\PassOptionsToPackage{hyphens}{url}
\PassOptionsToPackage{dvipsnames,svgnames,x11names}{xcolor}
\documentclass[
  12pt,
  a4paper,
]{article}
\usepackage{xcolor}
\usepackage[a4paper,margin=2.5cm,heightrounded]{geometry}
\usepackage{amsmath,amssymb}
\setcounter{secnumdepth}{5}
\usepackage{iftex}
\ifPDFTeX
  \usepackage[T1]{fontenc}
  \usepackage[utf8]{inputenc}
  \usepackage{textcomp} % provide euro and other symbols
\else % if luatex or xetex
  \usepackage{unicode-math} % this also loads fontspec
  \usepackage{kotex} % Added for Korean support
  \defaultfontfeatures{Scale=MatchLowercase}
  \defaultfontfeatures[\rmfamily]{Ligatures=TeX,Scale=1}
\fi
\usepackage{lmodern}
\ifPDFTeX\else
  % xetex/luatex font selection
  % kotex will handle font selection, so we can comment these out.
  % \setmainfont[]{Noto Serif}
  % \setmonofont[]{Noto Sans Mono}
\fi
% Use upquote if available, for straight quotes in verbatim environments
\IfFileExists{upquote.sty}{\usepackage{upquote}}{}
\IfFileExists{microtype.sty}{% use microtype if available
  \usepackage[]{microtype}
  \UseMicrotypeSet[protrusion]{basicmath} % disable protrusion for tt fonts
}{}
\makeatletter
\@ifundefined{KOMAClassName}{% if non-KOMA class
  \IfFileExists{parskip.sty}{%
    \usepackage{parskip}
  }{% else
    \setlength{\parindent}{0pt}
    \setlength{\parskip}{6pt plus 2pt minus 1pt}}
}{% if KOMA class
  \KOMAoptions{parskip=half}}
\makeatother
\setlength{\emergencystretch}{3em} % prevent overfull lines
\providecommand{\tightlist}{%
  \setlength{\itemsep}{0pt}\setlength{\parskip}{0pt}}
\usepackage{amsmath}   % already pulled in by Pandoc, but safe
\renewenvironment{equation}{\begin{equation*}}{\end{equation*}}
\usepackage{microtype}
\newcommand{\docversion}{0.1.1}
\setcounter{secnumdepth}{0}
\setcounter{tocdepth}{2} % keep TOC depth as desired
\makeatletter
\renewcommand{\sectionmark}[1]{\markboth{#1}{}}
\renewcommand{\subsectionmark}[1]{\markright{#1}}
\makeatother
\usepackage{fancyhdr}
\fancyhf{}%
\fancyhead[LE,RO]{\thepage}%
\fancyhead[LO,RE]{\nouppercase{\leftmark}}%
\fancyfoot[C]{버전 \docversion}%
\renewcommand{\headrulewidth}{0.4pt}%
\setlength{\headheight}{13.6pt}%
}
\usepackage{etoolbox}
\AtBeginDocument{\thispagestyle{plain}\pagestyle{content}}
\let\oldsection\section
\renewcommand{\section}{\clearpage\oldsection}
\usepackage{bookmark}
\IfFileExists{xurl.sty}{\usepackage{xurl}}{} % add URL line breaks if available
\urlstyle{same}
\hypersetup{
  pdftitle={선형대수학 소책자},
  pdfauthor={Duc-Tam Nguyen},
  colorlinks=true,
  linkcolor={blue},
  filecolor={Maroon},
  citecolor={Blue},
  urlcolor={blue},
  pdfcreator={LaTeX via pandoc}}

\title{선형대수학 소책자}
\usepackage{etoolbox}
\makeatletter
\providecommand{\subtitle}[1]{% add subtitle to \maketitle
  \apptocmd{\@title}{\par {\large #1 \par}}{}{}
}
\makeatother
\subtitle{버전 0.1.1}
\author{Duc-Tam Nguyen}
\date{\today}

\begin{document}
\maketitle

{
\hypersetup{linkcolor=}
\setcounter{tocdepth}{2}
\tableofcontents
}
\section{1장. 벡터}\label{chapter-1-vectors}

\subsection{1.1 스칼라와 벡터}\label{11-scalars-and-vectors}

스칼라는 단일 수량으로, 보통 실수 집합 \(\mathbb{R}\)에서 가져온 값입니다. 스칼라는 산술의 기본 구성 요소입니다: 더하고, 빼고, 곱하고, 0이 아닌 경우 나눌 수 있습니다. 선형대수학에서 스칼라는 계수, 스케일링 팩터, 그리고 벡터나 행렬과 같은 더 큰 구조의 항목 역할을 합니다. 스칼라는 더 복잡한 객체를 측정하고 결합하는 가중치를 제공합니다. 벡터는 스칼라의 순서 있는 모음으로, 행이나 열로 배열됩니다. 스칼라가 실수일 때, 벡터는 \emph{실수} \(n\)-차원 공간에 속한다고 하며, 다음과 같이 씁니다.

\[\mathbb{R}^n = \{ (x_1, x_2, \dots, x_n) \mid x_i \in \mathbb{R} \}.\]

\(\mathbb{R}^n\)의 원소는 차원 \(n\)의 벡터 또는 \(n\)-벡터라고 불립니다. 숫자 \(n\)은 벡터 공간의 차원이라고 합니다. 따라서 \(\mathbb{R}^2\)는 모든 순서 있는 실수 쌍의 공간이고, \(\mathbb{R}^3\)는 모든 순서 있는 세 쌍의 공간입니다.

예제 1.1.1.

\begin{itemize}
\item
  2차원 벡터: \((3, -1) \in \mathbb{R}^2\).
\item
  3차원 벡터: \((2, 0, 5) \in \mathbb{R}^3\).
\item
  1차원 벡터: \((7) \in \mathbb{R}^1\), 이는 스칼라 \$7\$ 자체에 해당합니다.
\end{itemize}

벡터는 종종 행렬 곱셈에서의 역할을 강조하기 위해 열 형태로 수직으로 씁니다:

\[\mathbf{v} = \begin{bmatrix}
2 \\
0 \\
5 \end{bmatrix} \in \mathbb{R}^3.\]

수직 레이아웃은 선형 결합을 고려하거나 행렬에 벡터를 곱할 때 구조를 더 명확하게 만듭니다.

\subsubsection{기하학적 해석}\label{geometric-interpretation}

\(\mathbb{R}^2\)에서 벡터 \((x_1, x_2)\)는 원점 \((0,0)\)에서 시작하여 점 \((x_1, x_2)\)에서 끝나는 화살표로 시각화할 수 있습니다. 길이는 원점으로부터의 거리에 해당하고, 방향은 평면에서의 방향을 나타냅니다. \(\mathbb{R}^3\)에서는 동일한 그림이 3차원으로 확장됩니다: 벡터는 원점에서 \((x_1, x_2, x_3)\)까지의 화살표입니다. 3차원을 넘어서는 직접적인 시각화는 더 이상 불가능하지만, 벡터의 대수적 규칙은 동일하게 유지됩니다. \(\mathbb{R}^{10}\)에서 벡터를 그릴 수는 없지만, 2차원이나 3차원 벡터와 마찬가지로 덧셈, 스케일링, 변환에 대해 정확히 동일하게 동작합니다. 이러한 추상적인 관점 덕분에 선형대수학은 데이터가 종종 매우 고차원 공간에 존재하는 데이터 과학, 물리학, 기계 학습에 적용될 수 있습니다. 따라서 벡터는 세 가지 보완적인 방식으로 간주될 수 있습니다:

\begin{enumerate}
\def\labelenumi{\arabic{enumi}.}
\item
  좌표로 설명되는 공간의 한 점.
\item
  방향과 길이로 설명되는 변위 또는 화살표.
\item
  기하학과 무관한 대수적 규칙을 따르는 속성을 가진 벡터 공간의 추상적 요소.
\end{enumerate}

\subsubsection{표기법}\label{notation}

\begin{itemize}
\item
  벡터는 굵은 소문자로 씁니다: \(\mathbf{v}, \mathbf{w}, \mathbf{x}\).
\item
  벡터 \(\mathbf{v}\)의 \emph{i}번째 항목은 \(v_i\)로 쓰며, 인덱스는 1에서 시작합니다.
\item
  \(\mathbb{R}\) 상의 모든 \emph{n}차원 벡터 집합은 \(\mathbb{R}^n\)으로 표기합니다.
\item
  달리 명시되지 않는 한 열 벡터가 기본 형식입니다.
\end{itemize}

\subsubsection{왜 여기서 시작하는가?}\label{why-begin-here}

스칼라와 벡터는 선형대수학의 원자를 형성합니다. 우리가 만들 모든 구조(벡터 공간, 선형 변환, 행렬, 고유값)는 수와 순서 있는 수의 모음이라는 기본 개념에 의존합니다. 벡터를 이해하면 덧셈 및 스칼라 곱셈과 같은 연산을 정의한 다음 부분 공간, 기저 및 좌표계로 일반화할 수 있습니다. 결국 이 프레임워크는 기하학, 계산 및 데이터에 강력한 응용 프로그램을 갖춘 완전한 선형대수학 이론으로 성장합니다.

\subsubsection{연습문제 1.1}\label{exercises-11}

\begin{enumerate}
\def\labelenumi{\arabic{enumi}.}
\item
  \(\mathbb{R}^2\)에서 세 개의 다른 벡터를 쓰고 원점에서 시작하는 화살표로 스케치하십시오. 좌표를 명시적으로 식별하십시오.
\item
  \(\mathbb{R}^4\)에서 벡터의 예를 드십시오. 직접 시각화할 수 있습니까? 고차원 시각화가 어려운 이유를 설명하십시오.
\item
  \(\mathbf{v} = (4, -3, 2)\)라고 합시다. \(\mathbf{v}\)를 열 형식으로 쓰고 \(v_1, v_2, v_3\)를 명시하십시오.
\item
  어떤 의미에서 집합 \(\mathbb{R}^1\)은 선과 벡터 공간 둘 다입니까? 예제를 들어 설명하십시오.
\item
  벡터 \(\mathbf{u} = (1,1,\dots,1) \in \mathbb{R}^n\)을 고려하십시오. \(n\)이 클 때 이 벡터의 특별한 점은 무엇입니까? 응용 분야에서 무엇을 나타낼 수 있습니까?
\end{enumerate}

\subsection{1.2 벡터 덧셈과 스칼라 곱셈}\label{12-vector-addition-and-scalar-multiplication}

선형대수학의 벡터는 정적인 객체가 아닙니다. 그들의 힘은 우리가 수행할 수 있는 연산에서 나옵니다. 두 가지 기본 연산이 벡터 공간의 구조를 정의합니다: 덧셈과 스칼라 곱셈. 이러한 연산은 전체 주제를 뒷받침하는 간단하지만 광범위한 규칙을 만족합니다.

\subsubsection{벡터 덧셈}\label{vector-addition}

동일한 차원의 두 벡터가 주어지면, 그 합은 해당 항목을 더하여 얻습니다. 공식적으로, 만약

\[\mathbf{u} = (u_1, u_2, \dots, u_n), \quad
\mathbf{v} = (v_1, v_2, \dots, v_n),\]

이면 그 합은

\[\mathbf{u} + \mathbf{v} = (u_1+v_1, u_2+v_2, \dots, u_n+v_n).\]

예제 1.2.1.\\
\(\mathbf{u} = (2, -1, 3)\)이고 \(\mathbf{v} = (4, 0, -5)\)라고 합시다. 그러면

\[\mathbf{u} + \mathbf{v} = (2+4, -1+0, 3+(-5)) = (6, -1, -2).\]

기하학적으로 벡터 덧셈은 \emph{평행사변형 법칙}에 해당합니다. 두 벡터를 원점에서 시작하는 화살표로 그리면, 한 벡터의 꼬리를 다른 벡터의 머리에 놓으면 합이 생성됩니다. 그들이 형성하는 평행사변형의 대각선은 결과 벡터를 나타냅니다.

\subsubsection{스칼라 곱셈}\label{scalar-multiplication}

벡터에 스칼라를 곱하면 벡터의 방향을 유지하면서 벡터를 늘리거나 줄입니다. 단, 스칼라가 음수이면 벡터도 반전됩니다. 만약 \(c \in \mathbb{R}\)이고

\[\mathbf{v} = (v_1, v_2, \dots, v_n),\]

이면

\[c \mathbf{v} = (c v_1, c v_2, \dots, c v_n).\]

예제 1.2.2.\\
\(\mathbf{v} = (3, -2)\)이고 \(c = -2\)라고 합시다. 그러면

\[c\mathbf{v} = -2(3, -2) = (-6, 4).\]

이것은 벡터를 원점을 통해 뒤집고 길이를 두 배로 늘리는 것에 해당합니다.

\subsubsection{선형 결합}\label{linear-combinations}

덧셈과 스칼라 곱셈의 상호 작용을 통해 \emph{선형 결합}을 형성할 수 있습니다. 벡터 \(\mathbf{v}_1, \mathbf{v}_2, \dots, \mathbf{v}_k\)의 선형 결합은 다음과 같은 형태의 모든 벡터입니다.

\[c_1 \mathbf{v}_1 + c_2 \mathbf{v}_2 + \cdots + c_k \mathbf{v}_k, \quad c_i \in \mathbb{R}.\]

선형 결합은 기존 벡터에서 새 벡터를 생성하는 메커니즘입니다. 벡터 집합의 스팬(모든 선형 결합의 모음)은 나중에 부분 공간의 아이디어로 이어질 것입니다.

예제 1.2.3.\\
\(\mathbf{v}_1 = (1,0)\)이고 \(\mathbf{v}_2 = (0,1)\)이라고 합시다. 그러면 \(\mathbb{R}^2\)의 모든 벡터 \((a,b)\)는 다음과 같이 표현될 수 있습니다.

\[a\mathbf{v}_1 + b\mathbf{v}_2.\]

따라서 \((1,0)\)과 \((0,1)\)은 평면의 기본 구성 요소를 형성합니다.

\subsubsection{표기법}\label{notation-2}

\begin{itemize}
\item
  덧셈: \(\mathbf{u} + \mathbf{v}\)는 성분별 덧셈을 의미합니다.
\item
  스칼라 곱셈: \(c\mathbf{v}\)는 \(\mathbf{v}\)의 각 항목에 \(c\)를 곱합니다.
\item
  선형 결합: \(c_1 \mathbf{v}_1 + \cdots + c_k \mathbf{v}_k\) 형태의 합.
\end{itemize}

\subsubsection{이것이 중요한 이유}\label{why-this-matters}

벡터 덧셈과 스칼라 곱셈은 선형대수학의 정의적인 연산입니다. 그들은 벡터 공간에 구조를 부여하고, 평행 이동 및 스케일링과 같은 기하학적 현상을 설명할 수 있게 하며, 방정식 계를 푸는 기초를 제공합니다. 기저, 차원, 변환 등 다음에 오는 모든 것은 이러한 간단하지만 심오한 규칙 위에 구축됩니다.

\subsubsection{연습문제 1.2}\label{exercises-12}

\begin{enumerate}
\def\labelenumi{\arabic{enumi}.}
\item
  \(\mathbf{u} = (1,2,3)\)이고 \(\mathbf{v} = (4, -1, 0)\)일 때 \(\mathbf{u} + \mathbf{v}\)를 계산하십시오.
\item
  \(\mathbf{v} = (-2,5)\)일 때 \$3\mathbf{v}\$를 찾으십시오. 두 벡터를 스케치하여 스케일링을 설명하십시오.
\item
  \((5,7)\)이 \((1,0)\)과 \((0,1)\)의 선형 결합으로 쓸 수 있음을 보이십시오.
\item
  \((4,4)\)를 \((1,1)\)과 \((1,-1)\)의 선형 결합으로 쓰십시오.
\item
  \(\mathbf{u}, \mathbf{v} \in \mathbb{R}^n\)이면, 스칼라 \(c,d \in \mathbb{R}\)에 대해 \((c+d)(\mathbf{u}+\mathbf{v}) = c\mathbf{u} + c\mathbf{v} + d\mathbf{u} + d\mathbf{v}\)임을 증명하십시오.
\end{enumerate}

\subsection{1.3 내적, 노름, 각도}\label{13-dot-product-norms-and-angles}

내적은 벡터 공간에서 대수와 기하학을 연결하는 기본 연산입니다. 길이를 측정하고, 각도를 계산하고, 직교성을 결정할 수 있게 해줍니다. 이 단일 정의에서 \emph{노름}과 \emph{각도}의 개념이 흘러나와 추상 벡터 공간에 기하학을 부여합니다.

\subsubsection{내적}\label{the-dot-product}

\(\mathbb{R}^n\)의 두 벡터에 대해 내적(또는 내적)은 다음과 같이 정의됩니다.

\[\mathbf{u} \cdot \mathbf{v} = u_1 v_1 + u_2 v_2 + \cdots + u_n v_n.\]

행렬 표기법으로는 다음과 같습니다.

\[\mathbf{u} \cdot \mathbf{v} = \mathbf{u}^T \mathbf{v}.\]

예제 1.3.1.\\
\(\mathbf{u} = (2, -1, 3)\)이고 \(\mathbf{v} = (4, 0, -2)\)라고 합시다. 그러면

\[\mathbf{u} \cdot \mathbf{v} = 2\cdot 4 + (-1)\cdot 0 + 3\cdot (-2) = 8 - 6 = 2.\]

내적은 다른 벡터가 아닌 단일 스칼라를 출력합니다.

\subsubsection{노름 (벡터의 길이)}\label{norms-length-of-a-vector}

벡터의 \emph{유클리드 노름}은 자신과의 내적의 제곱근입니다.

\[\|\mathbf{v}\| = \sqrt{\mathbf{v} \cdot \mathbf{v}} = \sqrt{v_1^2 + v_2^2 + \cdots + v_n^2}.\]

이것은 피타고라스 정리를 임의의 차원으로 일반화합니다.

예제 1.3.2.\\
\(\mathbf{v} = (3, 4)\)에 대해,

\[\|\mathbf{v}\| = \sqrt{3^2 + 4^2} = \sqrt{25} = 5.\]

이것은 평면에서 화살표로서의 벡터의 길이와 정확히 같습니다.

\subsubsection{벡터 간의 각도}\label{angles-between-vectors}

내적은 또한 두 벡터 사이의 각도를 인코딩합니다. 0이 아닌 벡터 \(\mathbf{u}, \mathbf{v}\)에 대해,

\[\mathbf{u} \cdot \mathbf{v} = \|\mathbf{u}\| \, \|\mathbf{v}\| \cos \theta,\]

여기서 \(\theta\)는 두 벡터 사이의 각도입니다. 따라서,

\[\cos \theta = \frac{\mathbf{u} \cdot \mathbf{v}}{\|\mathbf{u}\|\|\mathbf{v}\|}.\]

예제 1.3.3.\\
\(\mathbf{u} = (1,0)\)이고 \(\mathbf{v} = (0,1)\)이라고 합시다. 그러면

\[\mathbf{u} \cdot \mathbf{v} = 0, \quad \|\mathbf{u}\| = 1, \quad \|\mathbf{v}\| = 1.\]

따라서

\[\cos \theta = \frac{0}{1\cdot 1} = 0 \quad \Rightarrow \quad \theta = \frac{\pi}{2}.\]

벡터는 수직입니다.

\subsubsection{직교성}\label{orthogonality}

두 벡터는 내적이 0이면 직교한다고 합니다.

\[\mathbf{u} \cdot \mathbf{v} = 0.\]

직교성은 기하학에서 수직성의 아이디어를 고차원으로 일반화합니다.

\subsubsection{표기법}\label{notation-3}

\begin{itemize}
\item
  내적: \(\mathbf{u} \cdot \mathbf{v}\).
\item
  노름 (길이): \(|\mathbf{v}|\).
\item
  직교성: \(\mathbf{u} \cdot \mathbf{v} = 0\)이면 \(\mathbf{u} \perp \mathbf{v}\).
\end{itemize}

\subsubsection{이것이 중요한 이유}\label{why-this-matters-2}

내적은 벡터 공간을 기하학적 객체로 바꿉니다: 벡터는 길이, 각도 및 수직성의 개념을 얻습니다. 이 기초는 나중에 직교 투영, 그람-슈미트 직교화, 고유 벡터 및 최소 제곱 문제 연구를 지원할 것입니다.

\subsubsection{연습문제 1.3}\label{exercises-13}

\begin{enumerate}
\def\labelenumi{\arabic{enumi}.}
\item
  \(\mathbf{u} = (1,2,3)\), \(\mathbf{v} = (4,5,6)\)에 대해 \(\mathbf{u} \cdot \mathbf{v}\)를 계산하십시오.
\item
  \(\mathbf{v} = (2, -2, 1)\)의 노름을 찾으십시오.
\item
  \(\mathbf{u} = (1,1,0)\)과 \(\mathbf{v} = (1,-1,2)\)가 직교하는지 확인하십시오.
\item
  \(\mathbf{u} = (3,4)\), \(\mathbf{v} = (4,3)\)이라고 합시다. 두 벡터 사이의 각도를 계산하십시오.
\item
  \(|\mathbf{u} + \mathbf{v}|^2 = |\mathbf{u}|^2 + |\mathbf{v}|^2 + 2\mathbf{u}\cdot \mathbf{v}\)임을 증명하십시오. 이 항등식은 코사인 법칙의 대수적 버전입니다.
\end{enumerate}

\subsection{1.4 직교성}\label{14-orthogonality}

직교성은 벡터 공간에서 수직성의 개념을 포착합니다. 이것은 선형대수학에서 가장 중요한 기하학적 아이디어 중 하나로, 벡터를 분해하고, 투영을 정의하고, 우아한 속성을 가진 특수 기저를 구성할 수 있게 합니다.

\subsubsection{정의}\label{definition}

\(\mathbb{R}^n\)의 두 벡터 \(\mathbf{u}, \mathbf{v}\)는 내적이 0이면 직교한다고 합니다.

\[\mathbf{u} \cdot \mathbf{v} = 0.\]

이 조건은 두 벡터 사이의 각도가 \(\pi/2\) 라디안(90도)임을 보장합니다.

예제 1.4.1.\\
\(\mathbb{R}^2\)에서 벡터 \((1,2)\)와 \((2,-1)\)은 직교합니다. 왜냐하면

\[(1,2) \cdot (2,-1) = 1\cdot 2 + 2\cdot (-1) = 0.\]

\subsubsection{직교 집합}\label{orthogonal-sets}

벡터 모음은 집합의 모든 서로 다른 벡터 쌍이 직교하면 직교 집합이라고 합니다. 또한 각 벡터의 노름이 1이면, 그 집합은 정규직교 집합이라고 합니다.

예제 1.4.2.\\
\(\mathbb{R}^3\)에서 표준 기저 벡터

\[\mathbf{e}_1 = (1,0,0), \quad \mathbf{e}_2 = (0,1,0), \quad \mathbf{e}_3 = (0,0,1)\]

는 정규직교 집합을 형성합니다: 각 벡터의 길이는 1이고, 인덱스가 다를 때 내적은 0이 됩니다.

\subsubsection{투영}\label{projections}

직교성은 벡터를 두 성분으로 분해하는 것을 가능하게 합니다: 하나는 다른 벡터에 평행하고, 다른 하나는 직교합니다. 0이 아닌 벡터 \(\mathbf{u}\)와 임의의 벡터 \(\mathbf{v}\)가 주어졌을 때, \(\mathbf{v}\)의 \(\mathbf{u}\) 위로의 투영은

\[\text{proj}_{\mathbf{u}}(\mathbf{v}) = \frac{\mathbf{u} \cdot \mathbf{v}}{\mathbf{u} \cdot \mathbf{u}} \mathbf{u}.\]

차이

\[\mathbf{v} - \text{proj}_{\mathbf{u}}(\mathbf{v})\]

는 \(\mathbf{u}\)에 직교합니다. 따라서 모든 벡터는 다른 벡터에 대해 평행한 부분과 수직인 부분으로 고유하게 분해될 수 있습니다.

예제 1.4.3.\\
\(\mathbf{u} = (1,0)\), \(\mathbf{v} = (2,3)\)이라고 합시다. 그러면

\[\text{proj}_{\mathbf{u}}(\mathbf{v}) = \frac{(1,0)\cdot(2,3)}{(1,0)\cdot(1,0)} (1,0)
= \frac{2}{1}(1,0) = (2,0).\]

따라서

\[\mathbf{v} = (2,3) = (2,0) + (0,3),\]

여기서 \((2,0)\)은 \((1,0)\)에 평행하고 \((0,3)\)은 그것에 직교합니다.

\subsubsection{직교 분해}\label{orthogonal-decomposition}

일반적으로, 만약 \(\mathbf{u} \neq \mathbf{0}\)이고 \(\mathbf{v} \in \mathbb{R}^n\)이면,

\[\mathbf{v} = \text{proj}\_{\mathbf{u}}(\mathbf{v}) + \big(\mathbf{v} - \text{proj}\_{\mathbf{u}}(\mathbf{v})\big),\]

여기서 첫 번째 항은 \(\mathbf{u}\)에 평행하고 두 번째 항은 직교합니다. 이 분해는 최소 제곱 근사 및 그람-슈미트 과정과 같은 방법의 기초가 됩니다.

\subsubsection{표기법}\label{notation-4}

\begin{itemize}
\item
  \(\mathbf{u} \perp \mathbf{v}\): 벡터 \(\mathbf{u}\)와 \(\mathbf{v}\)는 직교합니다.
\item
  직교 집합: 벡터들이 쌍으로 직교합니다.
\item
  정규직교 집합: 쌍으로 직교하고, 각 벡터의 노름이 1입니다.
\end{itemize}

\subsubsection{이것이 중요한 이유}\label{why-this-matters-3}

직교성은 벡터 공간에 구조를 부여합니다. 독립적인 방향을 명확하게 분리하고, 계산을 단순화하고, 근사에서 오류를 최소화하는 방법을 제공합니다. 수치 선형대수학 및 데이터 과학의 많은 강력한 알고리즘(QR 분해, 최소 제곱 회귀, PCA)은 직교성에 의존합니다.

\subsubsection{연습문제 1.4}\label{exercises-14}

\begin{enumerate}
\def\labelenumi{\arabic{enumi}.}
\item
  벡터 \((1,2,2)\)와 \((2,0,-1)\)이 직교하는지 확인하십시오.
\item
  \((3,4)\)의 \((1,1)\) 위로의 투영을 찾으십시오.
\item
  \(\mathbb{R}^n\)에서 임의의 두 개의 서로 다른 표준 기저 벡터가 직교함을 보이십시오.
\item
  \((5,2)\)를 \((2,1)\)에 평행하고 직교하는 성분으로 분해하십시오.
\item
  \(\mathbf{u}, \mathbf{v}\)가 직교하는 0이 아닌 벡터라고 합시다.\\
  (a) \((\mathbf{u}+\mathbf{v})\cdot(\mathbf{u}-\mathbf{v})=\lVert \mathbf{u}\rVert^2-\lVert \mathbf{v}\rVert^2\)임을 보이십시오.\\
  (b) \(\mathbf{u}\)와 \(\mathbf{v}\)에 대한 어떤 조건에서 \((\mathbf{u}+\mathbf{v})\cdot(\mathbf{u}-\mathbf{v})=0\)이 됩니까?
\end{enumerate}

\section{2장. 행렬}\label{chapter-2-matrices}

\subsection{2.1 정의와 표기법}\label{21-definition-and-notation}

행렬은 선형대수학의 중심 객체로, 선형 변환, 연립 방정식, 구조화된 데이터를 간결하게 표현하고 조작하는 방법을 제공합니다. 행렬은 행과 열로 배열된 숫자의 직사각형 배열입니다.

\subsubsection{형식적 정의}\label{formal-definition}

\(m \times n\) 행렬은 \(m\)개의 행과 \(n\)개의 열을 가진 배열로, 다음과 같이 씁니다.

\[A =
\begin{bmatrix}
a_{11} & a_{12} & \cdots & a_{1n} \\
a_{21} & a_{22} & \cdots & a_{2n} \\
\vdots & \vdots & \ddots & \vdots \\
a_{m1} & a_{m2} & \cdots & a_{mn}
\end{bmatrix}.\]

각 항목 \(a_{ij}\)는 \emph{i}번째 행과 \emph{j}번째 열에 위치한 스칼라입니다. 행렬의 크기(또는 차원)는 \(m \times n\)으로 표기됩니다.

\begin{itemize}
\item
  \(m = n\)이면, 행렬은 정사각 행렬입니다.
\item
  \(m = 1\)이면, 행렬은 행 벡터입니다.
\item
  \(n = 1\)이면, 행렬은 열 벡터입니다.
\end{itemize}

따라서 벡터는 단순히 행렬의 특수한 경우입니다.

\subsubsection{예제}\label{examples}

예제 2.1.1. \$2 \textbackslash times 3\$ 행렬:

\[A = \begin{bmatrix}
1 & -2 & 4 \\
0 & 3 & 5
\end{bmatrix}.\]

여기서 \(a_{12} = -2\), \(a_{23} = 5\)이고, 행렬은 2개의 행, 3개의 열을 가집니다.

예제 2.1.2. \$3 \textbackslash times 3\$ 정사각 행렬:

\[B = \begin{bmatrix}
2 & 0 & 1 \\
-1 & 3 & 4 \\
0 & 5 & -2
\end{bmatrix}.\]

이것은 나중에 \(\mathbb{R}^3\)에 대한 선형 변환의 표현으로 사용될 것입니다.

\subsubsection{인덱싱과 표기법}\label{indexing-and-notation}

\begin{itemize}
\item
  행렬은 대문자 굵은 글씨로 표기합니다: \(A, B, C\).
\item
  항목은 \(a_{ij}\)로 쓰며, 행 인덱스가 먼저, 열 인덱스가 두 번째입니다.
\item
  모든 실수 \(m \times n\) 행렬의 집합은 \(\mathbb{R}^{m \times n}\)으로 표기합니다.
\end{itemize}

따라서 행렬은 각 행-열 위치에 스칼라를 할당하는 함수 \(A: {1,\dots,m} \times {1,\dots,n} \to \mathbb{R}\)입니다.

\subsubsection{이것이 중요한 이유}\label{why-this-matters-4}

행렬은 벡터를 일반화하고 선형 연산을 체계적으로 설명하는 언어를 제공합니다. 연립 방정식, 회전, 투영 및 데이터 변환을 인코딩합니다. 행렬을 사용하면 대수와 기하학이 함께 모입니다: 단일의 간결한 객체가 수치 데이터와 함수 규칙을 모두 나타낼 수 있습니다.

\subsubsection{연습문제 2.1}\label{exercises-21}

\begin{enumerate}
\def\labelenumi{\arabic{enumi}.}
\item
  선택한 \$3 \textbackslash times 2\$ 행렬을 쓰고 그 항목 \(a_{ij}\)를 식별하십시오.
\item
  모든 벡터는 행렬입니까? 모든 행렬은 벡터입니까? 설명하십시오.
\item
  다음 중 정사각 행렬은 어느 것입니까: \(A \in \mathbb{R}^{4\times4}\), \(B \in \mathbb{R}^{3\times5}\), \(C \in \mathbb{R}^{1\times1}\)?
\item
  다음 행렬을 고려하십시오.
\end{enumerate}

\[D = \begin{bmatrix} 1 & 0 \\\\ 0 & 1 \end{bmatrix}\]

이것은 어떤 종류의 행렬입니까?

\begin{enumerate}
\def\labelenumi{\arabic{enumi}.}
\item
  다음 행렬을 고려하십시오.
\end{enumerate}

\[E = \begin{bmatrix} a & b \\ c & d \end{bmatrix}\]

\(e_{11}, e_{12}, e_{21}, e_{22}\)를 명시적으로 표현하십시오.

\subsection{2.2 행렬 덧셈과 곱셈}\label{22-matrix-addition-and-multiplication}

행렬이 정의되면, 다음 단계는 그들이 어떻게 결합되는지 이해하는 것입니다. 벡터가 덧셈과 스칼라 곱셈을 통해 의미를 얻는 것처럼, 행렬은 두 가지 연산, 즉 덧셈과 곱셈을 통해 강력해집니다.

\subsubsection{행렬 덧셈}\label{matrix-addition}

같은 크기의 두 행렬은 해당 항목을 더하여 더해집니다. 만약

\[A = [a_{ij}] \in \mathbb{R}^{m \times n}, \quad
B = [b_{ij}] \in \mathbb{R}^{m \times n},\]

이면

\[A + B = [a_{ij} + b_{ij}] \in \mathbb{R}^{m \times n}.\]

예제 2.2.1.\\
다음과 같다고 합시다.

\[A = \begin{bmatrix}
1 & 2 \\
3 & 4
\end{bmatrix}, \quad
B = \begin{bmatrix}
-1 & 0 \\
5 & 2
\end{bmatrix}.\]

그러면

\[A + B = \begin{bmatrix}
1 + (-1) & 2 + 0 \\
3 + 5 & 4 + 2
\end{bmatrix} =
\begin{bmatrix}
0 & 2 \\
8 & 6
\end{bmatrix}.\]

행렬 덧셈은 교환 법칙(\(A+B = B+A\))과 결합 법칙(\((A+B)+C = A+(B+C)\))을 만족합니다. 모든 항목이 0인 영행렬은 덧셈에 대한 항등원 역할을 합니다.

\subsubsection{스칼라 곱셈}\label{scalar-multiplication-2}

스칼라 \(c \in \mathbb{R}\)와 행렬 \(A = [[a_{ij}]\)에 대해, 우리는 다음과 같이 정의합니다.

\[cA = [c \cdot a_{ij}].\]

이것은 행렬의 모든 항목을 균일하게 늘리거나 줄입니다.

예제 2.2.2.\\
만약

\[A = \begin{bmatrix}
2 & -1 \\
0 & 3
\end{bmatrix}, \quad c = -2,\]

이면

\[cA = \begin{bmatrix}
-4 & 2 \\
0 & -6
\end{bmatrix}.\]

\subsubsection{행렬 곱셈}\label{matrix-multiplication}

행렬의 정의적인 연산은 곱셈입니다. 만약

\[A \in \mathbb{R}^{m \times n}, \quad B \in \mathbb{R}^{n \times p},\]

이면, 그 곱은 \(m \times p\) 행렬입니다.

\[AB = C = [c_{ij}], \quad c_{ij} = \sum_{k=1}^n a_{ik} b_{kj}.\]

따라서 \(AB\)의 \(i\)번째 행과 \(j\)번째 열의 항목은 \(A\)의 \(i\)번째 행과 \(B\)의 \(j\)번째 열의 내적입니다.

예제 2.2.3.\\
다음과 같다고 합시다.

\[A = \begin{bmatrix}
1 & 2 \\
0 & 3
\end{bmatrix}, \quad
B = \begin{bmatrix}
4 & -1 \\
2 & 5
\end{bmatrix}.\]

그러면

\[AB = \begin{bmatrix}
1\cdot4 + 2\cdot2 & 1\cdot(-1) + 2\cdot5 \\
0\cdot4 + 3\cdot2 & 0\cdot(-1) + 3\cdot5
\end{bmatrix} =
\begin{bmatrix}
8 & 9 \\
6 & 15
\end{bmatrix}.\]

행렬 곱셈은 일반적으로 교환 법칙을 만족하지 않습니다: \(AB \neq BA\). 때로는 차원이 맞지 않으면 \(BA\)가 정의되지 않을 수도 있습니다.

\subsubsection{기하학적 의미}\label{geometric-meaning}

행렬 곱셈은 선형 변환의 합성에 해당합니다. 만약 \(A\)가 \(\mathbb{R}^n\)의 벡터를 변환하고 \(B\)가 \(\mathbb{R}^p\)의 벡터를 변환한다면, \(AB\)는 \(B\)를 먼저 적용한 다음 \(A\)를 적용하는 것을 나타냅니다. 이것은 행렬을 변환의 대수적 언어로 만듭니다.

\subsubsection{표기법}\label{notation-5}

\begin{itemize}
\item
  행렬 합: \(A+B\).
\item
  스칼라 곱: \(cA\).
\item
  곱: \(AB\), \(A\)의 열 수와 \(B\)의 행 수가 같을 때만 정의됩니다.
\end{itemize}

\subsubsection{이것이 중요한 이유}\label{why-this-matters-5}

행렬 곱셈은 선형대수학의 핵심 메커니즘입니다: 변환이 어떻게 결합되는지, 연립 방정식이 어떻게 해결되는지, 현대 알고리즘에서 데이터가 어떻게 흐르는지를 인코딩합니다. 덧셈과 스칼라 곱셈은 행렬을 벡터 공간으로 만들고, 곱셈은 기하학, 계산, 네트워크를 모델링할 수 있을 만큼 풍부한 대수적 구조를 제공합니다.

\subsubsection{연습문제 2.2}\label{exercises-22}

\begin{enumerate}
\def\labelenumi{\arabic{enumi}.}
\item
  다음에 대해 \(A+B\)를 계산하십시오.
\end{enumerate}

\[A = \begin{bmatrix} 2 & 3 \\
-1 & 0 \end{bmatrix}, \quad
B = \begin{bmatrix} 4 & -2 \\
5 & 7 \end{bmatrix}.\]

\begin{enumerate}
\def\labelenumi{\arabic{enumi}.}
\item
  \(A = \begin{bmatrix} 1 & -4 \\ 2 & 6 \end{bmatrix}\)일 때 \$3A\$를 찾으십시오.
\end{enumerate}

\begin{enumerate}
\def\labelenumi{\arabic{enumi}.}
\item
  다음을 곱하십시오.
\end{enumerate}

\[A = \begin{bmatrix} 1 & 0 & 2 \\
-1 & 3 & 1 \end{bmatrix}, \quad
B = \begin{bmatrix} 2 & 1 \\
0 & -1 \\
3 & 4 \end{bmatrix}.\]

\begin{enumerate}
\def\labelenumi{\arabic{enumi}.}
\item
  명시적인 예제로 \(AB \neq BA\)임을 확인하십시오.
\item
  행렬 곱셈이 분배 법칙을 만족함을 증명하십시오: \(A(B+C) = AB + AC\).
\end{enumerate}

\subsection{2.3 전치와 역행렬}\label{23-transpose-and-inverse}

행렬에 대한 두 가지 특별한 연산, 즉 전치와 역행렬은 깊은 대수적 및 기하학적 속성을 낳습니다. 전치는 주 대각선을 기준으로 행렬을 뒤집어 재배열하고, 역행렬은 존재할 경우 행렬 곱셈에 대한 되돌리기 연산으로 작용합니다.

\subsubsection{전치}\label{the-transpose}

\(m \times n\) 행렬 \(A = [a_{ij}]\)의 전치는 행과 열을 바꾼 \(n \times m\) 행렬 \(A^T = [a_{ji}]\)입니다.

형식적으로,

\[(A^T)\_{ij} = a\_{ji}.\]

예제 2.3.1.\\
만약

\[A = \begin{bmatrix}
1 & 4 & -2 \\
0 & 3 & 5
\end{bmatrix},\]

이면

\[A^T = \begin{bmatrix}
1 & 0 \\
4 & 3 \\
-2 & 5
\end{bmatrix}.\]

전치의 속성.

\begin{enumerate}
\def\labelenumi{\arabic{enumi}.}
\item
  \((A^T)^T = A\).
\item
  \((A+B)^T = A^T + B^T\).
\item
  \((cA)^T = cA^T\), 스칼라 \(c\)에 대해.
\item
  \((AB)^T = B^T A^T\).
\end{enumerate}

마지막 규칙은 중요합니다: 순서가 바뀝니다.

\subsubsection{역행렬}\label{the-inverse}

정사각 행렬 \(A \in \mathbb{R}^{n \times n}\)은 다른 행렬 \(A^{-1}\)이 존재하여

\[AA^{-1} = A^{-1}A = I_n,\]

을 만족할 때 가역(또는 비특이)이라고 합니다. 여기서 \(I_n\)은 \(n \times n\) 단위 행렬입니다. 이 경우 \(A^{-1}\)은 \(A\)의 역행렬이라고 합니다.

모든 행렬이 가역인 것은 아닙니다. 필요 조건은 \(\det(A) \neq 0\)이라는 사실이며, 이는 6장에서 다룰 것입니다.

예제 2.3.2.\\
다음과 같다고 합시다.

\[A = \begin{bmatrix}
1 & 2 \\
3 & 4
\end{bmatrix}.\]

그 행렬식은 \(\det(A) = (1)(4) - (2)(3) = -2 \neq 0\)입니다. 역행렬은

\[A^{-1} = \frac{1}{\det(A)} \begin{bmatrix}
4 & -2 \\
-3 & 1
\end{bmatrix} =
\begin{bmatrix}
-2 & 1 \\
1.5 & -0.5
\end{bmatrix}.\]

검증:

\[AA^{-1} = \begin{bmatrix}
1 & 2 \\
3 & 4
\end{bmatrix}
\begin{bmatrix}
-2 & 1 \\
1.5 & -0.5
\end{bmatrix} =
\begin{bmatrix}
1 & 0 \\
0 & 1
\end{bmatrix}.\]

\subsubsection{기하학적 의미}\label{geometric-meaning-2}

\begin{itemize}
\item
  전치는 대각선을 기준으로 선형 변환을 반사하는 것에 해당합니다. 벡터의 경우 행과 열 형식을 전환합니다.
\item
  역행렬은 존재할 경우 선형 변환을 되돌리는 것에 해당합니다. 예를 들어, \(A\)가 벡터를 스케일링하고 회전하면 \(A^{-1}\)은 다시 스케일링하고 회전하여 원래대로 되돌립니다.
\end{itemize}

\subsubsection{표기법}\label{notation-6}

\begin{itemize}
\item
  전치: \(A^T\).
\item
  역행렬: \(A^{-1}\), 가역 정사각 행렬에 대해서만 정의됩니다.
\item
  단위 행렬: \(I_n\), 곱셈에 대한 항등원 역할을 합니다.
\end{itemize}

\subsubsection{이것이 중요한 이유}\label{why-this-matters-6}

전치는 대칭 행렬과 직교 행렬을 정의할 수 있게 해주며, 이는 기하학과 수치 방법에 중심적입니다. 역행렬은 선형 시스템의 해법의 기초가 되며, 변환을 되돌리는 아이디어를 인코딩합니다. 함께, 이러한 연산은 행렬식, 고유값, 직교화의 무대를 설정합니다.

\subsubsection{연습문제 2.3}\label{exercises-23}

\begin{enumerate}
\def\labelenumi{\arabic{enumi}.}
\item
  다음의 전치를 계산하십시오.
\end{enumerate}

\[A = \begin{bmatrix} 2 & -1 & 3 \\ 0 & 4 & 5 \end{bmatrix}.\]

\begin{enumerate}
\def\labelenumi{\arabic{enumi}.}
\item
  다음에 대해 \((AB)^T = B^T A^T\)임을 확인하십시오.
\end{enumerate}

\[A = \begin{bmatrix}
1 & 2 \\
0 & 1 \end{bmatrix}, \quad
B = \begin{bmatrix}
3 & 4 \\
5 & 6 \end{bmatrix}.\]

\begin{enumerate}
\def\labelenumi{\arabic{enumi}.}
\item
  다음이 가역인지 확인하십시오.
\end{enumerate}

\[C = \begin{bmatrix}
2 & 1 \\
4 & 2 \end{bmatrix}\]

만약 그렇다면, \(C^{-1}\)를 찾으십시오.

\begin{enumerate}
\def\labelenumi{\arabic{enumi}.}
\item
  다음의 역행렬을 찾으십시오.
\end{enumerate}

\[D = \begin{bmatrix}
0 & 1 \\
-1 & 0 \end{bmatrix},\]

그리고 평면에서 벡터에 대한 기하학적 작용을 설명하십시오.

\begin{enumerate}
\def\labelenumi{\arabic{enumi}.}
\item
  \(A\)가 가역이면 \(A^T\)도 가역이고, \((A^T)^{-1} = (A^{-1})^T\)임을 증명하십시오.
\end{enumerate}

\subsection{2.4 특수 행렬}\label{24-special-matrices}

이론과 응용에서 매우 자주 발생하는 특정 행렬에는 특별한 이름이 부여됩니다. 그 속성을 인식하면 계산을 단순화하고 선형 변환의 구조를 더 명확하게 이해할 수 있습니다.

\subsubsection{단위 행렬}\label{the-identity-matrix}

단위 행렬 \(I_n\)은 대각선에 1이 있고 다른 곳에는 0이 있는 \(n \times n\) 행렬입니다:

\[I_n = \begin{bmatrix}
1 & 0 & \cdots & 0 \\
0 & 1 & \cdots & 0 \\
\vdots & \vdots & \ddots & \vdots \\
0 & 0 & \cdots & 1
\end{bmatrix}.\]

곱셈에 대한 항등원 역할을 합니다:

\[AI_n = I_nA = A, \quad \text{for all } A \in \mathbb{R}^{n \times n}.\]

기하학적으로, \(I_n\)은 모든 벡터를 변경하지 않는 변환을 나타냅니다.

\subsubsection{대각 행렬}\label{diagonal-matrices}

대각 행렬은 비대각 항목이 모두 0입니다:

\[D = \begin{bmatrix}
d_{11} & 0 & \cdots & 0 \\
0 & d_{22} & \cdots & 0 \\
\vdots & \vdots & \ddots & \vdots \\
0 & 0 & \cdots & d_{nn}
\end{bmatrix}.\]

대각 행렬에 의한 곱셈은 각 좌표를 독립적으로 스케일링합니다:

\[D\mathbf{x} = (d_{11}x_1, d_{22}x_2, \dots, d_{nn}x_n).\]

예제 2.4.1.\\
다음과 같다고 합시다.

\[D = \begin{bmatrix} 2 & 0 & 0 \\
0 & 3 & 0 \\
0 & 0 & -1 \end{bmatrix}, \quad
\mathbf{x} = \begin{bmatrix}
1 \\
4 \\
-2 \end{bmatrix}.\]

그러면

\[D\mathbf{x} = \begin{bmatrix}
2 \\
12 \\
2 \end{bmatrix}.\]

\subsubsection{순열 행렬}\label{permutation-matrices}

순열 행렬은 단위 행렬의 행을 순열하여 얻습니다. 벡터에 순열 행렬을 곱하면 좌표가 재정렬됩니다.

예제 2.4.2.\\
다음과 같다고 합시다.

\[P = \begin{bmatrix}
0 & 1 & 0 \\
1 & 0 & 0 \\
0 & 0 & 1
\end{bmatrix}.\]

그러면

\[P\begin{bmatrix}
a \\
b \\
c \end{bmatrix} =
\begin{bmatrix} b \\
a \\
c \end{bmatrix}.\]

따라서 \(P\)는 첫 두 좌표를 바꿉니다.

순열 행렬은 항상 가역입니다. 그 역행렬은 단순히 그 전치입니다.

\subsubsection{대칭 행렬과 반대칭 행렬}\label{symmetric-and-skew-symmetric-matrices}

행렬은 다음과 같을 때 대칭입니다.

\[A^T = A,\]

그리고 다음과 같을 때 반대칭입니다.\\
대칭 행렬은 이차 형식과 최적화에 나타나고, 반대칭 행렬은 기하학에서 회전과 외적을 설명합니다.

\subsubsection{직교 행렬}\label{orthogonal-matrices}

정사각 행렬 \(Q\)는 다음과 같을 때 직교입니다.

\[Q^T Q = QQ^T = I.\]

동등하게, \(Q\)의 행(과 열)은 정규직교 집합을 형성합니다. 직교 행렬은 길이와 각도를 보존합니다. 회전과 반사를 나타냅니다.

예제 2.4.3.\\
평면에서의 회전 행렬:

\[R(\theta) = \begin{bmatrix}
\cos\theta & -\sin\theta \\
\sin\theta & \cos\theta
\end{bmatrix}\]

는 직교입니다. 왜냐하면

\[R(\theta)^T R(\theta) = I_2.\]

\subsubsection{이것이 중요한 이유}\label{why-this-matters-7}

특수 행렬은 선형대수학의 구성 요소 역할을 합니다. 단위 행렬은 중립 요소를 정의하고, 대각 행렬은 계산을 단순화하고, 순열 행렬은 데이터를 재정렬하고, 대칭 및 직교 행렬은 기본 기하학적 구조를 설명합니다. 현대 응용 수학의 대부분은 복잡한 문제를 이러한 간단한 형태를 포함하는 연산으로 축소합니다.

\subsubsection{연습문제 2.4}\label{exercises-24}

\begin{enumerate}
\def\labelenumi{\arabic{enumi}.}
\item
  두 대각 행렬의 곱이 대각 행렬임을 보이고, 예제를 계산하십시오.
\item
  \((a,b,c)\)를 \((b,c,a)\)로 순환시키는 순열 행렬을 찾으십시오.
\item
  모든 순열 행렬이 가역이고 그 역행렬이 그 전치임을 증명하십시오.
\item
  다음이 직교임을 확인하십시오.
\end{enumerate}

\[Q = \begin{bmatrix}
0 & 1 \\
-1 & 0 \end{bmatrix}\]

어떤 기하학적 변환을 나타냅니까?\\
5. 다음이 대칭, 반대칭, 또는 둘 다 아닌지 확인하십시오.

\[A = \begin{bmatrix}
2 & 3 \\
3 & 2 \end{bmatrix}, \quad
B = \begin{bmatrix}
0 & 5 \\
-5 & 0 \end{bmatrix}\]

\section{3장. 연립 선형 방정식}\label{chapter-3-systems-of-linear-equations}

\subsection{3.1 선형 시스템과 해}\label{31-linear-systems-and-solutions}

선형대수학의 중심 동기 중 하나는 연립 선형 방정식을 푸는 것입니다. 이러한 시스템은 여러 제약 조건이 상호 작용할 때마다 과학, 공학, 데이터 분석에서 자연스럽게 발생합니다. 행렬은 이를 표현하고 해결하기 위한 간결한 언어를 제공합니다.

\subsubsection{선형 시스템}\label{linear-systems}

선형 시스템은 각 미지수가 1차로만 나타나고 변수 간의 곱이 없는 방정식으로 구성됩니다. \(n\)개의 미지수를 가진 \(m\)개의 방정식의 일반적인 시스템은 다음과 같이 쓸 수 있습니다:

\begin{aligned}
a_{11}x_1 + a_{12}x_2 + \cdots + a_{1n}x_n &= b_1, \\
a_{21}x_1 + a_{22}x_2 + \cdots + a_{2n}x_n &= b_2, \\
&\vdots \\
a_{m1}x_1 + a_{m2}x_2 + \cdots + a_{mn}x_n &= b_m.
\end{aligned}

여기서 계수 \(a_{ij}\)와 상수 \(b_i\)는 스칼라이고, 미지수는 \(x_1, x_2, \dots, x_n\)입니다.

\subsubsection{행렬 형태}\label{matrix-form}

시스템은 다음과 같이 간결하게 표현될 수 있습니다:

\[A\mathbf{x} = \mathbf{b},\]

여기서

\begin{itemize}
\item
  \(A \in \mathbb{R}^{m \times n}\)는 계수 행렬 \([a_{ij}]\)입니다.
\item
  \(\mathbf{x} \in \mathbb{R}^n\)는 미지수의 열 벡터입니다.
\item
  \(\mathbf{b} \in \mathbb{R}^m\)는 상수의 열 벡터입니다.
\end{itemize}

이 공식은 방정식을 푸는 문제를 행렬의 작용을 분석하는 문제로 바꿉니다.

예제 3.1.1.\\
시스템

\begin{cases}
x + 2y = 5, \\
3x - y = 4
\end{cases}

은 다음과 같이 쓸 수 있습니다.

\begin{bmatrix} 1 & 2 \\ 3 & -1 \end{bmatrix}
\begin{bmatrix} x \\ y \end{bmatrix}
=
\begin{bmatrix} 5 \\ 4 \end{bmatrix}.

\subsubsection{해의 종류}\label{types-of-solutions}

선형 시스템은 다음을 가질 수 있습니다:

\begin{enumerate}
\def\labelenumi{\arabic{enumi}.}
\item
  해가 없음 (모순): 방정식이 충돌합니다. 예:
\end{enumerate}

\begin{cases}
x + y = 1 \\
x + y = 2
\end{cases}

이 시스템은 해가 없습니다.

\begin{enumerate}
\def\labelenumi{\arabic{enumi}.}
\item
  정확히 하나의 해 (고유): 시스템의 방정식이 단일 지점에서 교차합니다.\\
  예: 다음 계수 행렬:
\end{enumerate}

\begin{bmatrix}
1 & 2 \\
3 & -1
\end{bmatrix}

은 고유한 해를 가집니다.

\begin{enumerate}
\def\labelenumi{\arabic{enumi}.}
\item
  무한히 많은 해: 방정식이 겹치는 제약 조건을 설명합니다 (예: 동일한 선이나 평면을 나타내는 여러 방정식).
\end{enumerate}

해의 성격은 \(A\)의 랭크와 첨가 행렬 \((A|\mathbf{b})\)와의 관계에 따라 달라지며, 이는 나중에 연구할 것입니다.

\subsubsection{기하학적 해석}\label{geometric-interpretation-2}

\begin{itemize}
\item
  \(\mathbb{R}^2\)에서 각 선형 방정식은 선을 나타냅니다. 시스템을 푼다는 것은 선들의 교점을 찾는 것을 의미합니다.
\item
  \(\mathbb{R}^3\)에서 각 방정식은 평면을 나타냅니다. 시스템은 해가 없을 수도 있고(평행한 평면), 하나의 해(고유한 교점)를 가질 수도 있고, 무한히 많은 해(교선)를 가질 수도 있습니다.
\item
  고차원에서는 그림이 일반화됩니다: 해는 초평면의 교집합을 형성합니다.
\end{itemize}

\subsubsection{이것이 중요한 이유}\label{why-this-matters-8}

선형 시스템은 선형대수학의 실용적인 기초입니다. 화학 반응의 균형, 회로 분석, 최소 제곱 회귀, 최적화, 컴퓨터 그래픽스에 나타납니다. 해를 표현하고 분류하는 방법을 이해하는 것은 가우스 소거법과 같은 체계적인 해결 방법의 첫 걸음입니다.

\subsubsection{연습문제 3.1}\label{exercises-31}

\begin{enumerate}
\def\labelenumi{\arabic{enumi}.}
\item
  다음 시스템을 행렬 형태로 쓰십시오:
\end{enumerate}

\begin{cases}
2x + 3y - z = 7, \\
x - y + 4z = 1, \\
3x + 2y + z = 5
\end{cases}

\begin{enumerate}
\def\labelenumi{\arabic{enumi}.}
\item
  시스템
\end{enumerate}

\begin{cases}
x + y = 1, \\
2x + 2y = 2
\end{cases}

이 해가 없는지, 하나의 해를 갖는지, 또는 무한히 많은 해를 갖는지 확인하십시오.

\begin{enumerate}
\def\labelenumi{\arabic{enumi}.}
\item
  시스템을 기하학적으로 해석하십시오.
\end{enumerate}

\begin{cases}
x + y = 3, \\
x - y = 1
\end{cases}

평면에서.

\begin{enumerate}
\def\labelenumi{\arabic{enumi}.}
\item
  시스템을 푸십시오.
\end{enumerate}

\begin{cases}
2x + y = 1, \\
x - y = 4
\end{cases}

그리고 해를 확인하십시오.

\begin{enumerate}
\def\labelenumi{\arabic{enumi}.}
\item
  \(\mathbb{R}^3\)에서 다음의 해 집합을 설명하십시오.
\end{enumerate}

\begin{cases}
x + y + z = 0, \\
2x + 2y + 2z = 0
\end{cases}

어떤 기하학적 객체를 나타냅니까?

\subsection{3.2 가우스 소거법}\label{32-gaussian-elimination}

선형 시스템을 효율적으로 풀기 위해 가우스 소거법을 사용합니다: 시스템을 해를 더 쉽게 볼 수 있는 더 간단한 동등한 시스템으로 변환하는 체계적인 방법입니다. 이 방법은 해 집합을 보존하는 기본 행 연산에 의존합니다.

\subsubsection{기본 행 연산}\label{elementary-row-operations}

첨가 행렬 \((A|\mathbf{b})\)에 대해 세 가지 연산이 허용됩니다:

\begin{enumerate}
\def\labelenumi{\arabic{enumi}.}
\item
  행 교환: 두 행을 서로 바꿉니다.
\item
  행 스케일링: 행에 0이 아닌 스칼라를 곱합니다.
\item
  행 교체: 한 행을 자신과 다른 행의 배수를 더한 것으로 교체합니다.
\end{enumerate}

이러한 연산은 방정식을 다르지만 동등한 형태로 다시 표현하는 것에 해당합니다.

\subsubsection{행 사다리꼴 형태}\label{row-echelon-form}

행렬이 행 사다리꼴 형태(REF)에 있으려면 다음을 만족해야 합니다:

\begin{enumerate}
\def\labelenumi{\arabic{enumi}.}
\item
  모든 0이 아닌 행은 0인 행 위에 있습니다.
\item
  각 선행 항목(행에서 왼쪽에서 첫 번째 0이 아닌 숫자)은 위 행의 선행 항목보다 오른쪽에 있습니다.
\item
  선행 항목 아래의 모든 항목은 0입니다.
\end{enumerate}

또한, 각 선행 항목이 1이고 해당 열에서 유일한 0이 아닌 항목이면, 행렬은 기약 행 사다리꼴 형태(RREF)에 있습니다.

\subsubsection{가우스 소거법 알고리즘}\label{algorithm-of-gaussian-elimination}

\begin{enumerate}
\def\labelenumi{\arabic{enumi}.}
\item
  시스템에 대한 첨가 행렬을 씁니다.
\item
  행 연산을 사용하여 각 피벗(행의 선행 항목) 아래에 0을 만듭니다.
\item
  행렬이 사다리꼴 형태가 될 때까지 열별로 계속합니다.
\item
  역대입으로 풉니다: 마지막 피벗 방정식에서 시작하여 위로 올라갑니다.
\end{enumerate}

RREF로 계속하면 해를 직접 읽을 수 있습니다.

\subsubsection{예제}\label{example}

예제 3.2.1. 풀기

\begin{cases}
x + 2y - z = 3, \\
2x + y + z = 7, \\
3x - y + 2z = 4.
\end{cases}

1단계. 첨가 행렬

\[\left[\begin{array}{ccc|c}
1 & 2 & -1 & 3 \\
2 & 1 & 1 & 7 \\
3 & -1 & 2 & 4
\end{array}\right].\]

2단계. 첫 번째 피벗 아래를 소거합니다.

행 2에서 행 1의 2배를 빼고, 행 3에서 행 1의 3배를 뺍니다:

\[\left[\begin{array}{ccc|c}
1 & 2 & -1 & 3 \\
0 & -3 & 3 & 1 \\
0 & -7 & 5 & -5
\end{array}\right].\]

3단계. 열 2의 피벗

행 2를 -3으로 나눕니다:

\[\left[\begin{array}{ccc|c}
1 & 2 & -1 & 3 \\
0 & 1 & -1 & -\tfrac{1}{3} \\
0 & -7 & 5 & -5
\end{array}\right].\]

행 3에 행 2의 7배를 더합니다:

\[\left[\begin{array}{ccc|c}
1 & 2 & -1 & 3 \\
0 & 1 & -1 & -\tfrac{1}{3} \\
0 & 0 & -2 & -\tfrac{22}{3}
\end{array}\right].\]

4단계. 열 3의 피벗

행 3을 -2로 나눕니다:

\[\left[\begin{array}{ccc|c}
1 & 2 & -1 & 3 \\
0 & 1 & -1 & -\tfrac{1}{3} \\
0 & 0 & 1 & \tfrac{11}{3}
\end{array}\right].\]

5단계. 역대입

마지막 행에서:

\[z = \tfrac{11}{3}.\]

두 번째 행:

\[y - z = -\tfrac{1}{3} \implies y = -\tfrac{1}{3} + \tfrac{11}{3} = \tfrac{10}{3}.\]

첫 번째 행:

\[x + 2y - z = 3 \implies x + 2\cdot\tfrac{10}{3} - \tfrac{11}{3} = 3.\]

그래서

\[x + \tfrac{20}{3} - \tfrac{11}{3} = 3 \implies x + 3 = 3 \implies x = 0.\]

해:

\[(x,y,z) = \big(0, \tfrac{10}{3}, \tfrac{11}{3}\big).\]

\subsubsection{이것이 중요한 이유}\label{why-this-matters-9}

가우스 소거법은 계산 선형대수학의 기초입니다. 복잡한 시스템을 해가 보이는 형태로 축소하고, 수치 분석, 과학 계산, 기계 학습에 사용되는 알고리즘의 기초를 형성합니다.

\subsubsection{연습문제 3.2}\label{exercises-32}

\begin{enumerate}
\def\labelenumi{\arabic{enumi}.}
\item
  가우스 소거법으로 푸십시오:
\end{enumerate}

\begin{cases}
x + y = 2, \\
2x - y = 0.
\end{cases}

\begin{enumerate}
\def\labelenumi{\arabic{enumi}.}
\item
  다음 첨가 행렬을 REF로 축소하십시오:
\end{enumerate}

\[\left[\begin{array}{ccc|c}
1 & 1 & 1 & 6 \\
2 & -1 & 3 & 14 \\
1 & 4 & -2 & -2
\end{array}\right].\]

\begin{enumerate}
\def\labelenumi{\arabic{enumi}.}
\item
  가우스 소거법이 항상 다음 중 하나를 생성함을 보이십시오:
\end{enumerate}

\begin{itemize}
\item
  고유한 해,
\item
  무한히 많은 해, 또는
\item
  모순 (해가 없음).
\end{itemize}

\begin{enumerate}
\def\labelenumi{\arabic{enumi}.}
\item
  가우스 소거법을 사용하여 다음의 모든 해를 찾으십시오.
\end{enumerate}

\begin{cases}
x + y + z = 0, \\
2x + y + z = 1.
\end{cases}

\begin{enumerate}
\def\labelenumi{\arabic{enumi}.}
\item
  피벗팅(가장 큰 사용 가능한 피벗 요소를 선택하는 것)이 수치 계산에서 유용한 이유를 설명하십시오.
\end{enumerate}

\subsection{3.3 랭크와 일관성}\label{33-rank-and-consistency}

가우스 소거법은 해를 제공할 뿐만 아니라 선형 시스템의 구조를 드러냅니다. 두 가지 핵심 아이디어는 행렬의 랭크와 시스템의 일관성입니다. 랭크는 방정식의 독립적인 정보의 양을 측정하고, 일관성은 시스템이 적어도 하나의 해를 갖는지 여부를 결정합니다.

\subsubsection{행렬의 랭크}\label{rank-of-a-matrix}

행렬의 랭크는 행 사다리꼴 형태의 선행 피벗의 수입니다. 동등하게, 선형 독립적인 행 또는 열의 최대 수입니다.

형식적으로,

\[\text{rank}(A) = \dim(\text{row space of } A) = \dim(\text{column space of } A).\]

랭크는 행(또는 열)에 의해 생성된 공간의 유효 차원을 알려줍니다.

예제 3.3.1.\\
에 대해

\[A = \begin{bmatrix}
1 & 2 & 3 \\
2 & 4 & 6 \\
3 & 6 & 9
\end{bmatrix},\]

행 축소는 다음을 제공합니다.

\begin{bmatrix}
1 & 2 & 3 \\
0 & 0 & 0 \\
0 & 0 & 0
\end{bmatrix}.

따라서 모든 행이 첫 번째 행의 배수이므로 \(\text{rank}(A) = 1\)입니다.

\subsubsection{선형 시스템의 일관성}\label{consistency-of-linear-systems}

시스템 \(A\mathbf{x} = \mathbf{b}\)를 고려하십시오.\\
시스템은 다음과 같은 경우에만 일관됩니다(적어도 하나의 해를 가짐).

\[\text{rank}(A) = \text{rank}(A|\mathbf{b}),\]

여기서 \((A|\mathbf{b})\)는 첨가 행렬입니다.\\
랭크가 다르면 시스템은 일관되지 않습니다.

\begin{itemize}
\item
  \(\text{rank}(A) = \text{rank}(A|\mathbf{b}) = n\) (미지수의 수)이면 시스템은 고유한 해를 가집니다.
\item
  \(\text{rank}(A) = \text{rank}(A|\mathbf{b}) < n\)이면 시스템은 무한히 많은 해를 가집니다.
\end{itemize}

\subsubsection{예제}\label{example-2}

예제 3.3.2.\\
고려하십시오

\begin{cases}
x + y + z = 1, \\
2x + 2y + 2z = 2, \\
x + y + z = 3.
\end{cases}

첨가 행렬은

\[\left[\begin{array}{ccc|c}
1 & 1 & 1 & 1 \\
2 & 2 & 2 & 2 \\
1 & 1 & 1 & 3
\end{array}\right].\]

행 축소는 다음을 제공합니다.

\[\left[\begin{array}{ccc|c}
1 & 1 & 1 & 1 \\
0 & 0 & 0 & 0 \\
0 & 0 & 0 & 2
\end{array}\right].\]

여기서 \(\text{rank}(A) = 1\)이지만, \(\text{rank}(A|\mathbf{b}) = 2\)입니다. 랭크가 다르므로 시스템은 일관되지 않습니다: 해가 존재하지 않습니다.

\subsubsection{무한한 해를 가진 예제}\label{example-with-infinite-solutions}

예제 3.3.3.\\
에 대해

\begin{cases}
x + y = 2, \\
2x + 2y = 4,
\end{cases}

첨가 행렬은 다음과 같이 축소됩니다.

\[\left[\begin{array}{cc|c}
1 & 1 & 2 \\
0 & 0 & 0
\end{array}\right].\]

여기서 \(\text{rank}(A) = \text{rank}(A|\mathbf{b}) = 1 < 2\)입니다. 따라서 무한히 많은 해가 존재하며, 선을 형성합니다.

\subsubsection{이것이 중요한 이유}\label{why-this-matters-10}

랭크는 독립성의 척도입니다: 얼마나 많은 진정으로 구별되는 방정식이나 방향이 있는지를 알려줍니다. 일관성은 방정식이 정렬될 때와 모순될 때를 설명합니다. 이러한 개념은 선형 시스템을 벡터 공간에 연결하고 차원, 기저 및 랭크-널리티 정리의 아이디어를 준비합니다.

\subsubsection{연습문제 3.3}\label{exercises-33}

\begin{enumerate}
\def\labelenumi{\arabic{enumi}.}
\item
  다음의 랭크를 계산하십시오.
\end{enumerate}

\[A = \begin{bmatrix}
1 & 2 & 1 \\
0 & 1 & -1 \\
2 & 5 & -1
\end{bmatrix}.\]

\begin{enumerate}
\def\labelenumi{\arabic{enumi}.}
\item
  시스템이 일관적인지 확인하십시오.
\begin{cases}
x + y + z = 1, \\
2x + 3y + z = 2, \\
3x + 5y + 2z = 3
\end{cases}
\end{enumerate}
\begin{enumerate}
\def\labelenumi{\arabic{enumi}.}
\item
  단위 행렬 \(I_n\)의 랭크가 \(n\)임을 보이십시오.
\item
  \(\mathbb{R}^3\)에서 무한히 많은 해를 가진 시스템의 예를 들고, 랭크 조건을 만족하는 이유를 설명하십시오.
\item
  임의의 행렬 \(A \in \mathbb{R}^{m \times n}\)에 대해, \(
  \text{rank}(A) \leq \min(m,n)
  \)임을 증명하십시오.
\end{enumerate}

\subsection{3.4 동차 시스템}\label{34-homogeneous-systems}

동차 시스템은 모든 상수 항이 0인 선형 시스템입니다:

\[A\mathbf{x} = \mathbf{0},\]

여기서 \(A \in \mathbb{R}^{m \times n}\)이고, \(\mathbf{0}\)는 \(\mathbb{R}^m\)의 영 벡터입니다.

\subsubsection{자명한 해}\label{the-trivial-solution}

모든 동차 시스템은 적어도 하나의 해를 가집니다:

\[\mathbf{x} = \mathbf{0}.\]

이것은 자명한 해라고 불립니다. 흥미로운 질문은 \emph{비자명한 해}(0이 아닌 벡터)가 존재하는지 여부입니다.

\subsubsection{비자명한 해의 존재}\label{existence-of-nontrivial-solutions}

비자명한 해는 미지수의 수가 계수 행렬의 랭크를 초과할 때 정확히 존재합니다:

\[\text{rank}(A) < n.\]

이 경우, 무한히 많은 해가 있으며, \(\mathbb{R}^n\)의 부분 공간을 형성합니다. 이 해 공간의 차원은

\[\dim(\text{null}(A)) = n - \text{rank}(A),\]

여기서 null(A)는 \(A\mathbf{x} = 0\)에 대한 모든 해의 집합입니다. 이 집합은 \(A\)의 영공간 또는 커널이라고 불립니다.

\subsubsection{예제}\label{example-3}

예제 3.4.1.\\
고려하십시오

\begin{cases}
x + y + z = 0, \\
2x + y - z = 0.
\end{cases}

첨가 행렬은

\[\left[\begin{array}{ccc|c}
1 & 1 & 1 & 0 \\
2 & 1 & -1 & 0
\end{array}\right].\]

행 축소:

\[\left[\begin{array}{ccc|c}
1 & 1 & 1 & 0 \\
0 & -1 & -3 & 0
\end{array}\right]
\quad\to\quad
\left[\begin{array}{ccc|c}
1 & 1 & 1 & 0 \\
0 & 1 & 3 & 0
\end{array}\right].\]

따라서 시스템은 다음과 동등합니다:

\begin{cases}
x + y + z = 0, \\
y + 3z = 0.
\end{cases}

두 번째 방정식에서, \(y = -3z\). 첫 번째에 대입:\\
\(
x - 3z + z = 0 \implies x = 2z.
\)

따라서 해는 다음과 같습니다:

\[(x,y,z) = z(2, -3, 1), \quad z \in \mathbb{R}.\]

영공간은 벡터 \((2, -3, 1)\)에 의해 생성된 선입니다.

\subsubsection{기하학적 해석}\label{geometric-interpretation-3}

동차 시스템의 해 집합은 항상 \(\mathbb{R}^n\)의 부분 공간입니다.

\begin{itemize}
\item
  \(\text{rank}(A) = n\)이면, 유일한 해는 영 벡터입니다.
\item
  \(\text{rank}(A) = n-1\)이면, 해 집합은 원점을 통과하는 선입니다.
\item
  \(\text{rank}(A) = n-2\)이면, 해 집합은 원점을 통과하는 평면입니다.
\end{itemize}

더 일반적으로, 영공간은 차원 \(n - \text{rank}(A)\)를 가지며, 이를 널리티라고 합니다.

\subsubsection{이것이 중요한 이유}\label{why-this-matters-11}

동차 시스템은 벡터 공간, 부분 공간, 차원을 이해하는 데 중심적입니다. 커널, 영공간, 선형 종속의 개념으로 직접 이어집니다. 응용 분야에서 동차 시스템은 평형 문제, 고유값 방정식, 컴퓨터 그래픽스 변환에 나타납니다.

\subsubsection{연습문제 3.4}\label{exercises-34}

\begin{enumerate}
\def\labelenumi{\arabic{enumi}.}
\item
  동차 시스템을 푸십시오.
\end{enumerate}

\begin{cases}
x + 2y - z = 0, \\
2x + 4y - 2z = 0.
\end{cases}

해 공간의 차원은 얼마입니까?

\begin{enumerate}
\def\labelenumi{\arabic{enumi}.}
\item
  다음의 모든 해를 찾으십시오.
\end{enumerate}

\begin{cases}
x - y + z = 0, \\
2x + y - z = 0.
\end{cases}

\begin{enumerate}
\def\labelenumi{\arabic{enumi}.}
\item
  임의의 동차 시스템의 해 집합이 \(\mathbb{R}^n\)의 부분 공간임을 보이십시오.
\item
  \(A\)가 \(\textbackslash text\{rank\}(A) = 2\)인 \$3 \textbackslash times 3\$ 행렬이라고 가정합니다. \(A\)의 영공간의 차원은 얼마입니까?
\item
  에 대해
\end{enumerate}

\[A = \begin{bmatrix} 1 & 2 & -1 \\ 0 & 1 & 3 \end{bmatrix},\]

\(A\)의 영공간에 대한 기저를 계산하십시오.\\

\section{4장. 벡터 공간}\label{chapter-4-vector-spaces}

\subsection{4.1 벡터 공간의 정의}\label{41-definition-of-a-vector-space}

지금까지 우리는 \(\mathbb{R}^n\)에서 구체적으로 벡터와 행렬을 연구했습니다. 다음 단계는 좌표를 넘어서 벡터 공간을 완전히 일반적으로 정의하는 것입니다. 벡터 공간은 덧셈과 스칼라 곱셈의 익숙한 규칙이 적용되는 추상적인 설정으로, 요소가 기하학적 벡터, 다항식, 함수 또는 다른 객체인지 여부에 관계없이 적용됩니다.

\subsubsection{형식적 정의}\label{formal-definition-2}

실수 \(\mathbb{R}\) 상의 벡터 공간은 두 가지 연산이 갖추어진 집합 \(V\)입니다:

\begin{enumerate}
\def\labelenumi{\arabic{enumi}.}
\item
  벡터 덧셈: 임의의 \(\mathbf{u}, \mathbf{v} \in V\)에 대해, 벡터 \(\mathbf{u} + \mathbf{v} \in V\)가 있습니다.
\item
  스칼라 곱셈: 임의의 스칼라 \(c \in \mathbb{R}\)와 임의의 \(\mathbf{v} \in V\)에 대해, 벡터 \(c\mathbf{v} \in V\)가 있습니다.
\end{enumerate}

이러한 연산은 다음 공리(모든 \(\mathbf{u}, \mathbf{v}, \mathbf{w} \in V\)와 모든 스칼라 \(a,b \in \mathbb{R}\)에 대해)를 만족해야 합니다:

\begin{enumerate}
\def\labelenumi{\arabic{enumi}.}
\item
  덧셈의 교환 법칙: \(\mathbf{u} + \mathbf{v} = \mathbf{v} + \mathbf{u}\).
\item
  덧셈의 결합 법칙: \((\mathbf{u} + \mathbf{v}) + \mathbf{w} = \mathbf{u} + (\mathbf{v} + \mathbf{w})\).
\item
  덧셈 항등원: \(\mathbf{v} + \mathbf{0} = \mathbf{v}\)를 만족하는 영 벡터 \(\mathbf{0} \in V\)가 존재합니다.
\item
  덧셈 역원: 각 \(\mathbf{v} \in V\)에 대해, \(\mathbf{v} + (-\mathbf{v}) = \mathbf{0}\)를 만족하는 \((-\mathbf{v} \in V\)가 존재합니다.
\item
  스칼라 곱셈의 호환성: \(a(b\mathbf{v}) = (ab)\mathbf{v}\).
\item
  스칼라의 항등원: \$1 \textbackslash cdot \textbackslash mathbf\{v\} = \textbackslash mathbf\{v\}\$.
\item
  벡터 덧셈에 대한 분배 법칙: \(a(\mathbf{u} + \mathbf{v}) = a\mathbf{u} + a\mathbf{v}\).
\item
  스칼라 덧셈에 대한 분배 법칙: \((a+b)\mathbf{v} = a\mathbf{v} + b\mathbf{v}\).
\end{enumerate}

만약 집합 \(V\)가 연산과 함께 모든 8개의 공리를 만족하면, 우리는 그것을 벡터 공간이라고 부릅니다.

\subsubsection{예제}\label{examples-2}

예제 4.1.1. 표준 유클리드 공간\\
일반적인 덧셈과 스칼라 곱셈을 갖춘 \(\mathbb{R}^n\)은 벡터 공간입니다. 이것은 공리가 추상화된 모델 케이스입니다.

예제 4.1.2. 다항식\\
실수 계수를 가진 모든 다항식의 집합, \(\mathbb{R}[x]\)로 표기되며, 벡터 공간을 형성합니다. 덧셈과 스칼라 곱셈은 항별로 정의됩니다.

예제 4.1.3. 함수\\
구간에서 모든 실수 값 함수의 집합, 예: \(f: [0,1] \to \mathbb{R}\)는 벡터 공간을 형성합니다. 왜냐하면 함수는 점별로 더하고 스케일링할 수 있기 때문입니다.

\subsubsection{반례}\label{non-examples}

모든 집합이 연산과 함께 자격이 있는 것은 아닙니다. 예를 들어, 일반적인 덧셈 하에서 양의 실수의 집합은 벡터 공간이 아닙니다. 왜냐하면 덧셈 역원(음수)이 없기 때문입니다. 모든 공리가 성립해야 합니다.

\subsubsection{기하학적 해석}\label{geometric-interpretation-4}

\(\mathbb{R}^2\)나 \(\mathbb{R}^3\)과 같은 익숙한 경우에서 벡터 공간은 기하학의 무대를 제공합니다: 벡터는 더하고, 스케일링하고, 결합하여 선, 평면 및 고차원 구조를 형성할 수 있습니다. 함수 공간과 같은 추상적인 설정에서 동일한 대수 규칙을 통해 기하학적 직관을 무한 차원 문제에 적용할 수 있습니다.

\subsubsection{이것이 중요한 이유}\label{why-this-matters-12}

벡터 공간의 개념은 겉보기에 다른 수학적 객체를 단일 프레임워크 아래에서 통합합니다. 물리학의 힘, 공학의 신호, 기계 학습의 데이터를 다루든, 벡터 공간의 공통 언어를 통해 모든 곳에서 동일한 기술을 사용할 수 있습니다.

\subsubsection{연습문제 4.1}\label{exercises-41}

\begin{enumerate}
\def\labelenumi{\arabic{enumi}.}
\item
  \(\mathbb{R}^2\)가 표준 덧셈과 스칼라 곱셈으로 모든 8개의 벡터 공간 공리를 만족하는지 확인하십시오.
\item
  정수 집합 \(\mathbb{Z}\)가 일반적인 연산으로 \(\mathbb{R}\) 상의 벡터 공간이 아님을 보이십시오. 어떤 공리가 실패합니까?
\item
  최대 3차 다항식의 모든 집합을 고려하십시오. 그것이 \(\mathbb{R}\) 상의 벡터 공간을 형성함을 보이십시오. 그 차원은 얼마입니까?
\item
  벡터가 기하학적 객체가 아닌 벡터 공간의 예를 드십시오.
\item
  임의의 벡터 공간에서 영 벡터가 고유함을 증명하십시오.
\end{enumerate}

\subsection{4.2 부분 공간}\label{42-subspaces}

부분 공간은 더 큰 벡터 공간 내에 사는 더 작은 벡터 공간입니다. 선과 평면이 3차원 공간 내에 자연스럽게 위치하는 것처럼, 부분 공간은 이러한 아이디어를 고차원 및 더 추상적인 설정으로 일반화합니다.

\subsubsection{정의}\label{definition-2}

\(V\)를 벡터 공간이라고 합시다. 부분 집합 \(W \subseteq V\)가 \(V\)의 부분 공간이라고 불리려면 다음을 만족해야 합니다:

\begin{enumerate}
\def\labelenumi{\arabic{enumi}.}
\item
  \(\mathbf{0} \in W\) (영 벡터를 포함),
\item
  모든 \(\mathbf{u}, \mathbf{v} \in W\)에 대해, 합 \(\mathbf{u} + \mathbf{v} \in W\) (덧셈에 대해 닫혀 있음),
\item
  모든 스칼라 \(c \in \mathbb{R}\)와 벡터 \(\mathbf{v} \in W\)에 대해, 곱 \(c\mathbf{v} \in W\) (스칼라 곱셈에 대해 닫혀 있음).
\end{enumerate}

이것들이 성립하면, \(W\)는 상속된 연산으로 그 자체로 벡터 공간입니다.

\subsubsection{예제}\label{examples-3}

예제 4.2.1. \(\mathbb{R}^2\)에서 원점을 통과하는 선\\
집합

\[W = \{ (t, 2t) \mid t \in \mathbb{R} \}\]

은 \(\mathbb{R}^2\)의 부분 공간입니다. 영 벡터를 포함하고, 덧셈에 대해 닫혀 있으며, 스칼라 곱셈에 대해 닫혀 있습니다.

예제 4.2.2. \(\mathbb{R}^3\)의 x-y 평면\\
집합

\[W = \{ (x, y, 0) \mid x,y \in \mathbb{R} \}\]

은 \(\mathbb{R}^3\)의 부분 공간입니다. 원점을 통과하고 x-y 평면에 평행한 평면에 놓인 모든 벡터의 모음입니다.

예제 4.2.3. 행렬의 영공간\\
행렬 \(A \in \mathbb{R}^{m \times n}\)에 대해, 영공간

\[\{ \mathbf{x} \in \mathbb{R}^n \mid A\mathbf{x} = \mathbf{0} \}\]

은 \(\mathbb{R}^n\)의 부분 공간입니다. 이 부분 공간은 동차 시스템의 모든 해를 나타냅니다.

\subsubsection{반례}\label{non-examples-2}

모든 부분 집합이 부분 공간인 것은 아닙니다.

\begin{itemize}
\item
  집합 \({ (x,y) \in \mathbb{R}^2 \mid x \geq 0 }\)은 부분 공간이 아닙니다: 스칼라 곱셈에 대해 닫혀 있지 않습니다 (음수 스칼라는 조건을 깨뜨립니다).
\item
  원점을 통과하지 않는 \(\mathbb{R}^2\)의 모든 선은 부분 공간이 아닙니다. 왜냐하면 \(\mathbf{0}\)을 포함하지 않기 때문입니다.
\end{itemize}

\subsubsection{기하학적 해석}\label{geometric-interpretation-5}

부분 공간은 벡터 공간 내부의 선형 구조입니다.

\begin{itemize}
\item
  \(\mathbb{R}^2\)에서 부분 공간은: 영 벡터, 원점을 통과하는 모든 선, 또는 전체 평면입니다.
\item
  \(\mathbb{R}^3\)에서 부분 공간은: 영 벡터, 원점을 통과하는 모든 선, 원점을 통과하는 모든 평면, 또는 전체 공간입니다.
\item
  고차원에서는 동일한 원칙이 적용됩니다: 부분 공간은 원점을 통과하는 평평한 선형 조각입니다.
\end{itemize}

\subsubsection{이것이 중요한 이유}\label{why-this-matters-13}

부분 공간은 선형 문제의 본질적인 구조를 포착합니다. 열 공간, 행 공간, 영공간은 모두 부분 공간입니다. 선형대수학의 많은 부분은 이러한 부분 공간이 어떻게 교차하고, 생성하고, 서로를 보완하는지 이해하는 것으로 구성됩니다.

\subsubsection{연습문제 4.2}\label{exercises-42}

\begin{enumerate}
\def\labelenumi{\arabic{enumi}.}
\item
  집합 \(W = { (x,0) \mid x \in \mathbb{R} } \subseteq \mathbb{R}^2\)가 부분 공간임을 증명하십시오.
\item
  선 \({ (1+t, 2t) \mid t \in \mathbb{R} }\)이 \(\mathbb{R}^2\)의 부분 공간이 아님을 보이십시오. 어떤 조건이 실패합니까?
\item
  \(x+y+z=0\)을 만족하는 모든 벡터 \((x,y,z) \in \mathbb{R}^3\)의 집합이 부분 공간인지 확인하십시오.
\item
  행렬
\end{enumerate}

\[A = \begin{bmatrix}
1 & 2 & 3 \\
4 & 5 & 6
\end{bmatrix}\]

에 대해 \(A\)의 영공간을 \(\mathbb{R}^3\)의 부분 공간으로 설명하십시오.

\begin{enumerate}
\def\labelenumi{\arabic{enumi}.}
\item
  \(\mathbb{R}^2\)의 모든 가능한 부분 공간을 나열하십시오.
\end{enumerate}

\subsection{4.3 스팬, 기저, 차원}\label{43-span-basis-dimension}

스팬, 기저, 차원의 아이디어는 부분 공간의 크기와 구조를 설명하는 언어를 제공합니다. 함께, 그들은 벡터 공간이 어떻게 생성되는지, 얼마나 많은 구성 요소가 필요한지, 그리고 그 구성 요소를 어떻게 선택할 수 있는지 알려줍니다.

\subsubsection{스팬}\label{span}

벡터 집합 \({\mathbf{v}_1, \mathbf{v}_2, \dots, \mathbf{v}_k} \subseteq V\)가 주어지면, 스팬은 모든 선형 결합의 모음입니다:

\[\text{span}\{\mathbf{v}_1, \dots, \mathbf{v}_k\} = \{ c_1\mathbf{v}_1 + \cdots + c_k\mathbf{v}_k \mid c_i \in \mathbb{R} \}.\]

스팬은 항상 \(V\)의 부분 공간이며, 즉 해당 벡터를 포함하는 가장 작은 부분 공간입니다.

예제 4.3.1.\\
\(\mathbb{R}^2\)에서, \( \text{span}{(1,0)} = \{(x,0) \mid x \in \mathbb{R}\},\) 즉 x축입니다.\\
마찬가지로, \(\text{span}\{(1,0),(0,1)\} = \mathbb{R}^2\)입니다.

\subsubsection{기저}\label{basis}

벡터 공간 \(V\)의 기저는 다음을 만족하는 벡터 집합입니다:

\begin{enumerate}
\def\labelenumi{\arabic{enumi}.}
\item
  \(V\)를 스팬합니다.
\item
  선형 독립입니다 (집합의 어떤 벡터도 다른 벡터의 선형 결합이 아님).
\end{enumerate}

어느 조건이라도 실패하면, 그 집합은 기저가 아닙니다.

예제 4.3.2.\\
\(\mathbb{R}^3\)에서, 표준 단위 벡터

\[\mathbf{e}_1 = (1,0,0), \quad \mathbf{e}_2 = (0,1,0), \quad \mathbf{e}_3 = (0,0,1)\]

는 기저를 형성합니다. 모든 벡터 \((x,y,z)\)는 다음과 같이 고유하게 쓸 수 있습니다.

\[x\mathbf{e}_1 + y\mathbf{e}_2 + z\mathbf{e}_3.\]

\subsubsection{차원}\label{dimension}

벡터 공간 \(V\)의 차원, \(\dim(V)\)로 쓰며, \(V\)의 임의의 기저에 있는 벡터의 수입니다. 이 수는 잘 정의되어 있습니다: 벡터 공간의 모든 기저는 동일한 카디널리티를 가집니다.

예제 4.3.3.

\begin{itemize}
\item
  \(\dim(\mathbb{R}^2) = 2\), 기저 \((1,0), (0,1)\)을 가짐.
\item
  \(\dim(\mathbb{R}^3) = 3\), 기저 \((1,0,0), (0,1,0), (0,0,1)\)을 가짐.
\item
  최대 3차 다항식 집합은 차원 4를 가지며, 기저 \((1, x, x^2, x^3)\)을 가짐.
\end{itemize}

\subsubsection{기하학적 해석}\label{geometric-interpretation-6}

\begin{itemize}
\item
  스팬은 벡터 집합의 도달 범위와 같습니다.
\item
  기저는 공간의 모든 것에 도달하는 데 필요한 최소한의 방향 집합입니다.
\item
  차원은 이러한 독립적인 방향의 수입니다.
\end{itemize}

선, 평면, 고차원 평면은 모두 스팬, 기저, 차원의 관점에서 설명될 수 있습니다.

\subsubsection{이것이 중요한 이유}\label{why-this-matters-14}

이러한 개념은 벡터 공간과 부분 공간을 크기와 구조의 관점에서 분류합니다. 랭크-널리티 정리와 같은 선형대수학의 많은 정리는 스팬, 기저, 차원을 이해한 결과입니다. 실용적인 용어로, 기저는 좌표로 데이터를 인코딩하는 방법이며, 차원은 시스템이 실제로 얼마나 많은 자유를 가지고 있는지 알려줍니다.

\subsubsection{연습문제 4.3}\label{exercises-43}

\begin{enumerate}
\def\labelenumi{\arabic{enumi}.}
\item
  \((1,0,0)\), \((0,1,0)\), \((1,1,0)\)이 \(\mathbb{R}^3\)에서 \(xy\)-평면을 스팬함을 보이십시오. 기저입니까?
\item
  \(\mathbb{R}^3\)에서 선 \(\{(2t,-3t,t) : t \in \mathbb{R}\}\)에 대한 기저를 찾으십시오.
\item
  \(x+y+z=0\)으로 정의된 \(\mathbb{R}^3\)의 부분 공간의 차원을 결정하십시오.
\item
  \(\mathbb{R}^n\)의 임의의 두 다른 기저는 정확히 \(n\)개의 벡터를 포함해야 함을 증명하십시오.
\item
  최대 2차 다항식 집합에 대한 기저를 제시하십시오. 그 차원은 얼마입니까?
\end{enumerate}

\subsection{4.4 좌표}\label{44-coordinates}

벡터 공간에 대한 기저가 선택되면, 모든 벡터는 기저 벡터의 선형 결합으로 고유하게 표현될 수 있습니다. 이 결합의 계수를 해당 기저에 대한 벡터의 좌표라고 합니다. 좌표를 통해 우리는 벡터 공간의 추상적인 세계와 숫자의 구체적인 세계 사이를 이동할 수 있습니다.

\subsubsection{기저에 대한 좌표}\label{coordinates-relative-to-a-basis}

\(V\)를 벡터 공간이라고 하고,

\[\mathcal{B} = \{\mathbf{v}_1, \mathbf{v}_2, \dots, \mathbf{v}_n\}\]

를 \(V\)에 대한 순서 있는 기저라고 합시다. 모든 벡터 \(\mathbf{u} \in V\)는 다음과 같이 고유하게 쓸 수 있습니다.

\[\mathbf{u} = c_1 \mathbf{v}_1 + c_2 \mathbf{v}_2 + \cdots + c_n \mathbf{v}_n.\]

스칼라 \((c_1, c_2, \dots, c_n)\)는 \(\mathcal{B}\)에 대한 \(\mathbf{u}\)의 좌표이며, 다음과 같이 씁니다.

\[[\mathbf{u}]_{\mathcal{B}} = \begin{bmatrix} c_1 \\ c_2 \\ \vdots \\ c_n \end{bmatrix}.\]

\subsubsection{\texorpdfstring{\(\mathbb{R}^2\)에서의 예제}{Example in \textbackslash mathbb\{R\}\^{}2}}\label{example-in--r-2}

예제 4.4.1.\\
기저를 다음과 같이 합시다.

\[\mathcal{B} = \{ (1,1), (1,-1) \}.\]

\(\mathcal{B}\)에 대한 \(\mathbf{u} = (3,1)\)의 좌표를 찾으려면, 다음을 푸십시오.

\[(3,1) = c_1(1,1) + c_2(1,-1).\]

이것은 다음 시스템을 제공합니다.

\begin{cases}
c_1 + c_2 = 3, \\
c_1 - c_2 = 1.
\end{cases}

더하면: \$2c\_1 = 4 \textbackslash implies c\_1 = 2\$. 그러면 \)c\_2 = 1\$.

그래서,

\[[\mathbf{u}]_{\mathcal{B}} = \begin{bmatrix} 2 \\ 1 \end{bmatrix}.\]

\subsubsection{표준 좌표}\label{standard-coordinates}

\(\mathbb{R}^n\)에서 표준 기저는 다음과 같습니다.

\[\mathbf{e}_1 = (1,0,\dots,0), \quad \mathbf{e}_2 = (0,1,0,\dots,0), \dots, \mathbf{e}_n = (0,\dots,0,1).\]

이 기저에 대해 벡터의 좌표는 단순히 그 항목입니다. 따라서 열 벡터는 기본적으로 좌표 표현입니다.

\subsubsection{기저 변경}\label{change-of-basis}

만약 \(\mathcal{B} = {\mathbf{v}_1, \dots, \mathbf{v}_n}\)가 \(\mathbb{R}^n\)의 기저이면, 기저 변경 행렬은 다음과 같습니다.

\[P = \begin{bmatrix} \mathbf{v}_1 & \mathbf{v}_2 & \cdots & \mathbf{v}_n \end{bmatrix},\]

기저 벡터를 열로 가집니다. 임의의 벡터 \(\mathbf{u}\)에 대해,

\[\mathbf{u} = P [\mathbf{u}]_{\mathcal{B}}, \qquad [\mathbf{u}]_{\mathcal{B}} = P^{-1}\mathbf{u}.\]

따라서 기저 간 전환은 행렬 곱셈으로 축소됩니다.

\subsubsection{기하학적 해석}\label{geometric-interpretation-7}

좌표는 선택된 방향 집합에 대한 벡터의 주소입니다. 다른 기저는 다른 좌표계와 같습니다: 데카르트, 회전, 기울어지거나 스케일링된 좌표계. 동일한 벡터는 기저에 따라 수치적으로 매우 다르게 보일 수 있지만, 기하학적 정체성은 변경되지 않습니다.

\subsubsection{이것이 중요한 이유}\label{why-this-matters-15}

좌표는 추상 벡터를 구체적인 수치 데이터로 바꿉니다. 기저 변경은 축의 회전, 행렬의 대각화, 데이터 과학의 주성분 분석에 대한 대수적 언어입니다. 좌표의 숙달은 기하학, 대수학, 계산 사이를 유동적으로 이동하는 데 필수적입니다.

\subsubsection{연습문제 4.4}\label{exercises-44}

\begin{enumerate}
\def\labelenumi{\arabic{enumi}.}
\item
  \((4,2)\)를 기저 \((1,1), (1,-1)\)로 표현하십시오.
\item
  \(\mathbb{R}^3\)의 표준 기저에 대한 \((1,2,3)\)의 좌표를 찾으십시오.
\item
  \(\mathcal{B} = \{(2,0), (0,3)\}\)이면, \([ (4,6) ]_{\mathcal{B}}\)를 계산하십시오.
\item
  \(\mathbb{R}^2\)의 표준 기저에서 \(\mathcal{B} = \{(1,1), (1,-1)\}\)로의 기저 변경 행렬을 구성하십시오.
\item
  기저에 대한 좌표 표현이 고유함을 증명하십시오.
\end{enumerate}

\section{5장. 선형 변환}\label{chapter-5-linear-transformations}

\subsection{5.1 선형성을 보존하는 함수}\label{51-functions-that-preserve-linearity}

선형대수학의 중심 주제는 벡터 공간 간의 함수로서 그 대수적 구조를 보존하는 선형 변환을 이해하는 것입니다. 이러한 변환은 행렬 곱셈의 아이디어를 일반화하고 선형 행동의 본질을 포착합니다.

\subsubsection{정의}\label{definition-3}

\(V\)와 \(W\)를 \(\mathbb{R}\) 상의 벡터 공간이라고 합시다. 함수

\[T : V \to W\]

는 모든 벡터 \(\mathbf{u}, \mathbf{v} \in V\)와 모든 스칼라 \(c \in \mathbb{R}\)에 대해 다음을 만족하면 선형 변환(또는 선형 맵)이라고 불립니다:

\begin{enumerate}
\def\labelenumi{\arabic{enumi}.}
\item
  가산성:
\end{enumerate}

\[T(\mathbf{u} + \mathbf{v}) = T(\mathbf{u}) + T(\mathbf{v}),\]

\begin{enumerate}
\def\labelenumi{\arabic{enumi}.}
\item
  동차성:
\end{enumerate}

\[T(c\mathbf{u}) = cT(\mathbf{u}).\]

두 조건이 모두 성립하면, \(T\)는 자동으로 선형 결합을 존중합니다:

\[T(c_1\mathbf{v}_1 + \cdots + c_k\mathbf{v}_k) = c_1 T(\mathbf{v}_1) + \cdots + c_k T(\mathbf{v}_k).\]

\subsubsection{예제}\label{examples-4}

예제 5.1.1. \(\mathbb{R}^2\)에서의 스케일링.\\
\(T:\mathbb{R}^2 \to \mathbb{R}^2\)를 다음과 같이 정의합시다.

\[T(x,y) = (2x, 2y).\]

이것은 모든 벡터의 길이를 두 배로 늘리면서 방향을 보존합니다. 이것은 선형입니다.

예제 5.1.2. 회전.

\(R_\theta: \mathbb{R}^2 \to \mathbb{R}^2\)를 다음과 같이 합시다.

\[R_\theta(x,y) = (x\cos\theta - y\sin\theta, \; x\sin\theta + y\cos\theta).\]

이것은 벡터를 각도 \(\theta\)만큼 회전시킵니다. 가산성과 동차성을 만족하므로 선형입니다.

예제 5.1.3. 미분.

\(D: \mathbb{R}[x] \to \mathbb{R}[x]\)를 미분이라고 합시다: \(D(p(x)) = p'(x)\).

도함수는 덧셈과 스칼라 곱을 존중하므로, 미분은 선형 변환입니다.

\subsubsection{반례}\label{non-example}

맵 \(S:\mathbb{R}^2 \to \mathbb{R}^2\)를 다음과 같이 정의합니다.

\[S(x,y) = (x^2, y^2)\]

는 일반적으로 \(S(\mathbf{u} + \mathbf{v}) \neq S(\mathbf{u}) + S(\mathbf{v})\)이므로 선형이 아닙니다.

\subsubsection{기하학적 해석}\label{geometric-interpretation-8}

선형 변환은 정확히 원점, 원점을 통과하는 선, 그리고 그 선들을 따라 비례를 보존하는 것들입니다. 익숙한 연산들을 포함합니다: 스케일링, 회전, 반사, 전단, 투영. 비선형 변환은 공간을 구부리거나 휘게 하여 이러한 속성을 깨뜨립니다.

\subsubsection{이것이 중요한 이유}\label{why-this-matters-16}

선형 변환은 기하학, 대수학, 계산을 통합합니다. 행렬이 벡터에 어떻게 작용하는지, 데이터가 어떻게 회전되거나 투영될 수 있는지, 시스템이 선형 규칙 하에서 어떻게 진화하는지를 설명합니다. 선형대수학의 많은 부분은 이러한 변환, 그 표현, 그리고 그 불변량을 이해하는 데 전념합니다.

\subsubsection{연습문제 5.1}\label{exercises-51}

\begin{enumerate}
\def\labelenumi{\arabic{enumi}.}
\item
  \(T(x,y) = (3x-y, 2y)\)가 \(\mathbb{R}^2\) 상의 선형 변환임을 확인하십시오.
\item
  \(T(x,y) = (x+1, y)\)가 선형이 아님을 보이십시오. 어떤 공리가 실패합니까?
\item
  \(T\)와 \(S\)가 선형 변환이면, \(T+S\)도 선형 변환임을 증명하십시오.
\item
  \(\mathbb{R}^3\)에서 \(\mathbb{R}^2\)로의 선형 변환의 예를 드십시오.
\item
  \(T:\mathbb{R}[x] \to \mathbb{R}[x]\)를 적분이라고 합시다:
\end{enumerate}

\[T(p(x)) = \int_0^x p(t)\\,dt.\]

\(T\)가 선형 변환임을 증명하십시오.

\subsection{5.2 선형 맵의 행렬 표현}\label{52-matrix-representation-of-linear-maps}

유한 차원 벡터 공간 간의 모든 선형 변환은 행렬로 표현될 수 있습니다. 이 대응은 선형대수학의 중심 통찰 중 하나입니다: 추상적인 변환을 연구하기 위해 행렬 산술의 도구를 사용할 수 있게 해줍니다.

\subsubsection{선형 맵에서 행렬로}\label{from-linear-map-to-matrix}

\(T: \mathbb{R}^n \to \mathbb{R}^m\)를 선형 변환이라고 합시다. \(\mathbb{R}^n\)의 표준 기저 \(\{ \mathbf{e}_1, \dots, \mathbf{e}_n \}\)를 선택합시다. 여기서 \(\mathbf{e}_i\)는 \(i\)번째 위치에 1이 있고 다른 곳에는 0이 있습니다.

각 기저 벡터에 대한 \(T\)의 작용은 전체 변환을 결정합니다:

\[T(\mathbf{e}\_j) = \begin{bmatrix}
a_{1j} \\
a_{2j} \\
\vdots \\
a_{mj} \end{bmatrix}.\]

이러한 출력을 열로 배치하면 \(T\)의 행렬이 됩니다:

\[[T] = A = \begin{bmatrix}
a_{11} & a_{12} & \cdots & a_{1n} \\
a_{21} & a_{22} & \cdots & a_{2n} \\
\vdots & \vdots & \ddots & \vdots \\
a_{m1} & a_{m2} & \cdots & a_{mn}
\end{bmatrix}.\]

그러면 임의의 벡터 \(\mathbf{x} \in \mathbb{R}^n\)에 대해:

\[T(\mathbf{x}) = A\mathbf{x}.\]

\subsubsection{예제}\label{examples-5}

예제 5.2.1. \(\mathbb{R}^2\)에서의 스케일링.\\
\(T(x,y) = (2x, 3y)\)라고 합시다. 그러면

\[T(\mathbf{e}_1) = (2,0), \quad T(\mathbf{e}_2) = (0,3).\]

그래서 행렬은

\[[T] = \begin{bmatrix}
2 & 0 \\
0 & 3
\end{bmatrix}.\]

예제 5.2.2. 평면에서의 회전.\\
회전 변환 \(R_\theta(x,y) = (x\cos\theta - y\sin\theta, \; x\sin\theta + y\cos\theta)\)는 행렬

\[[R_\theta] = \begin{bmatrix}
\cos\theta & -\sin\theta \\
\sin\theta & \cos\theta
\end{bmatrix}.\]

을 가집니다.

예제 5.2.3. x축으로의 투영.\\
맵 \(P(x,y) = (x,0)\)은

\[[P] = \begin{bmatrix}
1 & 0 \\
0 & 0
\end{bmatrix}.\]

에 해당합니다.

\subsubsection{기저 변경}\label{change-of-basis-2}

행렬 표현은 선택된 기저에 따라 달라집니다. \(\mathcal{B}\)와 \(\mathcal{C}\)가 \(\mathbb{R}^n\)와 \(\mathbb{R}^m\)의 기저이면, 이들 기저에 대한 \(T: \mathbb{R}^n \to \mathbb{R}^m\)의 행렬은 각 \(\mathbf{v}_j \in \mathcal{B}\)에 대해 \(T(\mathbf{v}_j)\)를 \(\mathcal{C}\)로 표현하여 얻습니다. 기저를 변경하는 것은 적절한 기저 변경 행렬로 행렬을 켤레 변환하는 것에 해당합니다.

\subsubsection{기하학적 해석}\label{geometric-interpretation-9}

행렬은 단지 편리한 표기법이 아닙니다 - 기저가 고정되면 \emph{선형 맵}입니다. 모든 회전, 반사, 투영, 전단 또는 스케일링은 특정 행렬을 곱하는 것에 해당합니다. 따라서 선형 변환을 연구하는 것은 그 행렬을 연구하는 것으로 귀결됩니다.

\subsubsection{이것이 중요한 이유}\label{why-this-matters-17}

행렬 표현은 선형 변환을 계산 가능하게 만듭니다. 추상적인 정의를 명시적인 계산에 연결하여 시스템 해결, 고유값 찾기, 분해 수행을 위한 알고리즘을 가능하게 합니다. 그래픽스에서 기계 학습에 이르기까지 응용 프로그램은 이 변환에 의존합니다.

\subsubsection{연습문제 5.2}\label{exercises-52}

\begin{enumerate}
\def\labelenumi{\arabic{enumi}.}
\item
  \(T:\mathbb{R}^2 \to \mathbb{R}^2\), \(T(x,y) = (x+y, x-y)\)의 행렬 표현을 찾으십시오.
\item
  선형 변환 \(T:\mathbb{R}^3 \to \mathbb{R}^2\), \(T(x,y,z) = (x+z, y-2z)\)의 행렬을 결정하십시오.
\item
  \(\mathbb{R}^2\)에서 선 \(y=x\)에 대한 반사를 나타내는 행렬은 무엇입니까?
\item
  \(\mathbb{R}^n\)에 대한 항등 변환의 행렬이 \(I_n\)임을 보이십시오.
\item
  미분 맵 \(D:\mathbb{R}_2[x] \to \mathbb{R}_1[x]\)에 대해, 여기서 \(\mathbb{R}_k[x]\)는 최대 \(k\)차 다항식의 공간이며, 기저 \(\{1,x,x^2\}\)와 \(\{1,x\}\)에 대한 \(D\)의 행렬을 찾으십시오.
\end{enumerate}

\subsection{5.3 커널과 이미지}\label{53-kernel-and-image}

선형 변환을 깊이 이해하려면, 그것이 무엇을 없애고 무엇을 생성하는지 검토해야 합니다. 이러한 아이디어는 커널과 이미지에 의해 포착되며, 이는 모든 선형 맵과 관련된 두 가지 기본 부분 공간입니다.

\subsubsection{커널}\label{the-kernel}

선형 변환 \(T: V \to W\)의 커널(또는 영공간)은 \(W\)의 영 벡터로 매핑되는 \(V\)의 모든 벡터 집합입니다:

\[\ker(T) = \{ \mathbf{v} \in V \mid T(\mathbf{v}) = \mathbf{0} \}.\]

커널은 항상 \(V\)의 부분 공간입니다. 변환의 퇴화, 즉 아무것도 아닌 것으로 붕괴되는 방향을 측정합니다.

예제 5.3.1.\\
\(T:\mathbb{R}^3 \to \mathbb{R}^2\)를 다음과 같이 정의합시다.

\[T(x,y,z) = (x+y, y+z).\]

행렬 형태로,

\[[T] = \begin{bmatrix}
1 & 1 & 0 \\
0 & 1 & 1
\end{bmatrix}.\]

커널을 찾으려면 다음을 푸십시오.

\begin{bmatrix}
1 & 1 & 0 \\
0 & 1 & 1
\end{bmatrix}
\begin{bmatrix} x \\ y \\ z \end{bmatrix}
= \begin{bmatrix} 0 \\ 0 \end{bmatrix}.

이것은 방정식 \(x + y = 0\), \(y + z = 0\)을 제공합니다. 따라서 \(x = -y, z = -y\)입니다. 커널은

\[\ker(T) = \{ (-t, t, -t) \mid t \in \mathbb{R} \},\]

\(\mathbb{R}^3\)의 선입니다.

\subsubsection{이미지}\label{the-image}

선형 변환 \(T: V \to W\)의 이미지(또는 범위)는 모든 출력의 집합입니다:

\[\text{im}(T) = \{ T(\mathbf{v}) \mid \mathbf{v} \in V \} \subseteq W.\]

동등하게, 그것은 표현 행렬의 열의 스팬입니다. 이미지는 항상 \(W\)의 부분 공간입니다.

예제 5.3.2.\\
위와 동일한 변환에 대해,

\[[T] = \begin{bmatrix}
1 & 1 & 0 \\
0 & 1 & 1
\end{bmatrix},\]

열은 \((1,0)\), \((1,1)\), \((0,1)\)입니다. \((1,1) = (1,0) + (0,1)\)이므로, 이미지는

\[\text{im}(T) = \text{span}\{ (1,0), (0,1) \} = \mathbb{R}^2.\]

\subsubsection{차원 공식 (랭크-널리티 정리)}\label{dimension-formula-rank--nullity-theorem}

유한 차원 \(V\)를 가진 선형 변환 \(T: V \to W\)에 대해,

\[\dim(\ker(T)) + \dim(\text{im}(T)) = \dim(V).\]

이 기본 결과는 잃어버린 방향(커널)을 달성된 방향(이미지)과 연결합니다.

\subsubsection{기하학적 해석}\label{geometric-interpretation-10}

\begin{itemize}
\item
  커널은 변환이 공간을 어떻게 평평하게 만드는지 설명합니다 (예: 3D 객체를 평면에 투영).
\item
  이미지는 변환에 의해 도달된 대상 부분 공간을 설명합니다.
\item
  랭크-널리티 정리는 트레이드오프를 정량화합니다: 더 많은 차원이 붕괴될수록 이미지에 남는 차원은 줄어듭니다.
\end{itemize}

\subsubsection{이것이 중요한 이유}\label{why-this-matters-18}

커널과 이미지는 선형 맵의 본질을 포착합니다. 변환을 분류하고, 시스템이 고유한 해 또는 무한한 해를 가질 때를 설명하며, 랭크-널리티 정리, 대각화, 스펙트럼 이론과 같은 중요한 결과의 백본을 형성합니다.

\subsubsection{연습문제 5.3}\label{exercises-53}

\begin{enumerate}
\def\labelenumi{\arabic{enumi}.}
\item
  \(T:\mathbb{R}^2 \to \mathbb{R}^2\), \(T(x,y) = (x-y, x+y)\)의 커널과 이미지를 찾으십시오.
\item
  다음을 고려하십시오.
\end{enumerate}

\[A = \begin{bmatrix} 1 & 2 & 3 \\ 0 & 1 & 4 \end{bmatrix}\]

\(\ker(A)\)와 \(\text{im}(A)\)에 대한 기저를 찾으십시오.

\begin{enumerate}
\def\labelenumi{\arabic{enumi}.}
\item
  투영 맵 \(P(x,y,z) = (x,y,0)\)에 대해 커널과 이미지를 설명하십시오.
\item
  \(\ker(T)\)와 \(\text{im}(T)\)가 항상 부분 공간임을 증명하십시오.
\item
  예제 5.3.1의 변환에 대해 랭크-널리티 정리를 확인하십시오.
\end{enumerate}

\subsection{5.4 기저 변경}\label{54-change-of-basis}

선형 변환은 우리가 사용하는 좌표계에 따라 매우 다르게 보일 수 있습니다. 벡터와 변환을 새 기저에 대해 다시 쓰는 과정을 기저 변경이라고 합니다. 이 개념은 대각화, 직교화 및 많은 계산 기법의 핵심에 있습니다.

\subsubsection{좌표 변경}\label{coordinate-change}

\(V\)가 \(n\)차원 벡터 공간이고, \(\mathcal{B} = \{\mathbf{v}_1, \dots, \mathbf{v}_n\}\)가 기저라고 합시다. 모든 벡터 \(\mathbf{x} \in V\)는 좌표 벡터 \([\mathbf{x}]_{\mathcal{B}} \in \mathbb{R}^n\)를 가집니다.

만약 \(P\)가 \(\mathcal{B}\)에서 표준 기저로의 기저 변경 행렬이면,

\[\mathbf{x} = P [\mathbf{x}]_{\mathcal{B}}.\]

동등하게,

\[[\mathbf{x}]_{\mathcal{B}} = P^{-1} \mathbf{x}.\]

여기서 \(P\)는 \(\mathcal{B}\)의 기저 벡터를 열로 가집니다:

\[P = \begin{bmatrix}
\mathbf{v}_1 & \mathbf{v}_2 & \cdots & \mathbf{v}_n
\end{bmatrix}.\]

\subsubsection{행렬의 변환}\label{transformation-of-matrices}

\(T: V \to V\)를 선형 변환이라고 합시다. 표준 기저에서의 행렬이 \(A\)라고 가정합시다. 기저 \(\mathcal{B}\)에서 표현 행렬은 다음과 같습니다.

\[[T]_{\mathcal{B}} = P^{-1} A P.\]

따라서 기저를 변경하는 것은 행렬의 유사성 변환에 해당합니다.

\subsubsection{예제}\label{example-4}

예제 5.4.1.\\
\(T:\mathbb{R}^2 \to \mathbb{R}^2\)를 다음과 같이 정의합시다.

\[T(x,y) = (3x + y, x + y).\]

표준 기저에서 그 행렬은

\[A = \begin{bmatrix}
3 & 1 \\
1 & 1
\end{bmatrix}.\]

이제 기저 \(\mathcal{B} = \{ (1,1), (1,-1) \}\)를 고려합시다. 기저 변경 행렬은

\[P = \begin{bmatrix}
1 & 1 \\
1 & -1
\end{bmatrix}.\]

그러면

\[[T]_{\mathcal{B}} = P^{-1} A P.\]

계산하면

\[[T]_{\mathcal{B}} =
\begin{bmatrix}
4 & 0 \\
0 & 0
\end{bmatrix}.\]

이 새 기저에서 변환은 대각선입니다: 한 방향은 4로 스케일링되고, 다른 방향은 0으로 붕괴됩니다.

\subsubsection{기하학적 해석}\label{geometric-interpretation-11}

기저 변경은 좌표 격자를 회전하거나 기울이는 것과 같습니다. 기본 변환은 변경되지 않지만, 숫자로 된 설명은 기저에 따라 더 간단하거나 더 복잡해집니다. 변환을 단순화하는 기저(종종 대각 기저)를 찾는 것은 선형대수학의 핵심 주제입니다.

\subsubsection{이것이 중요한 이유}\label{why-this-matters-19}

기저 변경은 유사성의 추상적 개념을 실제 계산에 연결합니다. 행렬을 대각화하고, 고유값을 계산하고, 복잡한 변환을 단순화할 수 있게 해주는 도구입니다. 응용 분야에서, 그것은 기하학, 물리학 또는 기계 학습에서 더 자연스러운 좌표계를 선택하는 것에 해당합니다.

\subsubsection{연습문제 5.4}\label{exercises-54}

\begin{enumerate}
\def\labelenumi{\arabic{enumi}.}
\item
  다음을 고려하십시오.
\end{enumerate}

\[A = \begin{bmatrix} 2 & 1 \\ 0 & 2 \end{bmatrix}\]

기저 \(\{(1,0),(1,1)\}\)에서의 표현을 계산하십시오.

\begin{enumerate}
\def\labelenumi{\arabic{enumi}.}
\item
  \(\mathbb{R}^2\)의 표준 기저에서 \(\{(2,1),(1,1)\}\)로의 기저 변경 행렬을 찾으십시오.
\item
  유사 행렬(\(P^{-1}AP\)에 의해 관련됨)이 다른 기저 하에서 동일한 선형 변환을 나타냄을 증명하십시오.
\item
  행렬을 대각화하십시오.
\end{enumerate}

\[A = \begin{bmatrix} 1 & 0 \\ 0 & -1 \end{bmatrix}\]

기저 \(\{(1,1),(1,-1)\}\)에서.

\begin{enumerate}
\def\labelenumi{\arabic{enumi}.}
\item
  \(\mathbb{R}^3\)에서, \(\mathcal{B} = \{(1,0,0),(1,1,0),(1,1,1)\}\)라고 합시다. 기저 변경 행렬 \(P\)를 구성하고 \(P^{-1}\)를 계산하십시오.
\end{enumerate}

\section{6장. 행렬식}\label{chapter-6-determinants}

\subsection{6.1 동기 및 기하학적 의미}\label{61-motivation-and-geometric-meaning}

행렬식은 정사각 행렬과 관련된 수치 값입니다. 처음에는 복잡한 공식으로 보일 수 있지만, 그 중요성은 그들이 측정하는 것에서 나옵니다: 행렬식은 선형 변환의 스케일링, 방향 및 가역성을 인코딩합니다. 그들은 대수와 기하학을 연결합니다.

\subsubsection{\$2 \textbackslash times 2\$ 행렬의 행렬식}\label{determinants-of-ux242-times-2ux24-matrices}

\$2 \textbackslash times 2\$ 행렬

\[A = \begin{bmatrix} a & b \\ c & d \end{bmatrix},\]

에 대해 행렬식은 다음과 같이 정의됩니다.

\[\det(A) = ad - bc.\]

기하학적 의미: 만약 \(A\)가 평면의 선형 변환을 나타낸다면, \(|\det(A)|\)는 면적 스케일링 팩터입니다. 예를 들어, \(\det(A) = 2\)이면, 모양의 면적이 두 배가 됩니다. \(\det(A) = 0\)이면, 변환은 평면을 선으로 붕괴시킵니다: 모든 면적이 사라집니다.

\subsubsection{\$3 \textbackslash times 3\$ 행렬의 행렬식}\label{determinants-of-ux243-times-3ux24-matrices}

에 대해

\[A = \begin{bmatrix}
a & b & c \\
d & e & f \\
g & h & i
\end{bmatrix},\]

행렬식은 다음과 같이 계산될 수 있습니다.

\[\det(A) = a(ei - fh) - b(di - fg) + c(dh - eg).\]

기하학적 의미: \(\mathbb{R}^3\)에서, \(|\det(A)|\)는 부피 스케일링 팩터입니다. \(\det(A) < 0\)이면, 방향이 반전됩니다 (손잡이 뒤집기), 예를 들어 오른손 좌표계를 왼손 좌표계로 바꾸는 것과 같습니다.

\subsubsection{일반적인 경우}\label{general-case}

\(A \in \mathbb{R}^{n \times n}\)에 대해, 행렬식은 \(A\)에 의해 주어진 선형 변환이 n차원 부피를 어떻게 스케일링하는지를 측정하는 스칼라입니다.

\begin{itemize}
\item
  \(\det(A) = 0\)이면: 변환은 공간을 더 낮은 차원으로 찌그러뜨리므로, \(A\)는 가역이 아닙니다.
\item
  \(\det(A) > 0\)이면: 부피는 \(\det(A)\)에 의해 스케일링되고, 방향은 보존됩니다.
\item
  \(\det(A) < 0\)이면: 부피는 \(|\det(A)|\)에 의해 스케일링되고, 방향은 반전됩니다.
\end{itemize}

\subsubsection{시각적 예제}\label{visual-examples}

\begin{enumerate}
\def\labelenumi{\arabic{enumi}.}
\item
  \(\mathbb{R}^2\)에서의 전단 변환:\\
  \(A = \begin{bmatrix} 1 & 1 \\ 0 & 1 \end{bmatrix}\).\\
  그러면 \(\det(A) = 1\)입니다. 변환은 단위 정사각형을 평행사변형으로 기울이지만 면적은 보존합니다.
\item
  \(\mathbb{R}^2\)에서의 투영:\\
  \(A = \begin{bmatrix} 1 & 0 \\ 0 & 0 \end{bmatrix}\).\\
  그러면 \(\det(A) = 0\)입니다. 단위 정사각형은 선분으로 붕괴됩니다: 면적이 사라집니다.
\item
  \(\mathbb{R}^2\)에서의 회전:\\
  \(R_\theta = \begin{bmatrix} \cos\theta & -\sin\theta \\ \sin\theta & \cos\theta \end{bmatrix}\).\\
  그러면 \(\det(R_\theta) = 1\)입니다. 회전은 면적과 방향을 보존합니다.
\end{enumerate}

\subsubsection{이것이 중요한 이유}\label{why-this-matters-20}

행렬식은 단지 공식이 아닙니다 - 변환의 척도입니다. 행렬이 가역인지, 공간을 어떻게 왜곡하는지, 방향을 뒤집는지 알려줍니다. 이 기하학적 통찰력은 분석, 기하학, 응용 수학에서 행렬식을 필수 불가결하게 만듭니다.

\subsubsection{연습문제 6.1}\label{exercises-61}

\begin{enumerate}
\def\labelenumi{\arabic{enumi}.}
\item
  다음의 행렬식을 계산하십시오.
\end{enumerate}

\begin{bmatrix} 2 & 3 \\ 1 & 4 \end{bmatrix}

어떤 면적 스케일링 팩터를 나타냅니까?

\begin{enumerate}
\def\labelenumi{\arabic{enumi}.}
\item
  전단 행렬의 행렬식을 찾으십시오.
\end{enumerate}

\begin{bmatrix} 1 & 2 \\ 0 & 1 \end{bmatrix}

단위 정사각형의 면적은 어떻게 됩니까?

\begin{enumerate}
\def\labelenumi{\arabic{enumi}.}
\item
  \$3 \textbackslash times 3\$ 행렬 \(\begin{bmatrix} 1 & 0 & 0 \\ 0 & 2 & 0 \\ 0 & 0 & 3 \end{bmatrix}\)에 대해 행렬식을 계산하십시오. \(\mathbb{R}^3\)에서 부피를 어떻게 스케일링합니까?
\item
  \(\mathbb{R}^2\)의 모든 회전 행렬의 행렬식이 \$1\$임을 보이십시오.
\item
  행렬식이 \(-1\)인 \$2 \textbackslash times 2\$ 행렬의 예를 드십시오. 어떤 기하학적 작용을 나타냅니까?
\end{enumerate}

\subsection{6.2 행렬식의 속성}\label{62-properties-of-determinants}

기하학적 의미 외에도, 행렬식은 선형대수학에서 강력한 도구로 만드는 대수적 규칙 모음을 만족합니다. 이러한 속성을 통해 효율적으로 계산하고, 가역성을 테스트하고, 행렬 연산 하에서 행렬식이 어떻게 작동하는지 이해할 수 있습니다.

\subsubsection{기본 속성}\label{basic-properties}

\(A, B \in \mathbb{R}^{n \times n}\)이고, \(c \in \mathbb{R}\)라고 합시다. 그러면:

\begin{enumerate}
\def\labelenumi{\arabic{enumi}.}
\item
  항등원:
\end{enumerate}

\[\det(I_n) = 1.\]

\begin{enumerate}
\def\labelenumi{\arabic{enumi}.}
\item
  삼각 행렬:\\
  \(A\)가 상삼각 또는 하삼각 행렬이면,
\end{enumerate}

\[\det(A) = a_{11} a_{22} \cdots a_{nn}.\]

\begin{enumerate}
\def\labelenumi{\arabic{enumi}.}
\item
  행/열 교환:\\
  두 행(또는 열)을 교환하면 행렬식이 \(-1\)배가 됩니다.
\item
  행/열 스케일링:\\
  한 행(또는 열)에 스칼라 \(c\)를 곱하면 행렬식이 \(c\)배가 됩니다.
\item
  행/열 덧셈:\\
  한 행에 다른 행의 배수를 더해도 행렬식은 변하지 않습니다.
\item
  전치:
\end{enumerate}

\[\det(A^T) = \det(A).\]

\begin{enumerate}
\def\labelenumi{\arabic{enumi}.}
\item
  곱셈성:
\end{enumerate}

\[\det(AB) = \det(A)\det(B).\]

\begin{enumerate}
\def\labelenumi{\arabic{enumi}.}
\item
  가역성:\\
  \(A\)는 \(\det(A) \neq 0\)일 때만 가역입니다.
\end{enumerate}

\subsubsection{계산 예제}\label{example-computations}

예제 6.2.1.\\
에 대해

\[A = \begin{bmatrix}
2 & 0 & 0 \\
1 & 3 & 0 \\
-1 & 4 & 5
\end{bmatrix},\]

\(A\)는 하삼각 행렬이므로,

\[\det(A) = 2 \cdot 3 \cdot 5 = 30.\]

예제 6.2.2.\\
다음과 같다고 합시다.

\[B = \begin{bmatrix} 1 & 2 \\ 3 & 4 \end{bmatrix}, \quad
C = \begin{bmatrix} 0 & 1 \\ 1 & 0 \end{bmatrix}.\]

그러면

\[\det(B) = 1\cdot 4 - 2\cdot 3 = -2, \quad \det(C) = -1.\]

\(CB\)는 \(B\)의 행을 교환하여 얻으므로,

\[\det(CB) = -\det(B) = 2.\]

이것은 곱셈성 규칙과 일치합니다: \(\det(CB) = \det(C)\det(B) = (-1)(-2) = 2.\)

\subsubsection{기하학적 통찰}\label{geometric-insights}

\begin{itemize}
\item
  행 교환: 공간의 방향을 뒤집습니다.
\item
  행 스케일링: 한 방향으로 공간을 늘립니다.
\item
  행 교체: 부피를 변경하지 않고 초평면을 밉니다.
\item
  곱셈성: 두 변환을 수행하면 스케일링 팩터가 곱해집니다.
\end{itemize}

이러한 속성은 행렬식을 계산적으로 다루기 쉽고 기하학적으로 해석 가능하게 만듭니다.

\subsubsection{이것이 중요한 이유}\label{why-this-matters-21}

행렬식 속성은 계산과 기하학 및 이론을 연결합니다. 가우스 소거법이 왜 작동하는지, 가역성이 0이 아닌 행렬식과 왜 동등한지, 그리고 부피 계산, 고유값 이론, 미분 방정식과 같은 분야에서 행렬식이 왜 자연스럽게 나타나는지를 설명합니다.

\subsubsection{연습문제 6.2}\label{exercises-62}

\begin{enumerate}
\def\labelenumi{\arabic{enumi}.}
\item
  다음의 행렬식을 계산하십시오.
\end{enumerate}

\[A = \begin{bmatrix} 1 & 2 & 3 \\ 0 & 1 & 4 \\ 0 & 0 & 2 \end{bmatrix}.\]

\begin{enumerate}
\def\labelenumi{\arabic{enumi}.}
\item
  정사각 행렬의 두 행이 동일하면 행렬식이 0임을 보이십시오.
\item
  \(\det(A^T) = \det(A)\)를 다음에 대해 확인하십시오.
\end{enumerate}

\[A = \begin{bmatrix} 2 & -1 \\ 3 & 4 \end{bmatrix}.\]

\begin{enumerate}
\def\labelenumi{\arabic{enumi}.}
\item
  \(A\)가 가역이면, 다음을 증명하십시오.
\end{enumerate}

\[\det(A^{-1}) = \frac{1}{\det(A)}.\]

\begin{enumerate}
\def\labelenumi{\arabic{enumi}.}
\item
  \(A\)가 \(\det(A) = 5\)인 \$3\textbackslash times 3\$ 행렬이라고 가정합니다. \(\det(2A)\)는 무엇입니까?
\end{enumerate}

\subsection{6.3 여인수 전개}\label{63-cofactor-expansion}

작은 행렬의 행렬식은 공식에서 직접 계산할 수 있지만, 큰 행렬은 체계적인 방법이 필요합니다. 여인수 전개(라플라스 전개라고도 함)는 행렬식을 더 작은 행렬식으로 분해하여 재귀적으로 계산하는 방법을 제공합니다.

\subsubsection{소행렬과 여인수}\label{minors-and-cofactors}

\(n \times n\) 행렬 \(A = [a_{ij}]\)에 대해:

\begin{itemize}
\item
  소행렬 \(M_{ij}\)는 \(A\)의 \(i\)번째 행과 \(j\)번째 열을 삭제하여 얻은 \((n-1) \times (n-1)\) 행렬의 행렬식입니다.
\item
  여인수 \(C_{ij}\)는 다음과 같이 정의됩니다.
\end{itemize}

\[C_{ij} = (-1)^{i+j} M_{ij}.\]

부호 인자 \((-1)^{i+j}\)는 체커보드 패턴으로 번갈아 나타납니다:

\begin{bmatrix}
+ & - & + & - & \cdots \\
- & + & - & + & \cdots \\
+ & - & + & - & \cdots \\
\vdots & \vdots & \vdots & \vdots & \ddots
\end{bmatrix}.

\subsubsection{여인수 전개 공식}\label{cofactor-expansion-formula}

\(A\)의 행렬식은 임의의 행 또는 열을 따라 전개하여 계산할 수 있습니다:

\[\det(A) = \sum_{j=1}^n a_{ij} C_{ij} \quad \text{(행 \(i\)를 따라 전개)},\]

\[\det(A) = \sum_{i=1}^n a_{ij} C_{ij} \quad \text{(열 \(j\)를 따라 전개)}.\]

\subsubsection{예제}\label{example-5}

예제 6.3.1.\\
계산

\[A = \begin{bmatrix}
1 & 2 & 3 \\
0 & 4 & 5 \\
1 & 0 & 6
\end{bmatrix}.\]

첫 번째 행을 따라 전개:

\[\det(A) = 1 \cdot C_{11} + 2 \cdot C_{12} + 3 \cdot C_{13}.\]

\begin{itemize}
\item
  \(C_{11}\)에 대해:
\end{itemize}

\[M_{11} = \det \begin{bmatrix} 4 & 5 \\ 0 & 6 \end{bmatrix} = 24\]

그래서 \(C_{11} = (+1)(24) = 24\)입니다.

\begin{itemize}
\item
  \(C_{12}\)에 대해:
\end{itemize}

\[M_{12} = \det \begin{bmatrix} 0 & 5 \\ 1 & 6 \end{bmatrix} = 0 - 5 = -5\]

그래서 \(C_{12} = (-1)(-5) = 5\)입니다.

\begin{itemize}
\item
  \(C_{13}\)에 대해:
\end{itemize}

\[M_{13} = \det \begin{bmatrix} 0 & 4 \\ 1 & 0 \end{bmatrix} = 0 - 4 = -4\]

그래서 \(C_{13} = (+1)(-4) = -4\)입니다.

따라서,

\[\det(A) = 1(24) + 2(5) + 3(-4) = 24 + 10 - 12 = 22.\]

\subsubsection{여인수 전개의 속성}\label{properties-of-cofactor-expansion}

\begin{enumerate}
\def\labelenumi{\arabic{enumi}.}
\item
  임의의 행 또는 열을 따라 전개하면 동일한 결과가 나옵니다.
\item
  여인수 전개는 행렬식의 재귀적 정의를 제공합니다: 크기 \(n\)의 행렬식은 크기 \(n-1\)의 행렬식으로 표현됩니다.
\item
  여인수는 역행렬에 대한 공식을 제공하는 수반 행렬을 구성하는 데 기본적입니다:
\end{enumerate}

\[A^{-1} = \frac{1}{\det(A)} \, \text{adj}(A), \quad \text{where adj}(A) = [C_{ji}].\]

\subsubsection{기하학적 해석}\label{geometric-interpretation-12}

여인수 전개는 행렬식을 한 번에 한 행 또는 열을 고정하여 정의된 하위 부피의 기여로 분해합니다. 각 여인수는 해당 행/열이 전체 부피 스케일링에 어떻게 영향을 미치는지 측정합니다.

\subsubsection{이것이 중요한 이유}\label{why-this-matters-22}

여인수 전개는 작은 행렬 공식을 일반화하고 행렬식의 개념적 정의를 제공합니다. 큰 행렬의 행렬식을 계산하는 가장 효율적인 방법은 아니지만, 이론, 증명, 수반 행렬, 크라메르 법칙, 고전 기하학과의 연결에 필수적입니다.

\subsubsection{연습문제 6.3}\label{exercises-63}

\begin{enumerate}
\def\labelenumi{\arabic{enumi}.}
\item
  다음의 행렬식을 계산하십시오.
\end{enumerate}

\begin{bmatrix}
2 & 0 & 1 \\
3 & -1 & 4 \\
1 & 2 & 0
\end{bmatrix}

첫 번째 열을 따라 여인수 전개로.

\begin{enumerate}
\def\labelenumi{\arabic{enumi}.}
\item
  예제 6.3.1의 두 번째 행을 따라 전개하면 동일한 행렬식이 나오는지 확인하십시오.
\item
  임의의 행을 따라 전개하면 동일한 값이 나옴을 증명하십시오.
\item
  행렬의 한 행이 0이면 행렬식이 0임을 보이십시오.
\item
  여인수 전개를 사용하여 \(\det(A) = \det(A^T)\)임을 증명하십시오.
\end{enumerate}

\subsection{6.4 응용 (부피, 가역성 테스트)}\label{64-applications-volume-invertibility-test}

행렬식은 단지 대수적 호기심이 아닙니다. 구체적인 기하학적 및 계산적 용도가 있습니다. 가장 중요한 두 가지 응용 프로그램은 부피 측정과 행렬의 가역성 테스트입니다.

\subsubsection{부피 스케일러로서의 행렬식}\label{determinants-as-volume-scalers}

벡터 \(\mathbf{v}_1, \mathbf{v}_2, \dots, \mathbf{v}_n \in \mathbb{R}^n\)가 주어지면, 행렬의 열로 배열합니다:

\[A = \begin{bmatrix}
| & | & & | \\
\mathbf{v}_1 & \mathbf{v}_2 & \cdots & \mathbf{v}_n \\
| & | & & |
\end{bmatrix}.\]

그러면 \(|\det(A)|\)는 이러한 벡터에 의해 생성된 평행육면체의 부피와 같습니다.

\begin{itemize}
\item
  \(\mathbb{R}^2\)에서, \(|\det(A)|\)는 \(\mathbf{v}_1, \mathbf{v}_2\)에 의해 생성된 평행사변형의 면적을 제공합니다.
\item
  \(\mathbb{R}^3\)에서, \(|\det(A)|\)는 \(\mathbf{v}_1, \mathbf{v}_2, \mathbf{v}_3\)에 의해 생성된 평행육면체의 부피를 제공합니다.
\item
  고차원에서는 \(n\)차원 부피(초부피)로 일반화됩니다.
\end{itemize}

예제 6.4.1.\\
다음과 같다고 합시다.

\[\mathbf{v}_1 = (1,0,0), \quad \mathbf{v}_2 = (1,1,0), \quad \mathbf{v}_3 = (1,1,1).\]

그러면

\[A = \begin{bmatrix}
1 & 1 & 1 \\
0 & 1 & 1 \\
0 & 0 & 1
\end{bmatrix}, \quad \det(A) = 1.\]

그래서 평행육면체는 부피가 \$1\$입니다. 비록 벡터가 직교하지 않더라도 말입니다.

\subsubsection{가역성 테스트}\label{invertibility-test}

정사각 행렬 \(A\)는 \(\det(A) \neq 0\)일 때만 가역입니다.

\begin{itemize}
\item
  \(\det(A) = 0\)이면: 변환은 공간을 더 낮은 차원으로 붕괴시킵니다 (면적/부피가 0). 역행렬이 존재하지 않습니다.
\item
  \(\det(A) \neq 0\)이면: 변환은 부피를 \(|\det(A)|\)만큼 스케일링하며, 가역적입니다.
\end{itemize}

예제 6.4.2.\\
행렬

\[B = \begin{bmatrix} 2 & 4 \\ 1 & 2 \end{bmatrix}\]

는 행렬식 \(\det(B) = 2 \cdot 2 - 4 \cdot 1 = 0\)을 가집니다.\\
따라서, \(B\)는 가역이 아닙니다. 기하학적으로, 두 열 벡터는 동일선상에 있으며, \(\mathbb{R}^2\)에서 단지 선만을 생성합니다.

\subsubsection{크라메르의 법칙}\label{cramers-rule}

행렬식은 또한 행렬이 가역일 때 선형 방정식 시스템을 푸는 명시적인 공식을 제공합니다. \(A\mathbf{x} = \mathbf{b}\)와 \(A \in \mathbb{R}^{n \times n}\)에 대해:

\[x_i = \frac{\det(A_i)}{\det(A)},\]

여기서 \(A_i\)는 \(A\)의 \(i\)번째 열을 \(\mathbf{b}\)로 교체하여 얻습니다.\\
계산적으로 비효율적이지만, 크라메르의 법칙은 해와 고유성에서 행렬식의 역할을 강조합니다.

\subsubsection{방향}\label{orientation}

\(\det(A)\)의 부호는 변환이 방향을 보존하는지 또는 반전하는지를 나타냅니다. 예를 들어, 평면에서의 반사는 행렬식이 \(-1\)이며, 손잡이를 뒤집습니다.

\subsubsection{이것이 중요한 이유}\label{why-this-matters-23}

행렬식은 핵심 정보를 압축합니다: 스케일링을 측정하고, 가역성을 테스트하고, 방향을 추적합니다. 이러한 통찰력은 기하학(면적 및 부피), 분석(미적분학의 야코비 행렬식), 계산(시스템 해결 및 특이점 확인)에서 필수 불가결합니다.

\subsubsection{연습문제 6.4}\label{exercises-64}

\begin{enumerate}
\def\labelenumi{\arabic{enumi}.}
\item
  \((2,1)\)과 \((1,3)\)에 의해 생성된 평행사변형의 면적을 계산하십시오.
\item
  \((1,0,0), (1,1,0), (1,1,1)\)에 의해 생성된 평행육면체의 부피를 찾으십시오.
\item
  행렬
\end{enumerate}

\begin{bmatrix} 1 & 2 \\ 3 & 6 \end{bmatrix}

이 가역인지 확인하십시오. 행렬식을 사용하여 정당화하십시오.

\begin{enumerate}
\def\labelenumi{\arabic{enumi}.}
\item
  크라메르의 법칙을 사용하여 다음을 푸십시오.
\end{enumerate}

\begin{cases}
x + y = 3, \\
2x - y = 0.
\end{cases}

\begin{enumerate}
\def\labelenumi{\arabic{enumi}.}
\item
  행렬식이 0이면 역행렬이 존재하지 않는 이유를 기하학적으로 설명하십시오.
\end{enumerate}

\section{7장. 내적 공간}\label{chapter-7-inner-product-spaces}

\subsection{7.1 내적과 노름}\label{71-inner-products-and-norms}

\(\mathbb{R}^2\)와 \(\mathbb{R}^3\)를 넘어 길이, 거리, 각도의 기하학적 아이디어를 확장하기 위해 내적을 도입합니다. 내적은 벡터 간의 유사성을 측정하는 방법을 제공하며, 이로부터 파생된 노름은 길이를 측정합니다. 이러한 개념은 벡터 공간 내부의 기하학의 기초입니다.

\subsubsection{내적}\label{inner-product}

실수 벡터 공간 \(V\)에서의 내적은 함수

\[\langle \cdot, \cdot \rangle : V \times V \to \mathbb{R}\]

로, 각 벡터 쌍 \((\mathbf{u}, \mathbf{v})\)에 다음 속성을 만족하는 실수를 할당합니다:

\begin{enumerate}
\def\labelenumi{\arabic{enumi}.}
\item
  대칭성:\\
  \(\langle \mathbf{u}, \mathbf{v} \rangle = \langle \mathbf{v}, \mathbf{u} \rangle.\)
\item
  첫 번째 인수의 선형성:\\
  \(\langle a\mathbf{u} + b\mathbf{w}, \mathbf{v} \rangle = a \langle \mathbf{u}, \mathbf{v} \rangle + b \langle \mathbf{w}, \mathbf{v} \rangle.\)
\item
  양의 정부호성:\\
  \(\langle \mathbf{v}, \mathbf{v} \rangle \geq 0\), 등호는 \(\mathbf{v} = \mathbf{0}\)일 때만 성립합니다.
\end{enumerate}

\(\mathbb{R}^n\)에서의 표준 내적은 점곱입니다:

\[\langle \mathbf{u}, \mathbf{v} \rangle = u_1 v_1 + u_2 v_2 + \cdots + u_n v_n.\]

\subsubsection{노름}\label{norms}

벡터의 노름은 내적으로 정의된 길이입니다:

\[\|\mathbf{v}\| = \sqrt{\langle \mathbf{v}, \mathbf{v} \rangle}.\]

\(\mathbb{R}^n\)의 점곱에 대해:

\[\|(x_1, x_2, \dots, x_n)\| = \sqrt{x_1^2 + x_2^2 + \cdots + x_n^2}.\]

\subsubsection{벡터 간의 각도}\label{angles-between-vectors-2}

내적은 두 0이 아닌 벡터 \(\mathbf{u}, \mathbf{v}\) 사이의 각도 \(\theta\)를 다음과 같이 정의할 수 있게 합니다.

\[\cos \theta = \frac{\langle \mathbf{u}, \mathbf{v} \rangle}{\|\mathbf{u}\| \, \|\mathbf{v}\|}.\]

따라서 두 벡터는 \(\langle \mathbf{u}, \mathbf{v} \rangle = 0\)일 때 직교합니다.

\subsubsection{예제}\label{examples-6}

예제 7.1.1.\\
\(\mathbb{R}^2\)에서, \(\mathbf{u} = (1,2)\), \(\mathbf{v} = (3,4)\)에 대해:

\[\langle \mathbf{u}, \mathbf{v} \rangle = 1\cdot 3 + 2\cdot 4 = 11.\]

\[\|\mathbf{u}\| = \sqrt{1^2 + 2^2} = \sqrt{5}, \quad \|\mathbf{v}\| = \sqrt{3^2 + 4^2} = 5.\]

그래서,

\[\cos \theta = \frac{11}{\sqrt{5}\cdot 5}.\]

예제 7.1.2.\\
함수 공간 \(C[0,1]\)에서, 내적

\[\langle f, g \rangle = \int_0^1 f(x) g(x)\, dx\]

은 길이

\[\|f\| = \sqrt{\int_0^1 f(x)^2 dx}.\]

를 정의합니다. 이것은 기하학을 무한 차원 공간으로 일반화합니다.

\subsubsection{기하학적 해석}\label{geometric-interpretation-13}

\begin{itemize}
\item
  내적: 벡터 간의 유사성을 측정합니다.
\item
  노름: 벡터의 길이.
\item
  각도: 두 방향 간의 정렬 측정.
\end{itemize}

이러한 개념은 대수적 연산을 기하학적 직관과 통합합니다.

\subsubsection{이것이 중요한 이유}\label{why-this-matters-24}

내적과 노름은 기하학을 추상 벡터 공간으로 확장할 수 있게 합니다. 직교성, 투영, 푸리에 급수, 최소 제곱 근사 및 물리학과 기계 학습의 많은 응용의 기초를 형성합니다.

\subsubsection{연습문제 7.1}\label{exercises-71}

\begin{enumerate}
\def\labelenumi{\arabic{enumi}.}
\item
  \(\langle (2,-1,3), (1,4,0) \rangle\)를 계산하십시오. 그런 다음 두 벡터 사이의 각도를 찾으십시오.
\item
  \(\|(x,y)\| = \sqrt{x^2+y^2}\)가 노름의 속성을 만족함을 보이십시오.
\item
  \(\mathbb{R}^3\)에서 \((1,1,0)\)과 \((1,-1,0)\)이 직교하는지 확인하십시오.
\item
  \(C[0,1]\)에서 \(f(x)=x\), \(g(x)=1\)에 대해 \(\langle f,g \rangle\)를 계산하십시오.
\item
  코시-슈바르츠 부등식을 증명하십시오:

  \[|\langle \mathbf{u}, \mathbf{v} \rangle| \leq \|\mathbf{u}\| \, \|\mathbf{v}\|.\]
\end{enumerate}

\subsection{7.2 직교 투영}\label{72-orthogonal-projections}

내적의 가장 유용한 응용 중 하나는 직교 투영의 개념입니다. 투영은 벡터를 부분 공간에 있는 다른 벡터로 근사하여 거리의 의미에서 오차를 최소화할 수 있게 합니다. 이 아이디어는 기하학, 통계학, 수치 해석의 기초가 됩니다.

\subsubsection{선으로의 투영}\label{projection-onto-a-line}

\(\mathbf{u} \in \mathbb{R}^n\)가 0이 아닌 벡터라고 합시다. \(\mathbf{u}\)에 의해 생성된 선은

\[L = \{ c\mathbf{u} \mid c \in \mathbb{R} \}.\]

벡터 \(\mathbf{v}\)가 주어지면, \(\mathbf{v}\)의 \(\mathbf{u}\) 위로의 투영은 \(L\)에서 \(\mathbf{v}\)에 가장 가까운 벡터입니다. 기하학적으로, 그것은 선 위의 \(\mathbf{v}\)의 그림자입니다.

공식은

\[\text{proj}_{\mathbf{u}}(\mathbf{v}) = \frac{\langle \mathbf{v}, \mathbf{u} \rangle}{\langle \mathbf{u}, \mathbf{u} \rangle} \, \mathbf{u}.\]

오차 벡터 \(\mathbf{v} - \text{proj}_{\mathbf{u}}(\mathbf{v})\)는 \(\mathbf{u}\)에 직교합니다.

\subsubsection{예제 7.2.1}\label{example-721}

\(\mathbf{u} = (1,2)\), \(\mathbf{v} = (3,1)\)이라고 합시다.

\[\langle \mathbf{v}, \mathbf{u} \rangle = 3\cdot 1 + 1\cdot 2 = 5, \quad
\langle \mathbf{u}, \mathbf{u} \rangle = 1^2 + 2^2 = 5.\]

그래서

\[\text{proj}_{\mathbf{u}}(\mathbf{v}) = \frac{5}{5}(1,2) = (1,2).\]

오차 벡터는 \((3,1) - (1,2) = (2,-1)\)이며, 이는 \((1,2)\)에 직교합니다.

\subsubsection{부분 공간으로의 투영}\label{projection-onto-a-subspace}

\(W \subseteq \mathbb{R}^n\)가 정규직교 기저 \(\{ \mathbf{w}_1, \dots, \mathbf{w}_k \}\)를 가진 부분 공간이라고 가정합시다. 벡터 \(\mathbf{v}\)의 \(W\) 위로의 투영은

\[\text{proj}_{W}(\mathbf{v}) = \langle \mathbf{v}, \mathbf{w}_1 \rangle \mathbf{w}_1 + \cdots + \langle \mathbf{v}, \mathbf{w}_k \rangle \mathbf{w}_k.\]

이것은 \(W\)에서 \(\mathbf{v}\)에 가장 가까운 고유한 벡터입니다. 차이 \(\mathbf{v} - \text{proj}_{W}(\mathbf{v})\)는 모든 \(W\)에 직교합니다.

\subsubsection{최소 제곱 근사}\label{least-squares-approximation}

직교 투영은 최소 제곱 방법을 설명합니다. 과결정 시스템 \(A\mathbf{x} \approx \mathbf{b}\)를 풀기 위해, 우리는 \(A\mathbf{x}\)를 \(A\)의 열 공간 위로의 \(\mathbf{b}\)의 투영으로 만드는 \(\mathbf{x}\)를 찾습니다. 이것은 정규 방정식을 제공합니다.

\[A^T A \mathbf{x} = A^T \mathbf{b}.\]

따라서 최소 제곱은 단지 위장된 투영입니다.

\subsubsection{기하학적 해석}\label{geometric-interpretation-14}

\begin{itemize}
\item
  투영은 주어진 벡터에 대해 부분 공간에서 가장 가까운 점을 찾습니다.
\item
  유클리드 노름의 의미에서 거리(오차)를 최소화합니다.
\item
  직교성은 오차 벡터가 부분 공간에서 직접 멀어지는 방향을 가리키도록 보장합니다.
\end{itemize}

\subsubsection{이것이 중요한 이유}\label{why-this-matters-25}

직교 투영은 순수 및 응용 수학 모두에서 중심적입니다. 부분 공간의 기하학, 푸리에 급수의 이론, 통계학의 회귀, 수치 선형대수학의 근사 방법의 기초가 됩니다. 데이터를 더 간단한 모델로 맞출 때마다 투영이 작동합니다.

\subsubsection{연습문제 7.2}\label{exercises-72}

\begin{enumerate}
\def\labelenumi{\arabic{enumi}.}
\item
  \((2,3)\)의 벡터 \((1,1)\) 위로의 투영을 계산하십시오.
\item
  \(\mathbf{v} - \text{proj}_{\mathbf{u}}(\mathbf{v})\)가 \(\mathbf{u}\)에 직교함을 보이십시오.
\item
  \(W = \text{span}\{(1,0,0), (0,1,0)\} \subseteq \mathbb{R}^3\)라고 합시다. \((1,2,3)\)의 \(W\) 위로의 투영을 찾으십시오.
\item
  최소 제곱 피팅이 \(A\)의 열 공간으로의 투영에 해당하는 이유를 설명하십시오.
\item
  부분 공간 \(W\)로의 투영이 고유함을 증명하십시오: 주어진 \(\mathbf{v}\)에 대해 \(W\)에서 가장 가까운 벡터는 정확히 하나입니다.
\end{enumerate}

\subsection{7.3 그람-슈미트 과정}\label{73-gram--schmidt-process}

그람-슈미트 과정은 임의의 선형 독립적인 벡터 집합을 정규직교 기저로 바꾸는 체계적인 방법입니다. 이것은 정규직교 기저가 계산을 단순화하기 때문에 특히 유용합니다: 내적은 간단한 좌표 비교가 되고, 투영은 깔끔한 형태를 취합니다.

\subsubsection{아이디어}\label{the-idea}

내적 공간에서 선형 독립적인 벡터 집합 \(\{\mathbf{v}_1, \mathbf{v}_2, \dots, \mathbf{v}_n\}\)이 주어지면, 동일한 부분 공간을 생성하는 정규직교 집합 \(\{\mathbf{u}_1, \mathbf{u}_2, \dots, \mathbf{u}_n\}\)을 구성하고 싶습니다.

단계별로 진행합니다:

\begin{enumerate}
\def\labelenumi{\arabic{enumi}.}
\item
  \(\mathbf{v}_1\)로 시작하여, 그것을 정규화하여 \(\mathbf{u}_1\)를 얻습니다.
\item
  \(\mathbf{v}_2\)에서 \(\mathbf{u}_1\) 위로의 투영을 빼서, \(\mathbf{u}_1\)에 직교하는 벡터를 남깁니다. 정규화하여 \(\mathbf{u}_2\)를 얻습니다.
\item
  각 \(\mathbf{v}_k\)에 대해, 이전에 구성된 모든 \(\mathbf{u}_1, \dots, \mathbf{u}_{k-1}\) 위로의 투영을 뺀 다음 정규화합니다.
\end{enumerate}

\subsubsection{알고리즘}\label{the-algorithm}

\(k = 1, 2, \dots, n\)에 대해:

\[\mathbf{w}_k = \mathbf{v}_k - \sum_{j=1}^{k-1} \langle \mathbf{v}_k, \mathbf{u}_j \rangle \mathbf{u}_j,\]

\[\mathbf{u}_k = \frac{\mathbf{w}_k}{\|\mathbf{w}_k\|}.\]

결과 \(\{\mathbf{u}_1, \dots, \mathbf{u}_n\}\)은 원래 벡터의 스팬의 정규직교 기저입니다.

\subsubsection{예제 7.3.1}\label{example-731}

\(\mathbb{R}^3\)에서 \(\mathbf{v}_1 = (1,1,0), \ \mathbf{v}_2 = (1,0,1), \ \mathbf{v}_3 = (0,1,1)\)를 취합니다.

\begin{enumerate}
\def\labelenumi{\arabic{enumi}.}
\item
  \(\mathbf{v}_1\)를 정규화합니다:
\end{enumerate}

\[\mathbf{u}_1 = \frac{1}{\sqrt{2}}(1,1,0).\]

\begin{enumerate}
\def\labelenumi{\arabic{enumi}.}
\item
  \(\mathbf{v}_2\)의 \(\mathbf{u}_1\) 위로의 투영을 뺍니다:
\end{enumerate}

\[\mathbf{w}_2 = \mathbf{v}_2 - \langle \mathbf{v}_2,\mathbf{u}_1 \rangle \mathbf{u}_1.\]

\[\langle \mathbf{v}_2,\mathbf{u}_1 \rangle = \frac{1}{\sqrt{2}}(1\cdot 1 + 0\cdot 1 + 1\cdot 0) = \tfrac{1}{\sqrt{2}}.\]

그래서

\[\mathbf{w}_2 = (1,0,1) - \tfrac{1}{\sqrt{2}}\cdot \tfrac{1}{\sqrt{2}}(1,1,0)
= (1,0,1) - \tfrac{1}{2}(1,1,0)
= \left(\tfrac{1}{2}, -\tfrac{1}{2}, 1\right).\]

정규화:

\[\mathbf{u}_2 = \frac{1}{\sqrt{\tfrac{1}{4}+\tfrac{1}{4}+1}} \left(\tfrac{1}{2}, -\tfrac{1}{2}, 1\right)
= \frac{1}{\sqrt{\tfrac{3}{2}}}\left(\tfrac{1}{2}, -\tfrac{1}{2}, 1\right).\]

\begin{enumerate}
\def\labelenumi{\arabic{enumi}.}
\item
  \(\mathbf{v}_3\)에서 투영을 뺍니다:
\end{enumerate}

\[\mathbf{w}_3 = \mathbf{v}_3 - \langle \mathbf{v}_3,\mathbf{u}_1 \rangle \mathbf{u}_1 - \langle \mathbf{v}_3,\mathbf{u}_2 \rangle \mathbf{u}_2.\]

계산 후, 정규화하여 \(\mathbf{u}_3\)를 얻습니다.

결과는 \(\{\mathbf{v}_1,\mathbf{v}_2,\mathbf{v}_3\}\)의 스팬의 정규직교 기저입니다.

\subsubsection{기하학적 해석}\label{geometric-interpretation-15}

그람-슈미트는 벡터 집합을 곧게 펴는 것과 같습니다: 원래 방향으로 시작하여 각 새 벡터를 이전의 모든 벡터에 수직이 되도록 조정합니다. 그런 다음 단위 길이로 스케일링합니다. 이 과정은 스팬을 보존하면서 직교성을 보장합니다.

\subsubsection{이것이 중요한 이유}\label{why-this-matters-26}

정규직교 기저는 내적, 투영 및 일반적인 계산을 단순화합니다. 좌표계를 다루기 쉽게 만들고 수치 방법, QR 분해, 푸리에 분석, 통계학(직교 다항식, 주성분 분석)에서 중요합니다.

\subsubsection{연습문제 7.3}\label{exercises-73}

\begin{enumerate}
\def\labelenumi{\arabic{enumi}.}
\item
  \(\mathbb{R}^2\)에서 \((1,0), (1,1)\)에 그람-슈미트를 적용하십시오.
\item
  \(\mathbb{R}^3\)에서 \((1,1,1), (1,0,1)\)을 직교화하십시오.
\item
  그람-슈미트의 각 단계가 이전의 모든 벡터에 직교하는 벡터를 생성함을 증명하십시오.
\item
  그람-슈미트가 원래 벡터의 스팬을 보존함을 보이십시오.
\item
  그람-슈미트가 행렬의 QR 분해로 이어지는 방법을 설명하십시오.
\end{enumerate}

\subsection{7.4 정규직교 기저}\label{74-orthonormal-bases}

정규직교 기저는 모든 벡터가 서로 직교하고 단위 길이를 갖는 벡터 공간의 기저입니다. 이러한 기저는 가장 편리한 좌표계입니다: 내적, 투영, 노름을 포함하는 계산이 매우 간단해집니다.

\subsubsection{정의}\label{definition-4}

내적 공간 \(V\)의 벡터 집합 \(\{\mathbf{u}_1, \mathbf{u}_2, \dots, \mathbf{u}_n\}\)은 다음과 같을 때 정규직교 기저라고 불립니다.

\begin{enumerate}
\def\labelenumi{\arabic{enumi}.}
\item
  \(\langle \mathbf{u}_i, \mathbf{u}_j \rangle = 0\) (직교성),
\item
  \(\|\mathbf{u}_i\| = 1\) (정규화),
\item
  집합이 \(V\)를 스팬합니다.
\end{enumerate}

\subsubsection{예제}\label{examples-7}

예제 7.4.1. \(\mathbb{R}^2\)에서, 표준 기저

\[\mathbf{e}_1 = (1,0), \quad \mathbf{e}_2 = (0,1)\]

는 점곱 하에서 정규직교입니다.

예제 7.4.2. \(\mathbb{R}^3\)에서, 표준 기저

\[\mathbf{e}_1 = (1,0,0), \quad \mathbf{e}_2 = (0,1,0), \quad \mathbf{e}_3 = (0,0,1)\]

는 정규직교입니다.

예제 7.4.3. 함수에 대한 푸리에 기저:

\[\{1, \cos x, \sin x, \cos 2x, \sin 2x, \dots\}\]

는 \([-\pi,\pi]\)에서 제곱 적분 가능한 함수의 공간에서 내적

\[\langle f,g \rangle = \int_{-\pi}^{\pi} f(x) g(x)\, dx.\]

을 가진 직교 집합입니다. 정규화 후, 정규직교 기저가 됩니다.

\subsubsection{속성}\label{properties}

\begin{enumerate}
\def\labelenumi{\arabic{enumi}.}
\item
  좌표 단순성: \(\{\mathbf{u}_1,\dots,\mathbf{u}_n\}\)이 \(V\)의 정규직교 기저이면, 임의의 벡터 \(\mathbf{v}\in V\)는 좌표

  \[[\mathbf{v}] = \begin{bmatrix} \langle \mathbf{v}, \mathbf{u}_1 \rangle \\ \vdots \\ \langle \mathbf{v}, \mathbf{u}_n \rangle \end{bmatrix}.\]

  를 가집니다. 즉, 좌표는 단지 내적입니다.
\item
  파세발의 항등식:\\
  임의의 \(\mathbf{v} \in V\)에 대해,

  \[\|\mathbf{v}\|^2 = \sum_{i=1}^n |\langle \mathbf{v}, \mathbf{u}_i \rangle|^2.\]
\item
  투영:\\
  \(\\{\mathbf{u}_1,\dots,\mathbf{u}_k\\}\)의 스팬 위로의 직교 투영은

  \[\text{proj}(\mathbf{v}) = \sum_{i=1}^k \langle \mathbf{v}, \mathbf{u}_i \rangle \mathbf{u}_i.\]
\end{enumerate}

\subsubsection{정규직교 기저 구성}\label{constructing-orthonormal-bases}

\begin{itemize}
\item
  임의의 선형 독립적인 집합으로 시작하여, 그람-슈미트 과정을 적용하여 동일한 부분 공간을 생성하는 정규직교 집합을 얻습니다.
\item
  실제로, 정규직교 기저는 수치 안정성과 계산의 단순성을 위해 종종 선택됩니다.
\end{itemize}

\subsubsection{기하학적 해석}\label{geometric-interpretation-16}

정규직교 기저는 완벽하게 정렬되고 동일하게 스케일링된 좌표계와 같습니다. 거리와 각도는 보정 계수 없이 좌표를 사용하여 직접 계산됩니다. 그들은 선형대수학의 이상적인 자입니다.

\subsubsection{이것이 중요한 이유}\label{why-this-matters-27}

정규직교 기저는 선형대수학의 모든 측면을 단순화합니다: 시스템 해결, 투영 계산, 함수 확장, 대칭 행렬 대각화, 푸리에 급수 작업. 데이터 과학에서 주성분 분석은 최대 분산을 포착하는 정규직교 방향을 생성합니다.

\subsubsection{연습문제 7.4}\label{exercises-74}

\begin{enumerate}
\def\labelenumi{\arabic{enumi}.}
\item
  \((1/\\sqrt{2})(1,1)\)과 \((1/\\sqrt{2})(1,-1)\)이 \(\mathbb{R}^2\)의 정규직교 기저를 형성함을 확인하십시오.
\item
  \((3,4)\)를 정규직교 기저 \(\{(1/\\sqrt{2})(1,1), (1/\\sqrt{2})(1,-1)\}\)로 표현하십시오.
\item
  점곱으로 \(\\mathbb{R}^n\)에 대한 파세발의 항등식을 증명하십시오.
\item
  평면 \(x+y+z=0\)에 대한 \(\\mathbb{R}^3\)의 정규직교 기저를 찾으십시오.
\item
  정규직교 기저가 계산에서 임의의 기저보다 수치적으로 더 안정적인 이유를 설명하십시오.
\end{enumerate}

\section{8장. 고유값과 고유벡터}\label{chapter-8-eigenvalues-and-eigenvectors}

\subsection{8.1 정의와 직관}\label{81-definitions-and-intuition}

고유값과 고유벡터의 개념은 선형 변환의 가장 기본적인 행동을 드러냅니다. 그들은 변환이 회전이나 왜곡 없이 단순한 늘리기나 압축으로 작용하는 특별한 방향을 식별합니다.

\subsubsection{정의}\label{definition-5}

\(T: V \to V\)가 벡터 공간 \(V\)에서의 선형 변환이라고 합시다. 0이 아닌 벡터 \(\mathbf{v} \in V\)가

\[T(\mathbf{v}) = \lambda \mathbf{v}\]

를 만족하면 \(T\)의 고유벡터라고 불립니다. 여기서 스칼라 \(\lambda \in \mathbb{R}\) (또는 \(\mathbb{C}\))입니다. 스칼라 \(\lambda\)는 \(\mathbf{v}\)에 해당하는 고유값입니다.

동등하게, 만약 \(A\)가 \(T\)의 행렬이면, 고유값과 고유벡터는 다음을 만족합니다.

\[A\mathbf{v} = \lambda \mathbf{v}.\]

\subsubsection{기본 예제}\label{basic-examples}

예제 8.1.1.\\
다음과 같다고 합시다.

\[A = \begin{bmatrix} 2 & 0 \\ 0 & 3 \end{bmatrix}.\]

그러면

\[A(1,0)^T = 2(1,0)^T, \quad A(0,1)^T = 3(0,1)^T.\]

그래서 \((1,0)\)은 고유값 \$2\)를 가진 고유벡터이고, \((0,1)\)은 고유값 \$3\)을 가진 고유벡터입니다.

예제 8.1.2.\\
\(\mathbb{R}^2\)에서의 회전 행렬:

\[R_\theta = \begin{bmatrix} \cos\theta & -\sin\theta \\ \sin\theta & \cos\theta \end{bmatrix}.\]

만약 \(\theta \neq 0, \pi\)이면, \(R_\theta\)는 실수 고유값을 가지지 않습니다: 모든 벡터는 회전되며, 스케일링되지 않습니다. 그러나 \(\mathbb{C}\) 상에서는 고유값 \(e^{i\theta}, e^{-i\theta}\)를 가집니다.

\subsubsection{대수적 공식}\label{algebraic-formulation}

고유값은 특성 방정식을 풀어서 발생합니다:

\[\det(A - \lambda I) = 0.\]

\(\lambda\)에 대한 이 다항식은 특성 다항식입니다. 그 근은 고유값입니다.

\subsubsection{기하학적 직관}\label{geometric-intuition}

\begin{itemize}
\item
  고유벡터는 변환 하에서 방향이 변하지 않는 방향입니다. 길이만 스케일링됩니다.
\item
  고유값은 해당 방향을 따른 스케일링 팩터를 알려줍니다.
\item
  행렬이 많은 독립적인 고유벡터를 가지면, 종종 기저를 변경하여 단순화(대각화)될 수 있습니다.
\end{itemize}

\subsubsection{기하학 및 과학에서의 응용}\label{applications-in-geometry-and-science}

\begin{itemize}
\item
  타원의 주축을 따른 늘리기 (이차 형식).
\item
  동적 시스템의 안정적인 방향.
\item
  통계 및 기계 학습의 주성분.
\item
  관측량이 고유값을 가진 연산자에 해당하는 양자 역학.
\end{itemize}

\subsubsection{이것이 중요한 이유}\label{why-this-matters-28}

고유값과 고유벡터는 대수와 기하학 사이의 다리입니다. 선형 변환을 가장 간단한 형태로 이해하기 위한 렌즈를 제공합니다. 미분 방정식, 통계학, 물리학, 컴퓨터 과학 등 선형대수학의 거의 모든 응용은 고유값 분석에 의존합니다.

\subsubsection{연습문제 8.1}\label{exercises-81}

\begin{enumerate}
\def\labelenumi{\arabic{enumi}.}
\item
  다음의 고유값과 고유벡터를 찾으십시오.\\
  \(\begin{bmatrix} 4 & 0 \\ 0 & -1 \end{bmatrix}\).
\item
  고유벡터의 모든 스칼라 배수가 동일한 고유값에 대한 고유벡터임을 보이십시오.
\item
  회전 행렬 \(R_\theta\)가 \(\theta = 0\) 또는 \(\pi\)가 아닌 한 실수 고유값을 가지지 않음을 확인하십시오.
\item
  다음의 특성 다항식을 계산하십시오.\\
  \(\begin{bmatrix} 1 & 2 \\ 2 & 1 \end{bmatrix}\).
\item
  전단 행렬에 대해 고유벡터와 고유값이 기하학적으로 무엇을 나타내는지 설명하십시오.\\
  \(\begin{bmatrix} 1 & 1 \\ 0 & 1 \end{bmatrix}\).
\end{enumerate}

\subsection{8.2 대각화}\label{82-diagonalization}

선형대수학의 중심 목표는 좋은 기저를 선택하여 행렬의 작용을 단순화하는 것입니다. 대각화는 행렬을 독립적인 방향을 따라 단순한 스케일링으로 작용하도록 다시 쓰는 과정입니다. 이것은 거듭제곱, 지수, 미분 방정식 풀기와 같은 계산을 훨씬 쉽게 만듭니다.

\subsubsection{정의}\label{definition-6}

정사각 행렬 \(A \in \mathbb{R}^{n \times n}\)은 가역 행렬 \(P\)가 존재하여

\[P^{-1} A P = D,\]

를 만족할 때 대각화 가능하다고 합니다. 여기서 \(D\)는 대각 행렬입니다.

\(D\)의 대각 항목은 \(A\)의 고유값이고, \(P\)의 열은 해당 고유벡터입니다.

\subsubsection{행렬이 대각화 가능한 경우}\label{when-is-a-matrix-diagonalizable}

\begin{itemize}
\item
  행렬은 \(n\)개의 선형 독립적인 고유벡터를 가질 때 대각화 가능합니다.
\item
  동등하게, 고유 공간의 차원의 합이 \(n\)과 같습니다.
\item
  대칭 행렬(\(\mathbb{R}\) 상에서)은 항상 대각화 가능하며, 정규직교 고유벡터 기저를 가집니다.
\end{itemize}

\subsubsection{예제 8.2.1}\label{example-821}

다음과 같다고 합시다.

\[A = \begin{bmatrix} 4 & 1 \\ 0 & 2 \end{bmatrix}.\]

\begin{enumerate}
\def\labelenumi{\arabic{enumi}.}
\item
  특성 다항식:
\end{enumerate}

\[\det(A - \lambda I) = (4-\lambda)(2-\lambda).\]

따라서 고유값은 \(\lambda_1 = 4\), \(\lambda_2 = 2\)입니다.

\begin{enumerate}
\def\labelenumi{\arabic{enumi}.}
\item
  고유벡터:
\end{enumerate}

\begin{itemize}
\item
  \(\lambda = 4\)에 대해, \((A-4I)\mathbf{v}=0\)을 풉니다:\\
  \(\begin{bmatrix} 0 & 1 \\ 0 & -2 \end{bmatrix}\mathbf{v} = 0\), \(\mathbf{v}_1 = (1,0)\)을 제공합니다.
\item
  \(\lambda = 2\)에 대해: \((A-2I)\mathbf{v}=0\), \(\mathbf{v}_2 = (1,-2)\)를 제공합니다.
\end{itemize}

\begin{enumerate}
\def\labelenumi{\arabic{enumi}.}
\item
  \(P = \begin{bmatrix} 1 & 1 \\ 0 & -2 \end{bmatrix}\)를 구성합니다. 그러면
\end{enumerate}

\[P^{-1} A P = \begin{bmatrix} 4 & 0 \\ 0 & 2 \end{bmatrix}.\]

따라서, \(A\)는 대각화 가능합니다.

\subsubsection{왜 대각화하는가?}\label{why-diagonalize}

\begin{itemize}
\item
  거듭제곱 계산:\\
  만약 \(A = P D P^{-1}\)이면,

  \[A^k = P D^k P^{-1}.\]

  \(D\)가 대각 행렬이므로, \(D^k\)는 계산하기 쉽습니다.
\item
  행렬 지수:\\
  \(e^A = P e^D P^{-1}\), 미분 방정식을 푸는 데 유용합니다.
\item
  기하학 이해:\\
  대각화는 변환이 공간을 독립적으로 늘리거나 압축하는 방향을 드러냅니다.
\end{itemize}

\subsubsection{대각화 불가능한 예제}\label{non-diagonalizable-example}

모든 행렬이 대각화될 수 있는 것은 아닙니다.

\[A = \begin{bmatrix} 1 & 1 \\ 0 & 1 \end{bmatrix}\]

는 고유값 \(\lambda = 1\)을 하나만 가지며, 고유 공간 차원은 1입니다. \(n=2\)이지만 독립적인 고유벡터가 1개뿐이므로, \(A\)는 대각화 가능하지 않습니다.

\subsubsection{기하학적 해석}\label{geometric-interpretation-17}

대각화는 우리가 고유벡터의 기저를 찾았음을 의미합니다. 이 기저에서 행렬은 각 좌표축을 따라 단순한 스케일링으로 작용합니다. 복잡한 움직임을 독립적인 1D 움직임으로 변환합니다.

\subsubsection{이것이 중요한 이유}\label{why-this-matters-29}

대각화는 선형대수학의 초석입니다. 계산을 단순화하고, 구조를 드러내며, 스펙트럼 정리, 조르당 형식, 그리고 물리학, 공학, 데이터 과학의 많은 응용의 출발점입니다.

\subsubsection{연습문제 8.2}\label{exercises-82}

\begin{enumerate}
\def\labelenumi{\arabic{enumi}.}
\item
  대각화하십시오.

  \[A = \begin{bmatrix} 2 & 0 \\ 0 & 3 \end{bmatrix}.\]
\item
  다음이 대각화 가능한지 확인하십시오.

  \[A = \begin{bmatrix} 1 & 1 \\ 0 & 1 \end{bmatrix}\]

  왜 그렇거나 그렇지 않습니까?
\item
  다음에 대해 \(A^5\)를 찾으십시오.

  \[A = \begin{bmatrix} 4 & 1 \\ 0 & 2 \end{bmatrix}\]

  대각화를 사용하여.
\item
  \(n\)개의 서로 다른 고유값을 가진 \(n \times n\) 행렬이 대각화 가능함을 보이십시오.
\item
  실수 대칭 행렬이 항상 대각화 가능한 이유를 설명하십시오.
\end{enumerate}

\subsection{8.3 특성 다항식}\label{83-characteristic-polynomials}

고유값을 찾는 핵심은 행렬의 특성 다항식입니다. 이 다항식은 행렬 \(A - \lambda I\)가 가역이 아닌 \(\lambda\) 값을 인코딩합니다.

\subsubsection{정의}\label{definition-7}

\(n \times n\) 행렬 \(A\)에 대해, 특성 다항식은

\[p_A(\lambda) = \det(A - \lambda I).\]

\(p_A(\lambda)\)의 근은 \(A\)의 고유값입니다.

\subsubsection{예제}\label{examples-8}

예제 8.3.1.\\
다음과 같다고 합시다.

\[A = \begin{bmatrix} 2 & 1 \\ 1 & 2 \end{bmatrix}.\]

그러면

\[p_A(\lambda) = \det\!\begin{bmatrix} 2-\lambda & 1 \\ 1 & 2-\lambda \end{bmatrix}
= (2-\lambda)^2 - 1 = \lambda^2 - 4\lambda + 3.\]

따라서 고유값은 \(\lambda = 1, 3\)입니다.

예제 8.3.2.\\
에 대해

\[A = \begin{bmatrix} 0 & -1 \\ 1 & 0 \end{bmatrix}\]

(90° 회전),

\[p_A(\lambda) = \det\!\begin{bmatrix} -\lambda & -1 \\ 1 & -\lambda \end{bmatrix}
= \lambda^2 + 1.\]

고유값은 \(\lambda = \pm i\)입니다. 순수 회전과 일치하게 실수 고유값은 존재하지 않습니다.

예제 8.3.3.\\
삼각 행렬에 대해

\[A = \begin{bmatrix} 2 & 1 & 0 \\ 0 & 3 & 5 \\ 0 & 0 & 4 \end{bmatrix},\]

행렬식은 단순히 대각 항목에서 \(\lambda\)를 뺀 것의 곱입니다:

\[p_A(\lambda) = (2-\lambda)(3-\lambda)(4-\lambda).\]

따라서 고유값은 \$2, 3, 4\$입니다.

\subsubsection{속성}\label{properties-2}

\begin{enumerate}
\def\labelenumi{\arabic{enumi}.}
\item
  \(n \times n\) 행렬의 특성 다항식은 차수 \(n\)을 가집니다.
\item
  고유값의 합(중복도 포함)은 \(A\)의 대각합과 같습니다:

  \[\text{tr}(A) = \lambda_1 + \cdots + \lambda_n.\]
\item
  고유값의 곱은 \(A\)의 행렬식과 같습니다:

  \[\det(A) = \lambda_1 \cdots \lambda_n.\]
\item
  유사 행렬은 동일한 특성 다항식을 가지므로, 동일한 고유값을 가집니다.
\end{enumerate}

\subsubsection{기하학적 해석}\label{geometric-interpretation-18}

특성 다항식은 \(A - \lambda I\)가 공간을 붕괴시킬 때를 포착합니다: 그 행렬식이 0이 되는 것은 정확히 변환 \(A - \lambda I\)가 특이할 때입니다. 따라서 고유값은 행렬이 가역성을 잃는 임계 스케일링을 표시합니다.

\subsubsection{이것이 중요한 이유}\label{why-this-matters-30}

특성 다항식은 고유값을 추출하는 계산 도구를 제공합니다. 행렬 불변량(대각합 및 행렬식)을 기하학과 연결하고, 동적 시스템에서 대각화, 스펙트럼 정리 및 안정성 분석의 기초를 형성합니다.

\subsubsection{연습문제 8.3}\label{exercises-83}

\begin{enumerate}
\def\labelenumi{\arabic{enumi}.}
\item
  다음의 특성 다항식을 계산하십시오.

  \[A = \begin{bmatrix} 4 & 2 \\ 1 & 3 \end{bmatrix}.\]
\item
  \(\begin{bmatrix} 5 & 0 \\ 0 & -2 \end{bmatrix}\)의 고유값의 합이 대각합과 같고, 그 곱이 행렬식과 같음을 확인하십시오.
\item
  임의의 삼각 행렬에 대해 고유값이 단지 대각 항목임을 보이십시오.
\item
  만약 \(A\)와 \(B\)가 유사 행렬이면, \(p_A(\lambda) = p_B(\lambda)\)임을 증명하십시오.
\item
  다음의 특성 다항식을 계산하십시오.\\
  \(\begin{bmatrix} 1 & 1 & 0 \\ 0 & 1 & 1 \\ 0 & 0 & 1 \end{bmatrix}\).
\end{enumerate}

\subsection{8.4 응용 (미분 방정식, 마르코프 연쇄)}\label{84-applications-differential-equations-markov-chains}

고유값과 고유벡터는 선형대수학 이론의 중심일 뿐만 아니라, 수학 및 응용 과학 전반에 걸쳐 필수적인 도구입니다. 두 가지 고전적인 응용은 미분 방정식 시스템 해결과 마르코프 연쇄 분석입니다.

\subsubsection{선형 미분 방정식}\label{linear-differential-equations}

시스템을 고려하십시오.

\[\frac{d\mathbf{x}}{dt} = A \mathbf{x},\]

여기서 \(A\)는 \(n \times n\) 행렬이고 \(\mathbf{x}(t)\)는 벡터 값 함수입니다.

만약 \(\mathbf{v}\)가 고유값 \(\lambda\)를 가진 \(A\)의 고유벡터이면, 함수

\[\mathbf{x}(t) = e^{\lambda t}\mathbf{v}\]

는 해입니다.

\begin{itemize}
\item
  고유값은 성장 또는 붕괴 속도를 결정합니다:

  \begin{itemize}
  \item
    \(\lambda < 0\)이면, 해는 붕괴합니다 (안정).
  \item
    \(\lambda > 0\)이면, 해는 성장합니다 (불안정).
  \item
    \(\lambda\)가 복소수이면, 진동이 발생합니다.
  \end{itemize}
\end{itemize}

고유벡터 해를 결합하여 일반적인 초기 조건을 해결할 수 있습니다.

예제 8.4.1.\\
다음과 같다고 합시다.

\[A = \begin{bmatrix}
2 & 0 \\
0 & -1 \end{bmatrix}.\]

그러면 고유값은 \$2, -1\)이고 고유벡터는 \((1,0)\), \((0,1)\)입니다. 해는

\[\mathbf{x}(t) = c_1 e^{2t}(1,0) + c_2 e^{-t}(0,1).\]

따라서 한 구성 요소는 지수적으로 성장하고, 다른 구성 요소는 붕괴합니다.

\subsubsection{마르코프 연쇄}\label{markov-chains}

마르코프 연쇄는 확률 행렬 \(P\)로 설명되며, 각 열의 합은 1이고 항목은 음수가 아닙니다. 만약 \(\mathbf{x}_k\)가 \(k\) 단계 후의 확률 분포를 나타내면,

\[\mathbf{x}_{k+1} = P \mathbf{x}_k.\]

반복하면

\[\mathbf{x}_k = P^k \mathbf{x}_0.\]

장기적인 행동을 이해하는 것은 \(P\)의 거듭제곱을 분석하는 것으로 귀결됩니다.

\begin{itemize}
\item
  고유값 \(\lambda = 1\)은 항상 존재합니다. 그 고유벡터는 정상 상태 분포를 제공합니다.
\item
  다른 모든 고유값은 \(|\lambda| \leq 1\)을 만족합니다. 그 영향은 \(k \to \infty\)로 갈수록 붕괴됩니다.
\end{itemize}

예제 8.4.2.\\
고려하십시오

\[P = \begin{bmatrix}
0.9 & 0.5 \\
0.1 & 0.5 \end{bmatrix}.\]

고유값은 \(\lambda_1 = 1\), \(\lambda_2 = 0.4\)입니다. \(\lambda = 1\)에 대한 고유벡터는 \((5,1)\)에 비례합니다. 정규화하면 정상 상태

\[\pi = \left(\tfrac{5}{6}, \tfrac{1}{6}\right).\]

를 제공합니다. 따라서 시작 분포에 관계없이 체인은 \(\pi\)로 수렴합니다.

\subsubsection{기하학적 해석}\label{geometric-interpretation-19}

\begin{itemize}
\item
  미분 방정식에서 고유값은 시간 진화를 결정합니다: 지수적 성장, 붕괴 또는 진동.
\item
  마르코프 연쇄에서 고유값은 확률 과정의 장기 평형을 결정합니다.
\end{itemize}

\subsubsection{이것이 중요한 이유}\label{why-this-matters-31}

고유값 방법은 복잡한 반복 또는 동적 시스템을 다루기 쉬운 문제로 바꿉니다. 물리학, 공학, 금융에서 안정성과 공명을 설명합니다. 컴퓨터 과학과 통계학에서 구글의 페이지랭크부터 현대 기계 학습까지 알고리즘을 구동합니다.

\subsubsection{연습문제 8.4}\label{exercises-84}

\begin{enumerate}
\def\labelenumi{\arabic{enumi}.}
\item
  \(\tfrac{d}{dt}\mathbf{x} = \begin{bmatrix} 3 & 0 \\ 0 & -2 \end{bmatrix}\mathbf{x}\)를 푸십시오.
\item
  만약 \(A\)가 복소수 고유값 \(\alpha \pm i\beta\)를 가지면, \(\tfrac{d}{dt}\mathbf{x} = A\mathbf{x}\)의 해가 주파수 \(\beta\)의 진동을 포함함을 보이십시오.
\item
  다음의 정상 상태 분포를 찾으십시오.

  \[P = \begin{bmatrix} 0.7 & 0.2 \\ 0.3 & 0.8 \end{bmatrix}.\]
\item
  임의의 확률 행렬 \(P\)에 대해, \$1\$이 항상 고유값임을 증명하십시오.
\item
  확률 행렬의 모든 고유값이 \(|\lambda| \leq 1\)을 만족하는 이유를 설명하십시오.
\end{enumerate}

\section{9장. 이차 형식과 스펙트럼 정리}\label{chapter-9-quadratic-forms-and-spectral-theorems}

\subsection{9.1 이차 형식}\label{91-quadratic-forms}

이차 형식은 여러 변수에서 2차인 다항식으로, 행렬을 사용하여 깔끔하게 표현됩니다. 이차 형식은 최적화, 원뿔 곡선의 기하학, 통계학(분산), 물리학(에너지 함수) 등 수학 전반에 걸쳐 나타납니다.

\subsubsection{정의}\label{definition-8}

\(A\)를 \(n \times n\) 대칭 행렬이라고 하고 \(\mathbf{x} \in \mathbb{R}^n\)라고 합시다. \(A\)와 관련된 이차 형식은

\[Q(\mathbf{x}) = \mathbf{x}^T A \mathbf{x}.\]

입니다. 확장하면,

\[Q(\mathbf{x}) = \sum_{i=1}^n \sum_{j=1}^n a_{ij} x_i x_j.\]

\(A\)가 대칭(\(a_{ij} = a_{ji}\))이므로, 교차 항은 자연스럽게 그룹화될 수 있습니다.

\subsubsection{예제}\label{examples-9}

예제 9.1.1.\\
에 대해

\[A = \begin{bmatrix}
2 & 1 \\
1 & 3 \end{bmatrix}, \quad \mathbf{x} = \begin{bmatrix}
x \\
y \end{bmatrix},\]

\[Q(x,y) = \begin{bmatrix} x & y \end{bmatrix}
\begin{bmatrix}
2 & 1 \\
1 & 3 \end{bmatrix}
\begin{bmatrix}
x \\
y \end{bmatrix}
= 2x^2 + 2xy + 3y^2.\]

예제 9.1.2.\\
이차 형식

\[Q(x,y) = x^2 + y^2\]

은 행렬 \(A = I_2\)에 해당합니다. 원점으로부터의 유클리드 거리의 제곱을 측정합니다.

예제 9.1.3.\\
원뿔 곡선 방정식

\[4x^2 + 2xy + 5y^2 = 1\]

은 이차 형식 \(\mathbf{x}^T A \mathbf{x} = 1\)로 설명되며,

\[A = \begin{bmatrix}
4 & 1 \\
1 & 5
\end{bmatrix}.\]

입니다.

\subsubsection{이차 형식의 대각화}\label{diagonalization-of-quadratic-forms}

\(A\)의 고유벡터로 구성된 새 기저를 선택함으로써, 교차 항 없이 이차 형식을 다시 쓸 수 있습니다. 만약 \(A = PDP^{-1}\)이고 \(D\)가 대각 행렬이면,

\[Q(\mathbf{x}) = \mathbf{x}^T A \mathbf{x} = (P^{-1}\mathbf{x})^T D (P^{-1}\mathbf{x}).\]

따라서 이차 형식은 항상 가중 제곱의 합으로 표현될 수 있습니다:

\[Q(\mathbf{y}) = \lambda_1 y_1^2 + \cdots + \lambda_n y_n^2,\]

여기서 \(\lambda_i\)는 \(A\)의 고유값입니다.

\subsubsection{기하학적 해석}\label{geometric-interpretation-20}

이차 형식은 기하학적 모양을 설명합니다:

\begin{itemize}
\item
  2D: 타원, 포물선, 쌍곡선.
\item
  3D: 타원체, 포물면, 쌍곡면.
\item
  고차원: 타원체의 일반화.
\end{itemize}

대각화는 좌표축을 모양의 주축과 정렬합니다.

\subsubsection{이것이 중요한 이유}\label{why-this-matters-32}

이차 형식은 기하학과 대수학을 통합합니다. 최적화(에너지 함수 최소화), 통계학(공분산 행렬 및 분산), 역학(운동 에너지), 수치 해석에서 중심적입니다. 이차 형식을 이해하는 것은 스펙트럼 정리로 직접 이어집니다.

\subsubsection{연습문제 9.1}\label{exercises-91}

\begin{enumerate}
\def\labelenumi{\arabic{enumi}.}
\item
  이차 형식 \(Q(x,y) = 3x^2 + 4xy + y^2\)를 어떤 대칭 행렬 \(A\)에 대해 \(\mathbf{x}^T A \mathbf{x}\)로 쓰십시오.
\item
  \(A = \begin{bmatrix} 1 & 2 \\ 2 & 1 \end{bmatrix}\)에 대해, \(Q(x,y)\)를 명시적으로 계산하십시오.
\item
  이차 형식 \(Q(x,y) = 2x^2 + 2xy + 3y^2\)를 대각화하십시오.
\item
  \(Q(x,y) = x^2 - y^2\)에 의해 주어진 원뿔 곡선을 식별하십시오.
\item
  만약 \(A\)가 대칭이면, \(A\)와 \(A^T\)에 의해 정의된 이차 형식이 동일함을 보이십시오.
\end{enumerate}

\subsection{9.2 양의 정부호 행렬}\label{92-positive-definite-matrices}

이차 형식은 관련 행렬이 양의 정부호일 때 특히 중요합니다. 왜냐하면 이는 에너지, 거리 또는 분산의 양성을 보장하기 때문입니다. 양의 정부호성은 최적화, 수치 해석, 통계학의 초석입니다.

\subsubsection{정의}\label{definition-9}

대칭 행렬 \(A \in \mathbb{R}^{n \times n}\)은 다음과 같이 불립니다:

\begin{itemize}
\item
  양의 정부호, 만약

  \[\mathbf{x}^T A \mathbf{x} > 0 \quad \text{for all nonzero } \mathbf{x} \in \mathbb{R}^n.\]
\item
  양의 준정부호, 만약

  \[\mathbf{x}^T A \mathbf{x} \geq 0 \quad \text{for all } \mathbf{x}.\]
\end{itemize}

마찬가지로, 음의 정부호(항상 < 0) 및 부정부호( < 0 및 > 0 모두 가능) 행렬이 정의됩니다.

\subsubsection{예제}\label{examples-10}

예제 9.2.1.

\[A = \begin{bmatrix}
2 & 0 \\
0 & 3 \end{bmatrix}\]

은 양의 정부호입니다. 왜냐하면

\[Q(x,y) = 2x^2 + 3y^2 > 0\]

모든 \((x,y) \neq (0,0)\)에 대해.

예제 9.2.2.

\[A = \begin{bmatrix}
1 & 2 \\
2 & 1 \end{bmatrix}\]

는 이차 형식

\[Q(x,y) = x^2 + 4xy + y^2.\]

을 가집니다. 이 행렬은 \(Q(1,-1) = -2 < 0\)이므로 양의 정부호가 아닙니다.

\subsubsection{특성}\label{characterizations}

대칭 행렬 \(A\)에 대해:

\begin{enumerate}
\def\labelenumi{\arabic{enumi}.}
\item
  고유값 테스트: \(A\)는 \(A\)의 모든 고유값이 양수일 때만 양의 정부호입니다.
\item
  주 소행렬 테스트 (실베스터의 기준): \(A\)는 모든 선행 주 소행렬(왼쪽 상단 \(k \times k\) 부분 행렬의 행렬식)이 양수일 때만 양의 정부호입니다.
\item
  촐레스키 분해: \(A\)는 다음과 같이 쓸 수 있을 때만 양의 정부호입니다.

  \[A = R^T R,\]

  여기서 \(R\)은 양의 대각 항목을 가진 상삼각 행렬입니다.
\end{enumerate}

\subsubsection{기하학적 해석}\label{geometric-interpretation-21}

\begin{itemize}
\item
  양의 정부호 행렬은 원점에 중심을 둔 타원체를 정의하는 이차 형식에 해당합니다.
\item
  양의 준정부호 행렬은 납작한 타원체(퇴화 가능)를 정의합니다.
\item
  부정부호 행렬은 쌍곡선 또는 안장 모양의 표면을 정의합니다.
\end{itemize}

\subsubsection{응용}\label{applications}

\begin{itemize}
\item
  최적화: 볼록 함수의 헤세 행렬은 양의 준정부호입니다. 강한 볼록성은 양의 정부호 헤세 행렬에 해당합니다.
\item
  통계학: 공분산 행렬은 양의 준정부호입니다.
\item
  수치 방법: 촐레스키 분해는 양의 정부호 행렬을 가진 시스템을 효율적으로 해결하는 데 널리 사용됩니다.
\end{itemize}

\subsubsection{이것이 중요한 이유}\label{why-this-matters-33}

양의 정부호성은 수학 및 계산에서 안정성과 보장을 제공합니다. 에너지 함수가 아래로 유계되고, 최적화 문제가 고유한 해를 가지며, 통계 모델이 의미 있음을 보장합니다.

\subsubsection{연습문제 9.2}\label{exercises-92}

\begin{enumerate}
\def\labelenumi{\arabic{enumi}.}
\item
  실베스터의 기준을 사용하여 다음을 확인하십시오.

  \[A = \begin{bmatrix} 2 & -1 \\ -1 & 2 \end{bmatrix}\]

  이 양의 정부호인지.
\item
  다음이 양의 정부호, 준정부호 또는 부정부호인지 확인하십시오.

  \[A = \begin{bmatrix} 0 & 1 \\ 1 & 0 \end{bmatrix}\]
\item
  다음의 고유값을 찾으십시오.

  \[A = \begin{bmatrix} 4 & 2 \\ 2 & 3 \end{bmatrix},\]

  그리고 그것들을 사용하여 정부호성을 분류하십시오.
\item
  양의 항목을 가진 모든 대각 행렬이 양의 정부호임을 증명하십시오.
\item
  만약 \(A\)가 양의 정부호이면, 임의의 가역 행렬 \(P\)에 대해 \(P^T A P\)도 양의 정부호임을 보이십시오.
\end{enumerate}

\subsection{9.3 스펙트럼 정리}\label{93-spectral-theorem}

스펙트럼 정리는 선형대수학에서 가장 강력한 결과 중 하나입니다. 대칭 행렬은 항상 직교 고유벡터 기저에 의해 대각화될 수 있다고 명시합니다. 이것은 대수(고유값), 기하학(직교 방향), 응용(안정성, 최적화, 통계학)을 연결합니다.

\subsubsection{스펙트럼 정리의 명제}\label{statement-of-the-spectral-theorem}

만약 \(A \in \mathbb{R}^{n \times n}\)가 대칭(\(A^T = A\))이면:

\begin{enumerate}
\def\labelenumi{\arabic{enumi}.}
\item
  \(A\)의 모든 고유값은 실수입니다.
\item
  \(\mathbb{R}^n\)의 \(A\)의 고유벡터로 구성된 정규직교 기저가 존재합니다.
\item
  따라서, \(A\)는 다음과 같이 쓸 수 있습니다.

  \[A = Q \Lambda Q^T,\]

  여기서 \(Q\)는 직교 행렬(\(Q^T Q = I\))이고 \(\Lambda\)는 \(A\)의 고유값을 대각선에 가진 대각 행렬입니다.
\end{enumerate}

\subsubsection{결과}\label{consequences}

\begin{itemize}
\item
  대칭 행렬은 항상 대각화 가능하며, 대각화는 수치적으로 안정적입니다.
\item
  이차 형식 \(\mathbf{x}^T A \mathbf{x}\)는 고유값과 고유벡터로 표현될 수 있으며, 고유 방향과 정렬된 타원체를 보여줍니다.
\item
  모든 고유값이 양수임을 확인하여 양의 정부호성을 확인할 수 있습니다.
\end{itemize}

\subsubsection{예제 9.3.1}\label{example-931}

다음과 같다고 합시다.

\[A = \begin{bmatrix}
2 & 1 \\
1 & 2 \end{bmatrix}.\]

\begin{enumerate}
\def\labelenumi{\arabic{enumi}.}
\item
  특성 다항식:
\end{enumerate}

\[p(\lambda) = (2-\lambda)^2 - 1 = \lambda^2 - 4\lambda + 3.\]

고유값: \(\lambda_1 = 1, \ \lambda_2 = 3\).

\begin{enumerate}
\def\labelenumi{\arabic{enumi}.}
\item
  고유벡터:
\end{enumerate}

\begin{itemize}
\item
  \(\lambda=1\)에 대해: \((A-I)\mathbf{v} = 0\)을 풀면, \((1,-1)\)을 얻습니다.
\item
  \(\lambda=3\)에 대해: \((A-3I)\mathbf{v} = 0\)을 풀면, \((1,1)\)을 얻습니다.
\end{itemize}

\begin{enumerate}
\def\labelenumi{\arabic{enumi}.}
\item
  고유벡터를 정규화합니다:
\end{enumerate}

\[\mathbf{u}_1 = \tfrac{1}{\sqrt{2}}(1,-1), \quad \mathbf{u}_2 = \tfrac{1}{\sqrt{2}}(1,1).\]

\begin{enumerate}
\def\labelenumi{\arabic{enumi}.}
\item
  그러면
\end{enumerate}

\[Q =
\begin{bmatrix}
\tfrac{1}{\sqrt{2}} & \tfrac{1}{\sqrt{2}} \[6pt] -\tfrac{1}{\sqrt{2}} & \tfrac{1}{\sqrt{2}}
\end{bmatrix}, \quad
\Lambda =
\begin{bmatrix}
1 & 0 \\
0 & 3
\end{bmatrix}.\]

그래서

\[A = Q \Lambda Q^T.\]

\subsubsection{기하학적 해석}\label{geometric-interpretation-22}

스펙트럼 정리는 모든 대칭 행렬이 직교 방향을 따라 독립적인 스케일링처럼 작용한다고 말합니다. 기하학에서, 이것은 수직 축을 따라 공간을 늘리는 것에 해당합니다.

\begin{itemize}
\item
  타원, 타원체, 이차 곡면은 고유값과 고유벡터를 통해 완전히 이해될 수 있습니다.
\item
  직교성은 변환 후에도 방향이 수직을 유지하도록 보장합니다.
\end{itemize}

\subsubsection{응용}\label{applications-2}

\begin{itemize}
\item
  최적화: 스펙트럼 정리는 헤세 행렬의 고유값을 통한 임계점 분류의 기초가 됩니다.
\item
  PCA (주성분 분석): 데이터 공분산 행렬은 대칭이며, PCA는 최대 분산의 직교 방향을 찾습니다.
\item
  미분 방정식 및 물리학: 대칭 연산자는 실수 고유값을 가진 측정 가능한 양에 해당합니다 (안정성, 에너지).
\end{itemize}

\subsubsection{이것이 중요한 이유}\label{why-this-matters-34}

스펙트럼 정리는 대칭 행렬이 가능한 한 간단함을 보장합니다: 항상 실수, 직교 고유벡터로 분석될 수 있습니다. 이것은 깊은 이론적 통찰력과 강력한 계산 도구를 모두 제공합니다.

\subsubsection{연습문제 9.3}\label{exercises-93}

\begin{enumerate}
\def\labelenumi{\arabic{enumi}.}
\item
  대각화하십시오.

  \[A = \begin{bmatrix} 4 & 2 \\ 2 & 3 \end{bmatrix}\]

  스펙트럼 정리를 사용하여.
\item
  실수 대칭 행렬의 모든 고유값이 실수임을 증명하십시오.
\item
  대칭 행렬의 서로 다른 고유값에 해당하는 고유벡터가 직교함을 보이십시오.
\item
  스펙트럼 정리가 이차 형식에 의해 정의된 타원체를 어떻게 설명하는지 기하학적으로 설명하십시오.
\item
  공분산 행렬에 스펙트럼 정리를 적용하십시오.

  \[\Sigma = \begin{bmatrix} 2 & 1 \\ 1 & 2 \end{bmatrix},\]

  그리고 고유벡터를 분산의 주 방향으로 해석하십시오.
\end{enumerate}

\subsection{9.4 주성분 분석 (PCA)}\label{94-principal-component-analysis-pca}

주성분 분석(PCA)은 데이터 과학, 기계 학습, 통계학에서 널리 사용되는 기법입니다. 핵심적으로, PCA는 공분산 행렬에 대한 스펙트럼 정리의 응용입니다: 데이터에서 최대 분산을 포착하는 직교 방향(주성분)을 찾습니다.

\subsubsection{아이디어}\label{the-idea-2}

벡터 데이터셋 \(\mathbf{x}_1, \mathbf{x}_2, \dots, \mathbf{x}_m \in \mathbb{R}^n\)이 주어지면:

\begin{enumerate}
\def\labelenumi{\arabic{enumi}.}
\item
  평균 벡터 \(\bar{\mathbf{x}}\)를 빼서 데이터를 중심화합니다.
\item
  공분산 행렬을 형성합니다.

  \[\Sigma = \frac{1}{m} \sum_{i=1}^m (\mathbf{x}_i - \bar{\mathbf{x}})(\mathbf{x}_i - \bar{\mathbf{x}})^T.\]
\item
  스펙트럼 정리를 적용합니다: \(\Sigma = Q \Lambda Q^T\).

  \begin{itemize}
  \item
    \(Q\)의 열은 정규직교 고유벡터(주 방향)입니다.
  \item
    \(\Lambda\)의 고유값은 각 방향에 의해 설명되는 분산을 측정합니다.
  \end{itemize}
\end{enumerate}

첫 번째 주성분은 가장 큰 고유값에 해당하는 고유벡터입니다. 최대 분산의 방향입니다.

\subsubsection{예제 9.4.1}\label{example-941}

대략 선 \(y = x\)를 따라 정렬된 2차원 데이터 포인트가 있다고 가정합시다. 공분산 행렬은 대략

\[\Sigma =
\begin{bmatrix}
2 & 1.9 \\
1.9 & 2
\end{bmatrix}.\]

입니다. 고유값은 약 \$3.9\)와 \$0.1\)입니다. \(\lambda = 3.9\)에 대한 고유벡터는 대략 \((1,1)/\sqrt{2}\)입니다.

\begin{itemize}
\item
  첫 번째 주성분: 선 \(y = x\).
\item
  대부분의 분산은 이 방향을 따라 있습니다.
\item
  두 번째 성분은 거의 직교(\(y = -x\))하지만, 그곳의 분산은 매우 작습니다.
\end{itemize}

따라서 PCA는 데이터를 본질적으로 1차원으로 축소합니다.

\subsubsection{PCA의 응용}\label{applications-of-pca}

\begin{enumerate}
\def\labelenumi{\arabic{enumi}.}
\item
  차원 축소: 대부분의 분산을 유지하면서 더 적은 특징으로 데이터를 표현합니다.
\item
  노이즈 감소: 작은 고유값은 노이즈에 해당합니다. 이를 버리면 데이터가 필터링됩니다.
\item
  시각화: 고차원 데이터를 상위 2개 또는 3개의 주성분으로 투영하면 구조가 드러납니다.
\item
  압축: PCA는 이미지 및 신호 압축에 사용됩니다.
\end{enumerate}

\subsubsection{스펙트럼 정리와의 연결}\label{connection-to-the-spectral-theorem}

공분산 행렬 \(\Sigma\)는 항상 대칭이며 양의 준정부호입니다. 따라서 스펙트럼 정리에 의해, 정규직교 고유벡터 기저와 음이 아닌 실수 고유값을 가집니다. PCA는 이 고유기저에서 데이터를 다시 표현하는 것 이상이 아닙니다.

\subsubsection{이것이 중요한 이유}\label{why-this-matters-35}

PCA는 추상적인 선형대수학이 현대 응용 프로그램을 어떻게 직접적으로 구동하는지 보여줍니다. 고유값과 고유벡터는 데이터를 단순화하고, 패턴을 드러내고, 복잡성을 줄이는 실용적인 방법을 제공합니다. 스펙트럼 정리에서 파생된 가장 중요한 알고리즘 중 하나입니다.

\subsubsection{연습문제 9.4}\label{exercises-94}

\begin{enumerate}
\def\labelenumi{\arabic{enumi}.}
\item
  공분산 행렬이 대칭이며 양의 준정부호임을 보이십시오.
\item
  데이터셋 \((1,2), (2,3), (3,4)\)의 공분산 행렬을 계산하고, 그 고유값과 고유벡터를 찾으십시오.
\item
  첫 번째 주성분이 최대 분산을 포착하는 이유를 설명하십시오.
\item
  이미지 압축에서, PCA가 상위 \(k\)개의 주성분만 유지하여 저장 공간을 어떻게 줄일 수 있는지 설명하십시오.
\item
  공분산 행렬의 고유값의 합이 데이터셋의 총 분산과 같음을 증명하십시오.
\end{enumerate}

\section{10장. 실제 선형대수학}\label{chapter-10-linear-algebra-in-practice}

\subsection{10.1 컴퓨터 그래픽스 (회전, 투영)}\label{101-computer-graphics-rotations-projections}

선형대수학은 현대 컴퓨터 그래픽스의 언어입니다. 화면에 렌더링되는 모든 이미지, 회전되거나 투영되는 모든 3D 모델은 궁극적으로 벡터에 행렬을 적용한 결과입니다. 회전, 반사, 스케일링, 투영은 모두 선형 변환이므로, 행렬은 기하학을 조작하는 자연스러운 도구입니다.

\subsubsection{2D 회전}\label{rotations-in-2d}

평면에서 각도 \(\theta\)만큼 시계 반대 방향으로 회전하는 것은 다음과 같이 표현됩니다.

\[R_\theta =
\begin{bmatrix}
\cos\theta & -\sin\theta \\
\sin\theta & \cos\theta
\end{bmatrix}.\]

임의의 벡터 \(\mathbf{v} \in \mathbb{R}^2\)에 대해, 회전된 벡터는

\[\mathbf{v}' = R_\theta \mathbf{v}.\]

입니다. \(R_\theta\)는 행렬식이 \$1\$인 직교 행렬이므로, 길이와 각도를 보존합니다.

\subsubsection{3D 회전}\label{rotations-in-3d}

3차원에서 회전은 행렬식이 \$1\$인 \$3 \textbackslash times 3\$ 직교 행렬로 표현됩니다. 예를 들어, \(z\)-축에 대한 회전은

\[R_z(\theta) =
\begin{bmatrix}
\cos\theta & -\sin\theta & 0 \\
\sin\theta & \cos\theta & 0 \\
0 & 0 & 1
\end{bmatrix}.\]

입니다. \(x\)- 및 \(y\)-축에 대한 회전에 대한 유사한 공식이 존재합니다.

더 일반적인 3D 회전은 축-각도 표현 또는 쿼터니언으로 설명될 수 있지만, 기본 아이디어는 여전히 행렬로 표현되는 선형 변환입니다.

\subsubsection{투영}\label{projections-2}

3D 객체를 2D 화면에 표시하기 위해 투영을 사용합니다:

\begin{enumerate}
\def\labelenumi{\arabic{enumi}.}
\item
  직교 투영: \(z\)-좌표를 버리고, \((x,y,z) \mapsto (x,y)\)로 매핑합니다.

  \[P = \begin{bmatrix}
  1 & 0 & 0 \\
  0 & 1 & 0
  \end{bmatrix}.\]
\item
  원근 투영: 카메라의 효과를 모방합니다. 점 \((x,y,z)\)는

  \[\left(\frac{x}{z}, \frac{y}{z}\right),\]

  로 투영되어, 멀리 있는 객체가 더 작게 보이는 것을 포착합니다.
\end{enumerate}

이러한 연산은 선형(직교 투영)이거나 거의 선형(원근 투영은 동차 좌표에서 선형이 됨)입니다.

\subsubsection{동차 좌표}\label{homogeneous-coordinates}

평행 이동과 투영을 선형 변환과 통합하기 위해, 컴퓨터 그래픽스는 동차 좌표를 사용합니다. 3D 점 \((x,y,z)\)는 4D 벡터 \((x,y,z,1)\)로 표현됩니다. 변환은 그런 다음 \$4 \textbackslash times 4\$ 행렬로, 회전, 스케일링, 평행 이동을 단일 프레임워크에서 나타낼 수 있습니다.

예: \((a,b,c)\)에 의한 평행 이동:

\[T = \begin{bmatrix}
1 & 0 & 0 & a \\
0 & 1 & 0 & b \\
0 & 0 & 1 & c \\
0 & 0 & 0 & 1
\end{bmatrix}.\]

\subsubsection{기하학적 해석}\label{geometric-interpretation-23}

\begin{itemize}
\item
  회전은 모양과 크기를 보존하고, 방향만 변경합니다.
\item
  투영은 차원을 줄입니다: 3D 월드 공간에서 2D 화면 공간으로.
\item
  동차 좌표를 사용하면 여러 변환(회전 + 평행 이동 + 투영)을 단일 행렬 곱셈으로 결합할 수 있습니다.
\end{itemize}

\subsubsection{이것이 중요한 이유}\label{why-this-matters-36}

선형대수학은 모든 실시간 그래픽스를 가능하게 합니다: 비디오 게임, 시뮬레이션, CAD 소프트웨어, 영화 효과. 간단한 행렬 연산을 연결함으로써, 복잡한 변환이 초당 수백만 개의 점에 효율적으로 적용됩니다.

\subsubsection{연습문제 10.1}\label{exercises-101}

\begin{enumerate}
\def\labelenumi{\arabic{enumi}.}
\item
  \(\mathbb{R}^2\)에서 90° 시계 반대 방향 회전에 대한 회전 행렬을 쓰십시오. \((1,0)\)에 적용하십시오.
\item
  점 \((1,1,0)\)을 \(z\)-축에 대해 180° 회전하십시오.
\item
  임의의 2D 또는 3D 회전 행렬의 행렬식이 1임을 보이십시오.
\item
  \(\mathbb{R}^3\)에서 \(xy\)-평면으로의 직교 투영 행렬을 유도하십시오.
\item
  동차 좌표가 평행 이동을 행렬 곱셈으로 표현할 수 있게 하는 방법을 설명하십시오.
\end{enumerate}

\subsection{10.2 데이터 과학 (차원 축소, 최소 제곱)}\label{102-data-science-dimensionality-reduction-least-squares}

선형대수학은 많은 데이터 과학 기법의 기초를 제공합니다. 가장 중요한 두 가지는 고차원 데이터셋을 필수 정보를 보존하면서 압축하는 차원 축소와, 회귀 및 모델 피팅의 기초가 되는 최소 제곱 방법입니다.

\subsubsection{차원 축소}\label{dimensionality-reduction}

고차원 데이터는 종종 중복성을 포함합니다: 많은 특징이 상관 관계가 있으며, 이는 데이터가 본질적으로 더 낮은 차원의 부분 공간 근처에 있음을 의미합니다. 차원 축소는 이러한 부분 공간을 식별합니다.

\begin{itemize}
\item
  PCA (주성분 분석):\\
  앞서 소개했듯이, PCA는 데이터의 공분산 행렬을 대각화합니다.

  \begin{itemize}
  \item
    고유벡터(주성분)는 최대 분산의 직교 방향을 정의합니다.
  \item
    고유값은 각 방향을 따라 얼마나 많은 분산이 있는지를 측정합니다.
  \item
    상위 \(k\)개의 성분만 유지하면 데이터를 \(n\)-차원 공간에서 \(k\)-차원 공간으로 축소하면서 대부분의 변동성을 유지합니다.
  \end{itemize}
\end{itemize}

예제 10.2.1. 각각 1024 픽셀을 가진 1000개의 이미지 데이터셋은 공분산 행렬의 단 50개의 고유벡터에 의해 대부분의 분산이 포착될 수 있습니다. 이러한 성분으로 투영하면 필수 특징을 보존하면서 데이터를 압축합니다.

\subsubsection{최소 제곱}\label{least-squares}

종종, 우리는 미지수보다 방정식이 더 많은 과결정 시스템을 가집니다:

\[A\mathbf{x} \approx \mathbf{b}, \quad A \in \mathbb{R}^{m \times n}, \ m > n.\]

정확한 해가 존재하지 않을 수 있습니다. 대신, 우리는 오차를 최소화하는 \(\mathbf{x}\)를 찾습니다.

\[\|A\mathbf{x} - \mathbf{b}\|^2.\]

이것은 정규 방정식으로 이어집니다:

\[A^T A \mathbf{x} = A^T \mathbf{b}.\]

해는 \(A\)의 열 공간 위로의 \(\mathbf{b}\)의 직교 투영입니다.

\subsubsection{예제 10.2.2}\label{example-1022}

데이터 포인트 \((x_i, y_i)\)에 선 \(y = mx + c\)를 맞춥니다.

행렬 형태:

\[A = \begin{bmatrix}
x_1 & 1 \\
x_2 & 1 \\
\vdots & \vdots \\
x_m & 1
\end{bmatrix},
\quad
\mathbf{b} =
\begin{bmatrix}
y_1 \\
y_2 \\
\vdots \\
y_m \end{bmatrix},
\quad
\mathbf{x} =
\begin{bmatrix}
m \\
c \end{bmatrix}.\]

\(A^T A \mathbf{x} = A^T \mathbf{b}\)를 풉니다. 이것은 최소 제곱 의미에서 최적의 선을 산출합니다.

\subsubsection{기하학적 해석}\label{geometric-interpretation-24}

\begin{itemize}
\item
  차원 축소: 대부분의 분산을 포착하는 최상의 부분 공간을 찾습니다.
\item
  최소 제곱: 대상 벡터를 예측 변수에 의해 생성된 부분 공간으로 투영합니다.
\end{itemize}

둘 다 내적과 직교성을 사용하여 해결되는 투영 문제입니다.

\subsubsection{이것이 중요한 이유}\label{why-this-matters-37}

차원 축소는 대규모 데이터셋을 다루기 쉽게 만들고, 노이즈를 필터링하고, 구조를 드러냅니다. 최소 제곱 피팅은 회귀, 통계학, 기계 학습을 구동합니다. 둘 다 선형대수학의 핵심 도구인 고유값, 고유벡터, 투영에 직접적으로 의존합니다.

\subsubsection{연습문제 10.2}\label{exercises-102}

\begin{enumerate}
\def\labelenumi{\arabic{enumi}.}
\item
  PCA가 작은 고유값 성분을 버림으로써 데이터셋의 노이즈를 줄이는 이유를 설명하십시오.
\item
  \((0,0), (1,1), (2,2)\)를 통과하는 선을 맞추는 최소 제곱 해를 계산하십시오.
\item
  최소 제곱 해가 \(A^T A\)가 가역일 때만 고유함을 보이십시오.
\item
  최소 제곱 해가 투영 인수를 통해 제곱 오차를 최소화함을 증명하십시오.
\item
  데이터 포인트 \((1,0), (2,1), (3,2)\)에 PCA를 적용하고 첫 번째 주성분을 찾으십시오.
\end{enumerate}

\subsection{10.3 네트워크와 마르코프 연쇄}\label{103-networks-and-markov-chains}

그래프와 네트워크는 선형대수학이 생생하게 살아나는 자연스러운 설정입니다. 흐름과 연결성을 모델링하는 것부터 장기적인 행동을 예측하는 것까지, 행렬은 네트워크 구조를 대수적 형태로 변환합니다. 섹션 8.4에서 이미 소개된 마르코프 연쇄는 시간이 지남에 따라 진화하는 네트워크의 중심 예입니다.

\subsubsection{인접 행렬}\label{adjacency-matrices}

\(n\)개의 노드를 가진 네트워크(그래프)는 인접 행렬 \(A \in \mathbb{R}^{n \times n}\)로 표현될 수 있습니다:

\[A_{ij} =
\begin{cases}
1 & \text{if there is an edge from node \(i\) to node \(j\)} \\
0 & \text{otherwise.}
\end{cases}\]

가중 그래프의 경우, 항목은 \$0/1\$ 대신 양의 가중치일 수 있습니다.

\begin{itemize}
\item
  노드 \(i\)에서 노드 \(j\)까지 길이 \(k\)의 경로 수는 항목 \((A^k)_{ij}\)에 의해 주어집니다.
\item
  따라서 인접 행렬의 거듭제곱은 시간 경과에 따른 연결성을 인코딩합니다.
\end{itemize}

\subsubsection{라플라시안 행렬}\label{laplacian-matrices}

또 다른 중요한 행렬은 그래프 라플라시안입니다:

\[L = D - A,\]

여기서 \(D\)는 대각 차수 행렬입니다 (\(D_{ii} = \text{degree}(i)\)).

\begin{itemize}
\item
  \(L\)은 대칭이며 양의 준정부호입니다.
\item
  가장 작은 고유값은 항상 \$0\)이며, 고유벡터는 \((1,1,\textbackslash dots,1)\)입니다.
\item
  고유값 \$0\)의 중복도는 그래프의 연결된 구성 요소의 수와 같습니다.
\end{itemize}

고유값과 연결성 사이의 이 연결은 스펙트럼 그래프 이론의 기초를 형성합니다.

\subsubsection{그래프에서의 마르코프 연쇄}\label{markov-chains-on-graphs}

마르코프 연쇄는 그래프에서의 무작위 보행으로 볼 수 있습니다. 만약 \(P\)가 노드 \(i\)에서 노드 \(j\)로 이동할 확률이 \(P_{ij}\)인 전이 행렬이면,

\[\mathbf{x}_{k+1} = P \mathbf{x}_k\]

는 \(k\) 단계 후의 위치 분포를 설명합니다.

\begin{itemize}
\item
  정상 상태 분포는 고유값 \$1\)을 가진 \(P\)의 고유벡터에 의해 주어집니다.
\item
  수렴 속도는 가장 큰 고유값(항상 \$1\$)과 두 번째로 큰 고유값 사이의 간격에 따라 달라집니다.
\end{itemize}

\subsubsection{예제 10.3.1}\label{example-1031}

간단한 3-노드 순환 그래프를 고려하십시오:

\[P = \begin{bmatrix}
0 & 1 & 0 \\
0 & 0 & 1 \\
1 & 0 & 0
\end{bmatrix}.\]

이 마르코프 연쇄는 노드 사이를 결정적으로 순환합니다. 고유값은 1의 세제곱근입니다: \$1, e\^{}\{2\textbackslash pi i/3\}, e\^{}\{4\textbackslash pi i/3\}\$. 고유값 \$1\)은 정상 상태에 해당하며, 이는 균일 분포 \((1/3,1/3,1/3)\)입니다.

\subsubsection{응용}\label{applications-3}

\begin{itemize}
\item
  검색 엔진: 구글의 페이지랭크 알고리즘은 웹을 마르코프 연쇄로 모델링하며, 여기서 정상 상태 확률은 페이지 순위를 매깁니다.
\item
  네트워크 분석: 인접 또는 라플라시안 행렬의 고유값은 커뮤니티, 병목 현상, 견고성을 드러냅니다.
\item
  역학 및 정보 흐름: 무작위 보행은 질병이나 아이디어가 네트워크를 통해 어떻게 퍼지는지 모델링합니다.
\end{itemize}

\subsubsection{이것이 중요한 이유}\label{why-this-matters-38}

선형대수학은 네트워크 문제를 행렬 문제로 변환합니다. 고유값과 고유벡터는 연결성, 흐름, 안정성, 장기적 동역학을 드러냅니다. 네트워크는 소셜 미디어, 생물학, 금융, 인터넷 등 모든 곳에 있으므로 이러한 도구는 필수 불가결합니다.

\subsubsection{연습문제 10.3}\label{exercises-103}

\begin{enumerate}
\def\labelenumi{\arabic{enumi}.}
\item
  4개의 노드를 가진 정사각형 그래프의 인접 행렬을 쓰십시오. \(A^2\)를 계산하고 항목을 해석하십시오.
\item
  연결된 그래프의 라플라시안이 정확히 하나의 0 고유값을 가짐을 보이십시오.
\item
  다음 마르코프 연쇄의 정상 상태 분포를 찾으십시오.

  \[P = \begin{bmatrix} 0.5 & 0.5 \\ 0.4 & 0.6 \end{bmatrix}.\]
\item
  라플라시안의 고유값이 그래프의 연결되지 않은 구성 요소를 어떻게 감지할 수 있는지 설명하십시오.
\item
  페이지랭크가 웹 그래프의 전이 행렬을 어떻게 수정하여 고유한 정상 상태 분포를 보장하는지 설명하십시오.
\end{enumerate}

\subsection{10.4 기계 학습 연결}\label{104-machine-learning-connections}

현대 기계 학습은 선형대수학 위에 구축됩니다. 데이터를 행렬로 표현하는 것부터 대규모 모델의 최적화에 이르기까지, 거의 모든 단계가 벡터 공간, 투영, 고유값, 행렬 분해와 같은 개념에 의존합니다.

\subsubsection{행렬로서의 데이터}\label{data-as-matrices}

\(m\)개의 예제와 \(n\)개의 특징을 가진 데이터셋은 행렬 \(X \in \mathbb{R}^{m \times n}\)으로 표현됩니다:

\[X =
\begin{bmatrix}
\- & \mathbf{x}_1^T & - \\
\- & \mathbf{x}_2^T & - \\
  & \vdots & \\
\- & \mathbf{x}_m^T & -
\end{bmatrix},\]

여기서 각 행 \(\mathbf{x}_i \in \mathbb{R}^n\)는 특징 벡터입니다. 선형대수학은 이 데이터를 분석, 압축, 변환하는 도구를 제공합니다.

\subsubsection{선형 모델}\label{linear-models}

기계 학습의 핵심에는 선형 예측 변수가 있습니다:

\(\hat{y} = X\mathbf{w},\)

여기서 \(\mathbf{w}\)는 가중치 벡터입니다. 훈련은 종종 최소 제곱 문제 또는 릿지 회귀와 같은 정규화된 변형을 해결하는 것을 포함합니다:

\(\min_{\mathbf{w}} \|X\mathbf{w} - \mathbf{y}\|^2 + \lambda \|\mathbf{w}\|^2.\)

이것은 행렬 분해를 사용하여 효율적으로 해결됩니다.

\subsubsection{특이값 분해 (SVD)}\label{singular-value-decomposition-svd}

행렬 \(X\)의 SVD는

\(X = U \Sigma V^T,\)

이며, 여기서 \(U, V\)는 직교이고 \(\Sigma\)는 음이 아닌 항목(특이값)을 가진 대각 행렬입니다.

\begin{itemize}
\item
  특이값은 특징 공간에서 방향의 중요성을 측정합니다.
\item
  SVD는 차원 축소(저랭크 근사), 토픽 모델링, 추천 시스템에 사용됩니다.
\end{itemize}

\subsubsection{기계 학습에서의 고유값}\label{eigenvalues-in-machine-learning}

\begin{itemize}
\item
  PCA (주성분 분석): 공분산 행렬의 대각화는 최대 분산의 방향을 식별합니다.
\item
  스펙트럼 클러스터링: 라플라시안의 고유벡터를 사용하여 데이터 포인트를 클러스터로 그룹화합니다.
\item
  안정성 분석: 헤세 행렬의 고유값은 최적화가 최소값으로 수렴하는지 여부를 결정합니다.
\end{itemize}

\subsubsection{신경망}\label{neural-networks}

비선형이지만 딥 러닝조차도 핵심에 선형대수학을 사용합니다:

\begin{itemize}
\item
  각 계층은 행렬 곱셈과 비선형 활성화 함수로 구성됩니다.
\item
  훈련에는 행렬 미적분으로 표현되는 기울기 계산이 필요합니다.
\item
  역전파는 본질적으로 선형대수학을 사용한 연쇄 법칙의 반복적인 적용입니다.
\end{itemize}

\subsubsection{이것이 중요한 이유}\label{why-this-matters-39}

기계 학습 모델은 종종 수백만 개의 특징과 매개변수를 포함합니다. 선형대수학은 훈련과 추론을 가능하게 하는 알고리즘과 추상화를 제공합니다. 그것 없이는 AI의 대규모 계산은 다루기 어려울 것입니다.

\subsubsection{연습문제 10.4}\label{exercises-104}

\begin{enumerate}
\def\labelenumi{\arabic{enumi}.}
\item
  릿지 회귀가 정규 방정식으로 이어짐을 보이십시오.
\end{enumerate}

\[(X^T X + \lambda I)\mathbf{w} = X^T \mathbf{y}.\]

\begin{enumerate}
\def\labelenumi{\arabic{enumi}.}
\item
  SVD가 픽셀 강도의 행렬로 표현된 이미지를 어떻게 압축할 수 있는지 설명하십시오.
\item
  공분산 행렬 \(\Sigma\)에 대해, 그 고유값이 주성분을 따른 분산을 나타내는 이유를 보이십시오.
\item
  라플라시안 행렬의 고유벡터가 작은 그래프를 클러스터링하는 데 어떻게 사용될 수 있는지 예를 들어 설명하십시오.
\item
  하나의 은닉 계층을 가진 신경망에서 순방향 패스를 행렬 형태로 쓰십시오.
\end{enumerate}

\end{document}
